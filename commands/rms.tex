\SACCMD{rms}
\label{cmd:rms}

\SACTitle{概要}
计算测量时间窗的信号的均方根

\SACTitle{语法}
\begin{SACSTX}
RMS [NOISE ON|OFF|pdw] [TO USERn]
\end{SACSTX}

\SACTitle{输入}
\begin{description}
\item [NOISE ON] 打开噪声归一化选项
\item [NOISE OFF] 关闭噪声归一化选项
\item [NOISE pdw] 打开噪声归一化选项并改变噪声的部分数据窗。关于pdw,参考\nameref{subsec:pdw}一节。
\item [TO USERn] 定义用于储存结果的头段变量USERn,其中n取0到9
\end{description}

\SACTitle{缺省值}
\begin{SACDFT}
rms noise off to user0
\end{SACDFT}

\SACTitle{说明}
该命令计算当前测量时间窗(参见 \nameref{cmd:mtw})中的数据的均方根。结果写入一个
浮点型头段变量 \texttt{USERn} 中,如果定义了一个噪声时间窗结果可以用于纠正噪声。
计算的一般形式是:对信号窗做一次求和并对可选的噪声窗做另一次求和。

\SACTitle{示例}
为了计算两个头段T1和T2间的数据的未修正的均方根,并将结果保存在头段USER4中:
\begin{SACCode}
SAC> mtw t1 t2
SAC> rms to user4
\end{SACCode}

使用一个5秒长的噪声窗(结束于头段值T3),并就算修正后的均方根:
\begin{SACCode}
SAC> mtw t1 t2
SAC> rms noise t3 -5.0 0.0
\end{SACCode}

\SACTitle{头段变量改变}
USERn

\SACTitle{相关命令}
\nameref{cmd:mtw}、\nameref{cmd:cut}