\SACCMD{setbb}
\label{cmd:setbb}

\SACTitle{概要}
设置黑板变量的值

\SACTitle{语法}
\begin{SACSTX}
SETBB variable [APPEND] value [variable [APPEND] value ...]
\end{SACSTX}

\SACTitle{输入}
\begin{description}
\item [variable] 黑板变量名,可以是一个新变量或一个已经有值的变量,
    变量名最长32字符
\item [value] 黑板变量的新值,若包含空格则必须用引号括起来
\item [APPEND] 将值加到变量的旧值之后,若无该选项,则新值将代替旧值
\end{description}

\SACTitle{说明}
\texttt{setbb} 命令可以给黑板变量赋值,这些值可以通过 \nameref{cmd:getbb}
命令获取,或在命令中直接引用。可以使用 \nameref{cmd:evaluate} 对黑板变量
做基本算术操作,并将结果保存在新的黑板变量中,也可以通过 \nameref{cmd:unsetbb}
命令删除一个黑板变量。

\SACTitle{示例}
同时设置多个黑板变量,并在稍后使用使用这些黑板变量:
\begin{SACCode}
SAC> setbb c1 2.45 c2 4.94
SAC> bandpass corners %c1% %c2%
\end{SACCode}

黑板变量的值中包含空格:
\begin{SACCode}
                            // 含空格的值需要用引号括起来
SAC> setbb mytitle 'sample filter response'
SAC> getbb mytitle          // 检查变量值是否正确
 MYTITLE = Sample filter response
SAC> title '%MYTITLE%'      // 引用时需要再次用引用括起来
\end{SACCode}
