\SACCMD{plot2}
\label{cmd:plot2}

\SACTitle{概要}
产生一个多波形单窗口绘图

\SACTitle{语法}
\begin{SACSTX}
P!LOT!2 [A!BSOLUTE!|R!ELATIVE!]
\end{SACSTX}

\SACTitle{输入}
\begin{description}
\item [ABSOLUTE|RELATIVE] 绝对/相对绘图模式
\end{description}

\SACTitle{缺省值}
\begin{SACDFT}
p2 absolute
\end{SACDFT}

\SACTitle{说明}
所有波形数据将绘制在一个绘图窗口中,所有波形共用X轴和Y轴。\nameref{cmd:xlim}
和 \nameref{cmd:ylim} 用于控制X轴和Y轴的范围,若未指定轴范围,则根据所有
文件的极值确定轴范围。由于所有波形重叠在一起,所以通常需要用 \nameref{cmd:line}、
\nameref{cmd:color}、\nameref{cmd:symbol}、\nameref{cmd:width} 为不同波形
设置不同的属性,图例的位置由 \nameref{cmd:fileid} 控制。

多个波形共用X轴时,有 !absolute! 和 !relative! 两种绘图
模式。在 !absolute! 模式下,所有波形将按照其绝对时刻对齐,即X轴
表示相对于某个固定参考时刻的秒数;在 !relative! 模式下,所有数据
将按照文件开始时间对齐,X轴的范围为0到最大时间差(所有文件中结束时间和
开始时间的最大时间差,即最大 !e-b!),每个波形从X轴的零点开始绘制。

与 \nameref{cmd:plot} 和 \nameref{cmd:plot1} 不同,\nameref{cmd:plot2}
可以直接用于绘制谱文件。实部-虚部格式的谱数据被绘制为实部-频率。振幅-相位
数据被绘制为振幅-频率图,虚部和相位信息被忽略。谱文件总是用 !relative!
模式绘图。在频率域 !b!、!e! 和 !delta! 分别被设置为0、
Nqquist频率以及频率间隔 !df!。头段值 !depmin! 和 !depmax!
未改变。与 \nameref{cmd:plotsp} 类似,若 !xlim off!,绘图将从
!df=delta! 处开始,而非0。若 \nameref{cmd:xlim} 或 \nameref{cmd:ylim}
在数据变换到频率域之前被改变了,最好在使用 !plot2! 绘图之前输入
!xlim off! 和 !ylim off!。

若内存中同时存在时间序列文件和谱文件,且绘图时没有选择 !relative!
模式,则时间序列将以绝对模式绘图,谱文件将以相对模式绘图绘图。相对
模式意味着相对于第一个文件,因而内存中文件的顺序将影响绘图之间的关系。

\SACTitle{示例}
\begin{SACCode}
SAC> read mnv.z.am knb.z.am elk.z.am
SAC> xlim 0.04 0.16
SAC> ylim 0.0001 0.006
SAC> linlog
SAC> symbol 2 increment
SAC> title 'rayleigh wave amplitude spectra for nessel'
SAC> xlabel 'frequency (Hz)'
SAC> plot2
SAC> fft
SAC> xlim off ylim off
SAC> line increment list 1 3
SAC> plot2
\end{SACCode}
