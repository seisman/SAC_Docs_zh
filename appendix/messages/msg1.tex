\pdfbookmark[1]{E1xxx}{E1xxx}
\SACTitle{1106 Not a valid SAC command.}
对于每一行命令,SAC首先会检查是否是SAC内部的命令,如果不是,则检查是否是系统自带
的命令,比如~\verb+ls+、\verb+cp+等。

一个例外是系统命令~\verb+rm+。在SAC中直接执行rm命令会出现如上所示的错误。出现该
错误的原因是SAC禁用了系统命令~\verb+rm+,主要目的是为了防止将~\verb+r *.SAC+误敲成
\verb+rm *.SAC+而导致数据的误删除。可以使用~\nameref{cmd:systemcommand}~命令显式
调用系统命令,如下:
\begin{SACCode}
SAC> rm BAD*.SAC
 ERROR 1106: Not a valid SAC command.
SAC> sc rm BAD*.SAC
\end{SACCode}

\SACTitle{1303 Overwrite flag is not on for file}
该错误主要出现在写SAC文件时,出现该错误的原因是SAC文件的头段变量~\verb+lovrok+的值
为~\verb+FALSE+,即磁盘中的数据不允许被覆盖。解决该问题的方法有两种:
\begin{itemize}
\item 以其他文件名写入磁盘,不覆盖磁盘文件;
\item 修改~\verb+lovrok+的值为~\verb+TRUE+;
\end{itemize}

\SACTitle{1312 Bad number of files in write file list}
通常在使用~\nameref{cmd:write}~命令时会出现该问题。出现该错误的原因是内存中的
波形文件的数目与write命令给出的文件名的数目不想匹配。在该错误信息的后面,紧接着
会给出write命令中给出的文件数目以及内存中的波形数目。

\SACTitle{1322 Undefined starting cut for file}
\nameref{cmd:cut}~命令中时间窗的起始参考头段未定义。默认情况下,会使用磁盘
文件的起始时间代替,也可以使用~\nameref{cmd:cuterr}~命令控制该错误的处理方式。

\SACTitle{1323 Undefined stop cut for file}
\nameref{cmd:cut}~命令中时间窗的结束参考头段未定义。默认情况下,会使用磁盘
文件的结束时间代替,也可以使用~\nameref{cmd:cuterr}~命令控制该错误的处理方式。

\SACTitle{1324 Start cut less than file begin for file}
\nameref{cmd:cut}~命令中时间窗的开始时间早于磁盘文件的开始时间。默认情况下,
会使用磁盘文件的开始时间代替,也可以使用~\nameref{cmd:cuterr}~命令控制该错误的
处理方式。

\SACTitle{1325 Stop cut greater than file end for file}
\nameref{cmd:cut}~命令中时间窗的结束时间晚于磁盘文件的结束时间。默认情况下,
会使用磁盘文件的结束时间代替,也可以使用~\nameref{cmd:cuterr}~命令控制该错误的
处理方式。

\SACTitle{1326 Start cut greater than file end for file}
\nameref{cmd:cut}~命令中时间窗的开始时间晚于文件结束时间。
