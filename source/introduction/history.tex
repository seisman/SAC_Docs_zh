\section{SAC发展史}
\label{sec:history}

\href{http://en.wikipedia.org/wiki/Lawrence\_Livermore\_National\_Laboratory}{Lawrence Livermore国家实验室}
和 \href{http://en.wikipedia.org/wiki/Los\_Alamos\_National\_Laboratory}{Los Alamos国家实验室}
是美国承担核武器设计工作的两个实验室。SAC于20世纪80年代诞生于实验室的
Treaty Verification Program小组里,该组由W. C. Tapley和Joe Tull共同领导。

起初,SAC是用Fortran语言实现的,并将源代码分发给感兴趣的学者,允许用户
进行非商业性的地震数据处理,用户和开发者之间的合作协议要求用户提交bug
修正和改进以换取SAC的使用权。到了大概1990年,SAC已经成为全球地震学家的
数据处理标准软件。

从1992年开始,SAC的开发逐渐由Livermore接管,并开始通过分发协议严格限制
源代码的分发。与此同时,开发者认为Fortran是一种过于局限的编程语言,
其阻碍了SAC特性的进一步开发,因而开发者使用
\href{http://www.netlib.org/f2c/}{f2c}\footnote{Fortran77语言到C语言的自动转换工具。}
转换工具将SAC的Fortran源码转换成了C源码\footnote{个人猜测,目前SAC源码的
混乱和不易读正是由于这次自动转换导致的。}。接下来,Livermore以转换得到的
C源码为基础,计划开发一个商业版的地震数据处理产品,命名为SAC2000。这个
版本扩展了很多功能,其中一个功能是建立一个日志数据库,记录一个波形从
原始数据到最终产品之间的所有处理步骤。这样的设计允许用户随时提交数据处理
的中间结果,也可随时回滚到之前的状态。

约1998年,\href{http://www.iris.edu}{IRIS} 意识到,SAC的核心
用户群(主要是IRIS的成员)无法确保能够获取SAC的源码。IRIS开始和Livermore
协商,希望将SAC的开发分成两条线:一个包含数据库特性,供核监测机构使用;
另一个不包含数据库特性,仅供学术机构使用。商业化的努力主要集中在含数据库
功能的版本上。

终于,在2005年,IRIS与Livermore签订了合同,Livermore提供给IRIS一个SAC
协议,允许其在IRIS社区内部分享SAC/SAC2000的源代码,并提供有限的支持以
促进社区的发展。而学术圈对于商业版的SAC没有太大兴趣,因而Livermore逐渐
撤出了对于SAC2000的支持。最终IRIS完全接手了SAC的开发和技术支持,成为了
一个独立的新分支,也就是本手册中介绍的SAC,有时为了区分,也称之为
SAC/IRIS。
