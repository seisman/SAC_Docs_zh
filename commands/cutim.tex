\SACCMD{cutim}
\label{cmd:cutim}

\SACTitle{概要}
截取内存中的文件\footnote{该命令似乎有问题,执行该命令时,似乎会删除内存中的全部
波形数据,请勿使用}

\SACTitle{语法}
\begin{SACSTX}
CUTIM pdw [pdw ... ]
\end{SACSTX}

\SACTitle{输入}
\begin{description}
\item [pdw] 要截取的时间窗。参考 \nameref{subsec:pdw}
\end{description}

\SACTitle{缺省值}
如果起始或结束offset省略则认为其为0,如果起始参考值省略则认为其为Z,如果结束参考值省略则认为其值与起始参考值相同。

\SACTitle{说明}
cut命令设置截窗选项,仅对即将读取的文件进行截窗,而对内存中的数据没有效果。
cutim则在这个命令给出的时候对内存中的数据进行截窗操作。

用户可以用read读入文件,然后用cutim对内存中的文件直接进行截窗。
cutim也允许使用多个截取区间,用户可以READ三个文件到内存,然后使用有4个截取区间的
CUTIM命令,最终内存中将得到12个文件。

\SACTitle{示例}
下面的宏文件展示了cutim命令的常见用法:
\begin{SACCode}
fg seismo
echo on
* no cutting
lh b e a kztime
* begin to end---same as not cutting.
cutim B E
lh b e a kztime
fg seismo
* First 3 secs of the file.
cutim B 0 3
lh b e a kztime
fg seismo
* From 0.5 secs before to 3 secs after first arrival
cutim A -0.5 3
lh b e a kztime
fg seismo
* From 0.5 to 5 secs relative to disk file start.
cutim 0.5 5
lh b e a kztime
fg seismo
* First 3 secs of the file and next 3 sec
cutim b 0 3 b 3 6
lh b e a kztime
p1
\end{SACCode}

\SACTitle{限制}
目前不支持截取非等间隔数据或谱文件

\SACTitle{相关命令}
\nameref{cmd:read}、\nameref{cmd:apk}、\nameref{cmd:plotpk}、\nameref{cmd:synchronize}、\nameref{cmd:cuterr}
