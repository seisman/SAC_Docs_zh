\SACCMD{symbol}
\label{cmd:symbol}

\SACTitle{概要}
控制符号绘图属性

\SACTitle{语法}
\begin{SACSTX}
SYM!BOL! [ON|OFF|n] [SIZE v] [SPACING v] [INCRE!MENT! ON|OFF]
    [LIST STANDARD|nlist]
\end{SACSTX}

\SACTitle{输入}
\begin{description}
\item [ON] 打开符号绘图开关,不改变符号号
\item [OFF] 关闭符号绘图开关
\item [n] 打开符号绘图开关。将符号号设置为n。目前有16个不同的符号,符号0意味着关闭符号开关
\item [SIZE v] 设置符号尺寸为v。值为0.01意味着占据整个绘图尺寸的1\%
\item [SPACING v] 设置符号间隔为v。这是绘图时符号间的最小间隔;如果你想每个数据点
    都有符号,则设其值为0,对注释行使用0.2到0.4
\item [INCREMENT ON] 对每个数据文件操作完成后按符号列表递增为下一个符号
\item [INCREMENT OFF] 关闭上述 \texttt{INCREMENT} 选项
\item [LIST nlist] 改变符号表的内容。输入符号号码表。设置表中的第一个符号码,并打开符号绘图开关
\item [LIST STANDARD] 改变到标准符号表。设置表中的第一个符号表,并打开符号绘制选项。
\end{description}

\SACTitle{缺省值}
\begin{SACDFT}
symbol off size 0.01 spacing 0. increment off list standard
\end{SACDFT}

\SACTitle{说明}
这些符号属性独立于由 \nameref{cmd:line} 命令定义的画线属性。打开画线选项,
它们也可以用于注释在相同图形上的不同的线。关闭画线选项,则可以绘制散点图。
如果你要将几个数据文件画在同一张图上,也许需要使用不同的符号。这是可以使用
\texttt{INCREMENT} 选项。当这个选项打开时,每次绘制数据文件,都从符号表中
将原来的符号码增加1,缺省符号表包含符号表从2到16的符号,你也可以使用 \texttt{LIST}
选项改变这个表的次序和内容。如果你在绘制一系列重叠绘图,并需要经相同符号用
在相同次序的每个图形上,这样做很有用。符号码为0相当于关闭符号绘制选项。
这个选项用于 \texttt{LIST} 选项和 \texttt{LINE} 选项,以在一张图上用线
表示一些数据,用符号表示另外一些数据。

\SACTitle{示例}
为了创建一个散点分布图,关闭画线选项,选择适当的符号,然后绘图:
\begin{SACCode}
SAC> line off
SAC> symbol 5
SAC> plot
\end{SACCode}

为了用符号7、4、6、8注释四条实线,间隔用0.3,用 \nameref{cmd:plot2} 绘图:
\begin{SACCode}
SAC> line solid
SAC> sym spacing .3 increment list 7 4 6 8
SAC> r file1 file2 file3 file4
SAC> plot2
\end{SACCode}

使用 \nameref{cmd:plot2} 在相同图形上绘制三个文件,第一个文件图形使用
实线无符号;第二个没有线,为三角符号;第三个没有线,带有交叉符号:
\begin{SACCode}
SAC> read file1 file2 file3
SAC> line list 1 0 0 increment
SAC> symbol list 0 3 7 increment
SAC> plot2
\end{SACCode}
