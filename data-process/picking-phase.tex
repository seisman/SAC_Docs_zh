\section{震相拾取}
\label{sec:phase-picking}
相关命令:\nameref{cmd:plotpk}

震相拾取,或者说标定到时,是SAC的一种常用功能。

要进行震相拾取,首先要进入``ppk模式''。
启动SAC,读取波形数据后,在终端中键入~\nameref{cmd:plotpk}(简写为~\verb+ppk+),
就会出现一个绘图窗口。若之前未曾打开过绘图窗口,则此时焦点位于ppk打开的绘图窗口
中;若之前曾经打开过绘图窗口,则需要鼠标点击一下绘图窗口以使得焦点位于绘图窗口
而不是终端中。此时,SAC就进入了``ppk模式''。此时,终端中光标所在行没有SAC
提示符``\verb+SAC> +''。

\begin{SACCode}
SAC> fg seis
SAC> ppk    // 焦点位于绘图窗口中,进入ppk模式
            // 光标所在行没有提示符"SAC> "
\end{SACCode}

学会如何进入ppk模式后,还要学会退出ppk模式。首先,确保焦点位于绘图窗口而不是终端,
然后将光标移动到绘图窗口中,按下``q''键即可退出ppk模式。此时,终端中光标所在行
会重新出现SAC提示符``\verb+SAC> +''。

只有当使用了ppk命令,焦点位于当前绘图窗口,且鼠标位于当前绘图窗口内才称为ppk模式。
在ppk模式下,所有的键盘输入都会被解释为``ppk命令''。若使用ppk命令后,不慎使焦点
位于终端内,此时所有的键盘输入都不会被SAC解释。只有当退出ppk模式时,SAC才会依次
解释终端中输入的命令。

下面介绍如何在ppk模式中拾取震相。先进入ppk模式,此时焦点位于绘图窗口,鼠标位于
绘图区内部,将鼠标移动到要标记到时的地方,依次按下``\verb+t+''、``\verb+0+'',
在要标记的到时处会出现一条竖线,旁边有标识``\verb+T0+'',此时已经将要标记的到时
(即竖线所对应的X轴位置)保存到头段变量~\verb+T0+中。再按下``q''以退出ppk模式,
最后在终端键入``\verb+wh+''将内存中的头段变量写回到磁盘文件中。

除了可以键入``\verb+t+''和``\verb+0+''之外,0还可以用1到9的任意数字替换,分别表示
将要标记的到时保存到T0到T9中。

\begin{SACCode}
SAC> fg seis
SAC> ppk
// 键入``t''和``0''标记到时,然后按``q''退出ppk模式
SAC> wh         // 保存头段
\end{SACCode}

\begin{note}
在键入"t"时,鼠标不仅要在绘图窗口内,还要在绘图区(即四个边框)的内部,否则会
得到"Bad cursor position. Please retry."的错误提示。
\end{note}

下面介绍一些标定到时时的技巧。

首先,建议将SAC全局变量``\verb+SAC_PPK_USE_CROSSHAIRS+''的值设置为~\verb+1+。
该全局变量可以控制ppk模式下鼠标在绘图窗口内的形态。若其量值为~\verb+0+,则鼠标
会以十字线的形式出现,即
``\tikz[scale=0.3]{
    \draw[very thick] (0,0) -- (1,0);
    \draw[very thick] (0.5,-0.5) -- (0.5,0.5);
}'';当其值为~\verb+1+时,会在十字线的基础上加上水平线和垂直线。具体的设置方式
参考``\nameref{sec:sac-install}''一节。

其次,是图幅的放大和缩小。有时数据时间较长,难以精确标定到时,此时需要将图幅放大,以显示整个波形的一小部分。首先需要将光标移动到绘图区域中的某位置,键入``\verb+x+'',
再移动至另一位置,再次键入``\verb+x+''。这样,两次键入确定了一个时间窗。
这时,绘图窗口中将只显示该时间窗内的波形,也就实现了图幅的放大。
可不断重复此步骤,进行多次放大。101.5之后的版本有更方便的方式:在绘图窗口中某位置按下鼠标左键,
并拖动至另一位置再松开鼠标左键,则两个位置之间的时间窗内的波形会被放大。
图幅的缩小通过键入``o''来实现,``o''最多可以回退5次绘图历史。

同一台站的三个分量的数据上的某震相的到时自然是在同一时刻,所以通常将同一台站的三个
分量画在仪器,一方面可以相互比较,利于震相的识别;另一方面也可以实现一次标定三个分量
以减少工作量。使用命令``\verb+ppk p 3 r m+''进入ppk模式即可每次显示并同时标记3个分量
的波形数据。

\begin{SACCode}
SAC> dg sub teleseis ntkl.[nez] nykl.[nez] onkl.[nez] sdkl.[nez]
SAC> ppk p 3 r m
// 键入一次``t0''波形ntkl台站的三分量到时
// 键入``n''以绘制下面的三个数据
// 键入一次``t0''波形nykl台站的三分量到时
// 键入``n''以绘制下面的三个数据
// 键入一次``t0''波形onkl台站的三分量到时
// 键入``n''以绘制下面的三个数据
// 键入一次``t0''波形sdkl台站的三分量到时
// 键入``q''退出ppk模式
SAC> wh
SAC> q
\end{SACCode}
\begin{note}
\verb+ppk p 3 r m+能够每次显示同一台站的三个分量的前提是波形数据在内存中是
按照正确的顺序排序的,即同一台站的三个分量在内存中位于相邻的位置。

若读入数据时,同一台站的三个分量不是紧挨着读取的,或者某个台站丢失了一个
台站的数据,都会影响到该命令的执行效果。
\end{note}

除了上面介绍的若干ppk命令之外,还有很多其他ppk命令。表~\ref{table:plotpk-commands}~列出
了ppk模式下的所有命令,其中常用的命令包括``b''、``l''、``n''、``o''、``q''、``t''和``x''。

\begin{table}[H]
\centering
\small
\ttfamily
\caption{ppk模式命令一览表}
\label{table:plotpk-commands}
\begin{tabular}{cll}
	\toprule
    命令	&	含义	                                &   说明    \\
	\midrule
    a	    &	定义事件初至a                           &   1,7	    \\
    b	    &	如果有,则显示上一张绘图	            &           \\
    c	    &	计算事件的初至和结束                    &   1,4,7	\\
    d	    &	设置震相方向为DOWN	                    &           \\
    e     	&	设置震相起始为EMERGENT(急始)	        &           \\
    f	    &	定义事件结束f                           &  1,2,3,7	\\
    g	    &	以HYPO格式将拾取显示到终端              &   4   	\\
    h   	&	将拾取写成HYPO格式                      &   3,4 	\\
    i	    &	设置震相起始为IMPULSIVE	                &           \\
    j	    &	设置噪声水平                            &   2,6,8	\\
    k       &   即kill,退出ppk模式                     &           \\
    l	    &	显示光标当前位置                        &   2,4	    \\
    m	    &	计算最大振幅波形                        &   2,3,5	\\
    n	    &	显示下一绘图	                        &           \\
    o	    &	显示前一个绘图窗,最多可以保存5个绘图窗	&           \\
    p	    &	定义P波到时                             &   1,2,3,7	\\
    q	    &	即quit,退出ppk模式	                            &           \\
    s	    &	定义S波到时                             &   1,2,3,7 \\
    t	    &	用户自定义到时tn,输入t之后需要输入0到9中的任一数	&   1,2,7\\
    u	    &	设置震相方向为UP	                    &           \\
    v	    &	定义一个Wood-Anderson波形               &   2,5 	\\
    w	    &	定义一个通用波形                        &   2,5 	\\
    x	    &	使用一个新的x轴时间窗,简单说就是放大。 &           \\
    z	    &	设置参考水平                            &   2,6,8	\\
    \textbackslash	    &	删除当前全部拾取的定义。当一个文件中包含多个事件时有用。&	\\
    +	    &	设置震相方向为PLUS	                    &           \\
    -	    &	设置震相方向为MINUS	                    &           \\
    \lstinline[showspaces]! !   &	设置震相方向为NEUTRAL	                &           \\
    n	    &	设置震相质量为n,n取0-4	                &           \\
	\bottomrule
\end{tabular}
\end{table}
注意:ppk模式的命令几乎都是由单个字符组成的,比如退出``q''(唯一的例外是命令``t'',由字符``t''和0-9的整数构成)。

不同的命令效果可能不同,有些会在绘图窗口显示信息,有些会将信息写入头段变量,
下面对表~\ref{table:plotpk-commands}~中的说明进行一个说明:
\begin{description}
    \item [1] 会将信息写入头段变量
    \item [2] 写入字符型震相拾取文件(若已打开)
    \item [3] 写入HYPO格式震相拾取文件(若已打开)
    \item [4] 在绘图窗口中显示信息
    \item [5] 窗口显示包含波形的矩形
    \item [6] 在指定的水平处放置水平光标
    \item [7] 绘图窗口显示含有到时标识的垂直线
    \item [8] 绘图窗口显示含有标识的水平线
\end{description}
