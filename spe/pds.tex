\SACCMD{pds}
\label{spe:pds}

\SACTitle{概要}
用能量密度谱方法计算谱估计

\SACTitle{语法}
\begin{SACSTX}
PDS [S!ECONDS! v|L!AGS! n] [N!UMBER! n]
    [T!YPE! HAM!MING!|HAN!NING!|C!OSINE!|R!ECTANGLE!|T!RIANGLE!]
\end{SACSTX}

\SACTitle{输入}
\begin{description}
\item [SECONDS v] 设置窗长为v秒
\item [LAGS n] 设置窗长为n?
\item [NUMBER n] 设置谱估计使用的数据点数
\item [TYPE type] 设置要使用的窗类型
\end{description}

\SACTitle{缺省值}
\begin{SACDFT}
pds type hamming
\end{SACDFT}

\SACTitle{说明}
该命令实现了传统的谱估计方法。样本相关函数首先进行相关窗截窗,生成的函数
再使用FFT获得谱估计。正如在 \nameref{spe:cor} 命令文档中提到的,在估计偏差
(即丧失分辨率)与估计方差之间存在tradeoff。随着窗长增加,频率域分辨率
增加,进而偏差减少;然而,样本相关函数在大延迟时值较大,谱估计的方差
也会增加,这是由于在大延迟时用于估计的数据点数变少了。

相关窗类型的选取与 \nameref{spe:cor} 文档中描述的数据窗有不同的效果。
这是两种不同类型的偏差之间的选择。

该谱估计方法用相关窗的Fourier变换来逼近真实谱的卷积。窗变换由两个特性控制,
一个是控制分辨率的中心瓣,一个是控制带外能量泄露的旁瓣。通常来说,用户想要
尽可能窄的主瓣以及尽可能小的旁瓣。

大的旁瓣会在谱估计上产生一个虚假的低值,可能会掩盖高动态范围的频谱衰减。
窗类型的选取在主瓣分辨率与旁瓣能量泄露之间存在tradeoff。

矩形窗有最窄的主瓣,因此也具有最好的分辨率,同时有最大的旁瓣。
余弦窗稍微减小了旁瓣,但基本不太影响主瓣宽度。这两种窗主要用于估计
瞬变信号的谱,以获取尽可能小的时间域失真。Hamming和Hanning窗则具有
很小的旁瓣和较宽的主瓣,当用户有大量数据的时候比较有用,可以通过增加
窗的尺寸来控制分辨率。Hamming和Hanning窗都是余弦窗的衍生,但Hamming窗
经过优化以尽可能地减少最大旁瓣的尺寸,因而该命令默认使用Hamming窗。
三角窗也有很好的旁瓣结构,但是它有特别好的性质,即保证了谱估计总是非负值。

通常情况下,当用户需要处理大量数据时,PDS方法要优于另外两种参数化方法。
因为在这种情况下,分辨率不受限制,并且对这种算法的了解要比对其他方法更
多一些。例如,理论上我们可以给出该方法的置信度和分辨率。这些指标包括在
SPE中。参量方法通常显示出比PDS更好的分辨率(尤其是在估计线性谱的时候),
并且在数据量有限时更加有用。

\SACTitle{相关命令}
\nameref{spe:cor}、\nameref{spe:writespe}、\nameref{spe:plotspe}
