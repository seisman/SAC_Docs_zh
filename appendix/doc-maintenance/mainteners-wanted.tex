\section{征集维护者}
从2012年的1.0版到2015年的3.3版,本手册一直由我一人维护并持续更新。鉴于个人精力有限,
因而希望有志愿者参与到这个开源项目中,一起维护并完善该手册。

\subsection{维护的内容}
本手册需要维护的部分可以分为如下几类(按由简到繁、由易到难排序):
\begin{enumerate}
\item 错别字;
\item 个别行或页面的排版问题;
\item 需要解释清楚或易引起歧义的部分;
\item 部分未翻译的命令,以及个别命令中未翻译的部分;
\item 某些命令在不同SAC版本间的语法不兼容;
\item 随着SAC新版本的发布,更新手册中相应的内容;
\item 补充命令示例以及脚本示例;
\item 补充其他尚未包含在手册中的与SAC相关的知识点;
\item 文档结构的调整;
\item 项目源码的优化;
\end{enumerate}

\subsection{参与方式}
想要参与到本手册的维护,有三种方式(从易到难依次为):
\begin{itemize}
\item 通过邮件发送给 \url{seisman.info@gmail.com}
\item 在项目主页提交 \href{https://github.com/seisman/SAC_Docs_zh/issues}{Issue}
\item 在项目主页提交 \href{https://github.com/seisman/SAC_Docs_zh/pulls}{Pull Request}
\end{itemize}
方式一仅用于修正错别字等简单的维护,方式二和方式三则适用于全部的维护。其中,方式一
对维护者没有其他要求;方式二要求维护者注册GitHub账户并懂得如何提交Issue;
方式三要求维护者注册GitHub账户,会使用版本控制工具Git,会在GitHub上
提交Issue和Pull Request,也要求掌握 \LaTeX 及编译的基础知识。
关于方式三的具体如何操作,在后面的章节会专门介绍。

\subsection{维护者的收益}
维护者全凭志愿,没有任何物质报酬。除此之外,维护者将会:
\begin{enumerate}
\item 被加入到维护者列表中\footnote{只有具有足够维护量的维护者才会被加入到列表中,最终解释权归SeisMan所有。};
\item 结识更多的SAC用户以及地震学同行;
\item 对SAC有更深的理解;
\item 学会Git,参与开源项目;
\item 了解MarkDown和\LaTeX ;
\end{enumerate}

\subsection{对维护者的要求}
所有SAC用户,都可以通过方式一参与到该项目的维护中,没有其他额外的需求。

如果想要对项目有更多的贡献,则需要满足如下要求:
\begin{enumerate}
\item 使用Linux或Mac操作系统;
\item 拥有GitHub账户;
\item 了解Git的基础知识;
\item 了解 \LaTeX 的基础知识;
\end{enumerate}
