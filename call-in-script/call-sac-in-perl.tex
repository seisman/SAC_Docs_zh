\section{在Perl中调用SAC}
\label{sec:sac-perl}

\subsection{简介}
下面的脚本中给出了一个简单的例子,展示了如何在Perl中调用SAC。相对于Bash来说,Perl脚本似乎
需要更多的键入,但是相对于Perl的优势来说,这些都不算什么。
\inputminted{perl}{./call-in-script/simple-script.pl}

\subsection{头段变量}
Perl无法直接引用SAC文件的头段变量值,依然需要利用SAC宏的的语法。Perl的优势在于可以
一边向SAC传递信息,一边运行自身的命令。
\inputminted{perl}{./call-in-script/variables.pl}
本例中第9行,在SAC运行的同时Perl临时定义了一个变量,并成功将其传递给了SAC,这在
Bash中是不容易做到的。

\subsection{内联函数}
Perl可以完成各种复杂的数学运算:
\inputminted{perl}{./call-in-script/arithmetic-functions.pl}

Perl对字符串的处理更是Perl的杀手锏:
\inputminted{perl}{./call-in-script/string-functions.pl}

\subsection{条件判断和循环控制}
Perl也可以很容易地实现条件判断和循环控制:
\inputminted{perl}{./call-in-script/do-loops.pl}
这个例子中只启动了一次SAC,然后开始循环读入当前目录下的每一个SAC文件,最后统一
做数据处理,并写到新文件中。

该Perl脚本与上一个Bash脚本相比,实现了几乎相同的功能,但Perl脚本中仅启动和退出SAC
一次,与Bash脚本的多次启动相比,其效率要高很多。

\subsection{修改发震时刻}
\label{subsec:ch-origin}
除此之外,在设置发震时刻的时候,还会遇到另一个问题,如何计算发震日期是一年的第几天。
没有简单的办法,只能通过编程实现,下面给出一个半完整的示例:
\inputminted{perl}{./call-in-script/ch-origin.pl}
本例中,子程序``ymd2doy''用于计算某日期是一年中的第几天。

\subsection{文件重命名}
\label{subsec:rename-in-perl}
Perl下先用 \texttt{split} 函数将文件名用点号分割,由于 \texttt{split} 函数的第一个参数
实际上是正则表达式,故而这里的点号需要转义。点号分割之后的字符串数组赋值给一系列
变量,然后使用 \texttt{rename} 函数实现重命名。
\inputminted{perl}{./call-in-script/rename1.pl}

上面的方法中定义了一堆变量但是却没有用到。可以将 \texttt{split} 函数的结果赋值给数组,
然后直接取数组中需要的元素即可。需要注意的是,Perl的数组下标从零开始。
\inputminted{perl}{./call-in-script/rename2.pl}
