\section{数据重命名}
从压缩数据格式中解压得到的SAC数据,其命名方式不够友好。比如用rdseed解压SEED数据
得到的SAC数据,文件名的格式如下:
\begin{minted}{console}
yyyy.ddd.hh.mm.ss.ffff.NN.SSSSS.LL.CCC.Q.SAC
\end{minted}
其中,
\begin{itemize}
\item yyy.ddd.hh.mm.ss.fff是SAC文件中第一个数据点对应的时刻;
\item NN为台网名,用两个字符表示;
\item SSSSS为台站名,一般为3-4字符,最多5字符;
\item LL为location id,一般为空或两字符,常见的是00、01、10,用于表征一个台站处的不同仪器;
\item CCC为通道名,3字符,如BHE、BHN、BHZ;
\item Q为质量控制标识,可以取D、M、R、Q。
\end{itemize}

示例如下:
\begin{minted}{bash}
2012.055.12.34.56.7777.YW.MAIO.01.BHE.Q.SAC
2012.055.12.34.50.6666.YW.MAIO.01.BHN.Q.SAC
2012.055.12.34.54.5555.YW.MAIO.01.BHZ.Q.SAC
\end{minted}
三个文件代表了YW台网MAIO台站的宽频地震仪记录的三分量波形数据。这样的长文件名在数据处理
时显得很麻烦,一般都会根据实际需求进行适当的简化。

在某些情况下,我们会将同一事件在所有台站的波形数据放在同一个文件夹下,并将文件名以事件的
发生日期/时间来命名。那么,SAC文件名中的时间信息就可以被简化:
\begin{minted}{bash}
YW.MAIO.BHE
YW.MAIO.BHN
YW.MAIO.BHZ
\end{minted}

有时候,我们会将不同事件在同一个台站的波形数据放在同一个文件夹下,并将文件名以台站
名来命名,此时数据文件名中可能需要保留事件的日期信息:
\begin{minted}{bash}
MAIO.20120224.BHE
MAIO.20120224.BHN
MAIO.20120224.BHZ
\end{minted}

鉴于SAC命令的语法,在数据命名时最好将分量名放在最后,而将台站名放在最前面。
这样,在使用SAC的通配符读取特定事件的所有台站的垂直分量波形数据时,可以:
\begin{SACCode}
SAC> r *.20120224.BHZ
\end{SACCode}
或者读取所有事件在同一台站的波形记录:
\begin{SACCode}
SAC> r MAIO.*.BHZ
\end{SACCode}