\SACCMD{int}
\label{cmd:int}

\SACTitle{概要}
利用梯形法或矩形法对数据进行积分

\SACTitle{语法}
\begin{SACSTX}
INT [T!RAPEZEIDAL!|R!ECTANGULAR!]
\end{SACSTX}

\SACTitle{缺省值}
\begin{SACDFT}
int trapezoidal
\end{SACDFT}

\SACTitle{说明}
该命令使用梯形法或矩形法对数据进行数值积分,可以处理等间隔数据或非等间隔数据。除积分之外,
还会根据具体情况修改因变量类型 !idep!,并重新计算 !depmax!、
!depmin!、!depmen!。

对于函数$f(x)$其积分用梯形法表示为
\[
    y_n = y_{n-1} + \frac{1}{2}(x_{n+1}-x_n) (f(x_n)+f(x_{n+1})), \quad n\in[1,npts-1]
\]
用矩形法表示为:
\[
    y_n = y_{n-1} + (x_n-x_{n-1})f(x_n), \quad n\in[1,npts]
\]
二者均有边界条件$y_0=0$。

对于等间隔数据,若使用梯形积分,数据点数 !npts! 将减1,文件的头段变量
!b! 将增加半个采样周期\footnote{若使用矩形积分,理论上 !npts! 和
!b! 也应有所修改,但实际代码中却未对此多做处理,暂不确定是否是Bug。}。

\SACTitle{头段变量}
depmin、depmax、depmen、idep、npts、b、e
