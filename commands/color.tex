\SACCMD{color}
\label{cmd:color}

\SACTitle{概要}
控制彩色图形设备的颜色选项

\SACTitle{语法}
\begin{SACSTX}
COL!OR! [ON|OFF|color] [I!NCREMENT! [ON|OFF]] [S!KELETON! color]
    [B!ACKGROUND! color] [L!IST! S!TANDARD!|colorlist]
\end{SACSTX}
color是下面中的一个:
\begin{SACSTX}
W!HITE!|R!ED!|G!REEN!|Y!ELLOW!|BLU!E!|M!AGENTA!|C!YAN!|BLA!CK!
\end{SACSTX}

这里有些参数在缩写的情况下可能会有歧义,请谨慎使用,而且LIST选项必须放在命令的最后

\SACTitle{输入}
\begin{description}
\item [ON] 打开颜色选项单数不改变其他选项
\item [OFF] 关闭颜色选项
\item [color] 打开颜色选项并将数据设置为颜色color
\item [INCREMENT ON] 每个数据文件绘出后,根据colorlist的顺序改变颜色
\item [INCREMENT OFF] 不改变数据颜色
\item [SKELETON colo]: 按照标准颜色名或颜色号修改边框颜色
\item [BACKGROUND color] 修改背景色为color
        \footnote{白色背景与黑色线条对比强烈,可以考虑设置背景色为cyan}
\item [LIST colorlist] 改变颜色列表,将数据颜色设置为列表中第一个颜色,并打开颜色开关
\item [LIST STANDARD] 将颜色列表设为标准列表,将数据颜色设置为列表中第一个颜色,并打开颜色开关
\end{description}

\SACTitle{缺省值}
\begin{SACDFT}
color black increment off skeleton black background white
    list standard
\end{SACDFT}

\SACTitle{说明}
该命令控制设备的颜色属性,数据颜色是用于绘制这个数据文件的颜色。当一个数据文件绘制
完毕后,数据颜色可以根据颜色列表自动改变。skeleton颜色是用于绘制注释轴、标题、网格 、
框架的颜色。背景色是空框架在未绘制任何图形之前的颜色。

多数情况下你会选择标准颜色名,比如red,这是与图形设备无关的。然而有时候你可能想选择一个
非标准颜色,比如aquamarine,这个可以将颜色表装入图形设备来实现。

这个表将特定的颜色、亮度、对比度等与一个数字联系起来,然后你就可以通过设定对应的整数值
选择aquamarine作为你的绘图的一个部分的颜色,这个需要点工作量,可是如果你喜欢,这就值得。

如果你正在同一张图上绘制多个数据文件,通过INCRMENT选项可以使得不同数据有不同的颜色。
标准颜色表顺序如下:
\begin{minted}{console}
RED, GREEN, BLUE, YELLOW, CYAN, MAGENTA, BLACK
\end{minted}

\SACTitle{示例}
为了使数据颜色从红色开始不断变换:
\begin{SACCode}
SAC> color red increment
\end{SACCode}

为了设置数据颜色为红色,背景白色,蓝色边框:
\begin{SACCode}
SAC> color red background white skeleton blue
\end{SACCode}

为了设置一个数据颜色不断变换,颜色列表为red、white、blue,背景色为aquamarine(!!!):
\begin{SACCode}
SAC> color red increment backgroud 47 list red white blue
\end{SACCode}
上面的例子假设aquamarine是颜色表的47号。
