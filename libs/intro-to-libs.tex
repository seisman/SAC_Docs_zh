\section{SAC库简介}
SAC提供了两个函数库:libsacio.a和libsac.a,用户可以在自己的C或Fortran程序
中直接使用函数库中的子函数。这些库文件位于 \verb|${SACHOME}/lib| 中。

\subsection{libsacio库}
这个库文件的子函数可用于读写SAC数据文件、头段变量、黑板变量。这些子函数可以在用户
的C或Fortran程序中直接使用。

libsacio.a中可用的子函数包括:
\begin{table}[H]
\centering
\caption{libsacio子函数}
\ttfamily
\begin{tabular}{ll}
\toprule
子函数      &       说明            \\
\midrule
rsac1       &       读取等间隔文件  \\
rsac2       &       读取不等间隔文件和谱文件    \\
wsac1       &       写入等间隔文件  \\
wsac2       &       写入不等间隔文件    \\
wsac0       &       可以写等间隔文件或不等间隔文件  \\
getfhv      &       获取浮点型头段变量值    \\
setfhv      &       设置浮点型头段变量值    \\
getihv      &       获取枚举型头段变量值    \\
setihv      &       设置枚举型头段变量值    \\
getkhv      &       获取字符串头段变量值    \\
setkhv      &       设置字符串头段变量值    \\
getlhv      &       获取逻辑型头段变量值    \\
setlhv      &       设置逻辑型头段变量值    \\
getnhv      &       获取整型头段变量值      \\
setnhv      &       设置整型头段变量值  \\
readbbf     &       读取一个黑板变量文件    \\
writebbf    &       写一个黑板变量文件      \\
getbbv      &       获取一个黑板变量的值    \\
setbbv      &       给一个黑板变量赋值      \\
distaz      &       计算地球上任意两点间的震中距、方位角和反方位角  \\
\bottomrule
\end{tabular}
\end{table}

对于C源码,用如下命令编译
\begin{minted}{console}
$ gcc -c source.c -I/usr/local/sac/include
$ gcc -o prog source.o -lm -L/usr/local/sac/lib -lsacio
\end{minted}
也可以利用SAC提供的sac-config命令简化此编译命令:
\begin{minted}{console}
$ gcc -c source.c `sac-config -c`
$ gcc -o prog source.o -lm `sac-config -l sacio`
\end{minted}

对于Fortran77源码,用如下命令编译
\begin{minted}{console}
$ gfortran -c source.f
$ gfortran -o prog source.o -L/usr/local/sac/lib/ -lsacio
\end{minted}
也可以利用SAC提供的sac-config命令简化此编译命令:
\begin{minted}{console}
$ gfortran -c souce.f
$ gfortran -o prog source.o `sac-config -l sacio`
\end{minted}

\subsection{libsac.a库}
这个库是从101.2版本才引入的,其是libsacio.a的超集,包含了几个数据处理常用的子函数。

libsac.a包含如下子函数:
\begin{itemize}
\item xapiir  无限脉冲响应滤波器;
\item firtrn  有限脉冲滤波器,Hilbert变换;
\item crscor  互相关;
\item next2   返回比输入值大的最小的2的幂次;
\item envelope 计算包络函数;
\end{itemize}

对于C源码,用如下命令编译
\begin{minted}{console}
$ gcc -c source.c -I/usr/local/sac/include
$ gcc -o prog source.o -lm -L/usr/local/sac/lib -lsac
\end{minted}
也可以利用SAC提供的sac-config命令简化此编译命令:
\begin{minted}{console}
$ gcc -c source.c `sac-config -c`
$ gcc -o prog source.o -lm `sac-config -l sac`
\end{minted}

对于Fortran77源码,用如下命令编译
\begin{minted}{console}
$ gfortran -c source.f
$ gfortran -o prog source.o -L/usr/local/sac/lib/ -lsac
\end{minted}
也可以利用SAC提供的sac-config命令简化此编译命令:
\begin{minted}{console}
$ gfortran -c souce.f
$ gfortran -o prog source.o `sac-config -l sac`
\end{minted}
