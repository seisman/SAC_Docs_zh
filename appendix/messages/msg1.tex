\pdfbookmark[1]{E1xxx}{E1xxx}
\SACTitle{1028 External command does not exist}
执行 \nameref{cmd:load} 命令时找不到外部命令。产生该错误的原因很多,可能
是环境变量 \verb|LD_LIBRARY_PATH| 不包含要载入的共享库的所在目录,或
\verb|$SACSOLIST| 中不包含要载入的共享库的名字。

\SACTitle{1103 No help package is available}
没有可用的帮助文档包,可能原因是没找到 \verb|${SACAUX}/help| 目录。

\SACTitle{1106 Not a valid SAC command.}
对于每一行命令,SAC首先会检查是否是SAC内部的命令,如果不是,则检查是否是系统自带
的命令,比如 \texttt{ls}、\texttt{cp} 等。

一个例外是系统命令 \texttt{rm}。在SAC中直接执行rm命令会出现如上所示的错误。出现该
错误的原因是SAC禁用了系统命令 \texttt{rm},主要目的是为了防止将 \texttt{r *.SAC} 误敲成
\texttt{rm *.SAC} 而导致数据的误删除。可以使用 \nameref{cmd:systemcommand} 命令显式
调用系统命令,如下:
\begin{SACCode}
SAC> rm BAD*.SAC
 ERROR 1106: Not a valid SAC command.
SAC> sc rm BAD*.SAC
\end{SACCode}

\SACTitle{1301 No data files read in}
内存中未读入数据。可能是未指定要读取的文件列表,或列表中的文件不可读。

\SACTitle{1303 Overwrite flag is not on for file}
该错误主要出现在写SAC文件时,出现该错误的原因是SAC文件的头段变量 \texttt{lovrok} 的值
为 \texttt{FALSE},即磁盘中的数据不允许被覆盖。解决该问题的方法有两种:
\begin{itemize}
\item 以其他文件名写入磁盘,不覆盖磁盘文件;
\item 修改 \texttt{lovrok} 的值为 \texttt{TRUE};
\end{itemize}

\SACTitle{1304 Illegal operation on data file}
对数据文件的非法操作。

\SACTitle{1305 Illegal operation on time series file}
某些命令无法对时间序列数据进行操作。

\SACTitle{1306 Illegal operation on unevenly spaced file}
某些命令无法对不等间隔数据进行操作。

\SACTitle{1307 Illegal operation on spectral file}
命令不能对内存中的谱文件进行操作。

\SACTitle{1311 No list of filenames to write}
没有要写入的文件列表。

\SACTitle{1312 Bad number of files in write file list}
通常在使用 \nameref{cmd:write} 命令时会出现该问题。出现该错误的原因是内存中的
波形文件的数目与write命令给出的文件名的数目不想匹配。在该错误信息的后面,紧接着
会给出write命令中给出的文件数目以及内存中的波形数目。

\SACTitle{1317 The following file is not a SAC data file:}
试图读入非SAC格式的文件所产生的错误。

\SACTitle{1320 Available memory too small to read file}
系统内存不足。

\SACTitle{1322 Undefined starting cut for file}
\nameref{cmd:cut} 命令中时间窗的起始参考头段未定义。默认情况下,会使用磁盘
文件的起始时间代替,也可以使用 \nameref{cmd:cuterr} 命令控制该错误的处理方式。

\SACTitle{1323 Undefined stop cut for file}
\nameref{cmd:cut} 命令中时间窗的结束参考头段未定义。默认情况下,会使用磁盘
文件的结束时间代替,也可以使用 \nameref{cmd:cuterr} 命令控制该错误的处理方式。

\SACTitle{1324 Start cut less than file begin for file}
\nameref{cmd:cut} 命令中时间窗的开始时间早于磁盘文件的开始时间。默认情况下,
会使用磁盘文件的开始时间代替,也可以使用 \nameref{cmd:cuterr} 命令控制该错误的
处理方式。

\SACTitle{1325 Stop cut greater than file end for file}
\nameref{cmd:cut} 命令中时间窗的结束时间晚于磁盘文件的结束时间。默认情况下,
会使用磁盘文件的结束时间代替,也可以使用 \nameref{cmd:cuterr} 命令控制该错误的
处理方式。

\SACTitle{1326 Start cut greater than file end for file}
\nameref{cmd:cut} 命令中时间窗的开始时间晚于文件结束时间。

\SACTitle{1340 data points outside allowed range contained in file}
文件中数据点的值超过了所允许的范围。比如 \nameref{cmd:log} 中要求数据为正。

\SACTitle{1379 No SORT parameters given}
使用了 \nameref{cmd:sort} 命令,但未指定按照哪个参数排序。

\SACTitle{1380 Too many SORT parameters:}
\nameref{cmd:sort} 命令中用于排序的参数太多。

\SACTitle{1381 Not a valid SORT parameter}
无效的 \nameref{cmd:sort} 参数。

\SACTitle{1383 SORT failed}
排序失败。

\SACTitle{1606 Maximum allowable DFT is 16777216}
SAC中与FFT相关的命令,所能允许的最大数据点数是$2^{24}=16777216$。

\SACTitle{1611 Corner frequency greater than Nyquist for file}
对数据进行滤波时,拐角频率超过了文件的Nyquist采样率。

\SACTitle{1701 Can't divide by zero}
除零的非法操作。

\SACTitle{1702 Non-positive values found in file}
数据文件中存在非正值。

\SACTitle{1801 Header field mismatch:}
该错误出现在 \nameref{cmd:addf}、\nameref{cmd:subf}、\nameref{cmd:divf}、\nameref{cmd:mulf}
以及 \nameref{cmd:merge} 和 \nameref{cmd:beam} 中。

出现该错误的原因是多个数据文件中的头段变量不匹配。该命令会明确给出不匹配的头段变量名,以及
出现不匹配的数据文件,以供用户查错。会出现不匹配的头段变量包括npts、delta、kstnm、knetwk、
kcmpnm。

\SACTitle{1802 Time overlap}
要进行操作的两个数据的时间段不完全重合。

\SACTitle{1803 No binary data files read in.}
\nameref{cmd:addf}、\nameref{cmd:subf}、\nameref{cmd:merge} 等命令需要先读入二进制数据,再对数据做操作。

\SACTitle{1805 Time gap}
使用 \nameref{cmd:merge} 命令时,两段数据间存在时间间断。
