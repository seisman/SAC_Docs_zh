\SACCMD{vspace}
\label{cmd:vspace}

\SACTitle{概要}
设置图形的最大尺寸和长宽比

\SACTitle{语法}
\begin{SACSTX}
VSP!ACE! [FULL|v]
\end{SACSTX}

\SACTitle{输入}
\begin{description}
\item [FULL] 使用整个viewspace,这是最大窗口尺寸
\item [v] 设置viewspace的纵横比为v,具有这个纵横比的最大区域即为viewspace
\end{description}

\SACTitle{缺省值}
\begin{SACDFT}
vspace full
\end{SACDFT}

\SACTitle{说明}
viewspace是屏幕上可以用于绘图的部分。viewspace的形状和尺寸在不同图形
设备之间有很大的变化。
\begin{enumerate}
\item 尽管在尺寸上有很大不同,许多图形终端都具有0.75的纵横比
\item SGF文件的纵横比为0.75,其大约是标准的8.5*11英寸纸张的纵横比
\item 由XWindows或SunWindows图形设备建立的窗口可以有你想要的任意纵横比
\end{enumerate}

默认情况下绘图会占据整个viewspace。该命令可以控制viewspace的纵横比,
从而使你能够控制图形的形状。如果确定了一个纵横比,则viewspace就是设备上
具有这个纵横比的最大区域。

当你使用 \nameref{cmd:plotc} 命令在交互设备上建立一张图,并且最终要
将它发送到SGF设备上,这个命令特别有用。在绘制任何图形之前,必须设置
纵横比为0.75。这将保证图形在SGF文件上与在交互设备上相同。如果你要建立
一个独立于图形设备的正方形viewspace,则可以简单地设置纵横比为1.0。
