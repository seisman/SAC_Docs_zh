\section{编译文档}
在维护文档的同时,你肯定想要将源码编译成PDF以检查自己的修改效果。这一节
简单介绍如何编译文档。

该项目源码是用 \LaTeX 写成的,Linux下可以用 \TeX~Live~2015 编译。

\subsection{安装\TeX~Live~2015}
\TeX~Live 是由国际 \TeX 用户组(\TeX~Users Group,TUG)整理和发布的 \TeX
软件发行套装,包含与 \TeX 系统相关的各种程序、编辑与查看工具、常用宏包
及文档、常用字体及多国语言支持。

参考 \url{http://seisman.info/install-texlive-under-linux.html} 安装
TeX~Live~2015 并更新所有宏包至最新版本。

\subsection{安装pygment}
源码中的代码高亮使用 \texttt{minted} 和 \texttt{listings} 宏包实现。
前者依赖于Python的第三方模块 \texttt{pygments}。

CentOS/RHEL下可以用如下命令安装:
\begin{minted}{console}
$ sudo yum install python-pygments
\end{minted}

Ubuntu/Debian下可以用如下命令安装:
\begin{minted}{console}
$ sudo apt-get install python-pygments
\end{minted}

\subsection{编译文档}
进入项目所在目录,输入命令进行编译:
\begin{minted}{console}
$ make
\end{minted}
若编译时出现错误,请到项目主页处提交Issue。
