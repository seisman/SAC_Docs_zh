\SACCMD{bandpass}
\label{cmd:bandpass}

\SACTitle{概要}
对数据文件使用无限脉冲带通滤波器

\SACTitle{语法}
\begin{SACSTX}
B!AND!P!ASS! [BU!TTER!|BE!SSEL!|C1|C2] [C!ORNERS! v1 v2] [N!POLES! n] [P!ASSES! n]
    [T!RANBW! v] [A!TTEN! v]
\end{SACSTX}

\SACTitle{输入}
\begin{description}
\item [BUTTER] 应用一个Butterworth滤波器
\item [BESSEL] 应用一个Bessel滤波器
\item [C1] 应用一个Chebyshev I型滤波器
\item [C2] 应用一个Chebyshev II滤波器
\item [CORNERS v1 v2] 设定拐角频率分别为v1和v2,即频率通带为v1--v2
\item [NPOLES n] 设置极数为N,范围:1--10
\item [PASSES n] 设通道数为N,范围:1--2
\item [TRANBW v] 设置Chebyshev转换带宽为v
\item [ATTEN v] 设置Chebyshev衰减因子为v
\end{description}

\SACTitle{缺省值}
\begin{SACDFT}
bandpass butter corner 0.1 0.4 npoles 2 passes 1 tranbw 0.3 atten 30
\end{SACDFT}

\SACTitle{说明}
在SAC中有一系列无限脉冲滤波器(IIR)可以使用。这些递归的数字滤波器是基于
传统的模拟滤波器设计的:Butterworth、Bessel、Chebyshev I型以及
Chebyshev II型。这些模拟滤波器经过双线性变换(一种可以保持模拟滤波器稳定
性的变换方式)转换成数字滤波器。

一般来说,多数情况下Butterworth滤波器是个不错的选择。因为它有一个相当
尖锐的转换带以及平缓的群延迟响应。Butterworth滤波器是默认的滤波器类型,
它的 \SI{3}{dB} 点指定在截止频率处。Bessel滤波器对于那些需要线性相位而
没有双通滤波的应用来说是最好的。它的振幅响应并不够好。SAC的Bessel滤波器
经过归一化因此它的 \SI{3}{\dB} 点也在指定的截止频率处。两个Chebyshev滤波
器可以用于适应通带与阻带之间具有较为尖锐的转变的要求。尽管他们有较好的
振幅响应,但是它们的群延迟响应是SAC所有滤波器中最差的。

在使用这些滤波器时需要小心。首先,所有的递归滤波器都有非线性相位响应,
这将导致滤波后波形的频散。对于滤波后波形的相位很重要的应用来说,SAC
提供了一个递归滤波的零相位工具。零相位滤波器可以通过正向和反向(而不仅仅
只是正向滤波)两次滤波来实现。这个双向操作的滤波器产生一个等效振幅响应,
它等于原来振幅响应的平方。它同时也产生一个非因果的滤波脉冲响应,使得信号
在尖锐起始时间之前附加一个虚假的前驱信号。因此双通滤波后数据无法正确的
给出起跳到时。对于信号前驱不可忽略的情况,例如读取震相起跳到时,使用
双通滤波器不是一个很好的选择。其次,当滤波器的通带宽度相比折叠频率
很小时,滤波器可能会出现数值不稳定。这个问题在增加极数时会更加严重。
在要求使用一个非常窄的通带时,一个有效的办法是首先对数据进行采样,对
采样之后的数据用一个通带宽一些的滤波器进行滤波,最后对数据进行插值回到
原始采样率。当所需通带宽降到折叠频率的百分之几时很有必要使用这种策略。

一般来说,随着极数的增加滤波器将有一个从通带到阻带的尖锐的转变。然而,
极数太大是要付出代价的。滤波器群延迟一般随着极数的增加而变得更宽,结果
导致波形混淆。那需要3或4个极的应用应该重新考虑。

Butterworth和Bessel滤波器的设计特别简单。你只需要指定截止频率和极数即可。

Chebyshev滤波器设计起来更复杂一点,你还需要提供转换带宽以及阻带衰减因子。
转换带宽是滤波器通带和阻带之间的区域的宽度,它被指定为模拟滤波器通带宽度
的一部分由于双线性变换频率轴的非线性弯曲,递归数字滤波器的转换带宽可能会
比设计时指定的要小。在SAC里,模拟滤波器的截止频率在双线性变换之后要做补偿
以保证其满足设计要求。阻带边界同样也是不真实的,因此,如果明确的设置阻带
边界是重要的,当你的截止频率选定后你必须对其进行补偿。

阻带衰减是通带增益与阻带增益的比值。

\SACTitle{示例}
应用一个四极Butterworth滤波器,拐角频率为 \SI{2}{\Hz} 和 \SI{5}{\Hz}:
\begin{SACCode}
SAC> bp n 4 c 2 5
\end{SACCode}

在此之后如果要应用一个二极双通具有相同频率的Bessel:
\begin{SACCode}
SAC> bp n 2 be p 2
\end{SACCode}

\SACTitle{头段变量改变}
depmin、depmax、depmen
