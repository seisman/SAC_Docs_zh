\section{IRIS地震数据申请工具}
从IRIS申请地震波形数据时,有很多工具可供选择与使用。这一节不会去介绍工具
的具体用法,而是试着总结不同工具各自的优缺点以及适用范围。读者应根据实际
需求选择合适的工具,并自行阅读相应工具的说明文档。

\subsection{Wilber III}
\noindent 主页:\url{http://ds.iris.edu/wilber3/find_event}

\noindent 优缺点及适用范围:
\begin{itemize}
\item 网页端
\item 直观地显示地震分布、台站分布
\item 适用于下载基于事件的波形数据
\item 只能下载事件波形数据,无法下载连续波形数据
\item 一次只能下载单个事件波形数据,难以自动化
\end{itemize}

\subsection{BREQ\_FAST}
\noindent 主页:\url{https://ds.iris.edu/ds/nodes/dmc/manuals/breq_fast/}

\noindent 优缺点及适用范围:
\begin{itemize}
\item 通过发送邮件申请数据
\item 邮件的内容格式固定
\item 易于自动化及批量申请
\item 申请得到的数据放在IRIS的FTP里
\end{itemize}

\subsection{JWEED}
\noindent 主页:\url{https://ds.iris.edu/ds/nodes/dmc/software/downloads/jweed/}

\noindent 优缺点及适用范围:
\begin{itemize}
\item Java写的带界面客户端
\item 跨平台,但兼容性不够
\item 运行稍慢
\item 适用于下载基于事件的波形数据
\item 只能下载事件波形数据,无法下载连续波形数据
\end{itemize}

\subsection{Web Service}
\noindent 主页:\url{http://service.iris.edu/}

\noindent 优缺点及适用范围:
\begin{itemize}
\item 网页服务
\item 按照固定的格式构建一个URL即可申请相应的数据
\end{itemize}

\subsection{Web Service fetch scripts}
\noindent 主页:\url{https://seiscode.iris.washington.edu/projects/ws-fetch-scripts}

\noindent 优缺点及适用范围:
\begin{itemize}
\item Perl脚本
\item 基于Web Service
\end{itemize}

\subsection{irisfetch.m}
\noindent 主页:\url{https://ds.iris.edu/ds/nodes/dmc/software/downloads/irisfetch.m/}

\noindent 优缺点及适用范围:
\begin{itemize}
\item Matlab脚本,可以在Matlab中直接调用相关函数获取数据
\item 基于Web Service
\end{itemize}

\subsection{SOD}
\noindent 主页:\url{http://www.seis.sc.edu/sod/}

\noindent 优缺点及适用范围:
\begin{itemize}
\item 命令行工具
\item 易于批量处理
\end{itemize}
