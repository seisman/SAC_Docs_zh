\SACCMD{yfudge}
\label{cmd:yfudge}

\SACTitle{概要}
改变Y轴的``插入因子''

\SACTitle{语法}
\begin{SACSTX}
YFUDGE [ON|OFF|v]
\end{SACSTX}

\SACTitle{输入}
\begin{description}
\item [ON] 打开插入选项,但不改变插入因子 
\item [OFF] 关闭插入选项 
\item [v] 打开插入因子,改变插入因子为v 
\end{description}

\SACTitle{缺省值}
\begin{SACDFT}
yfudge 0.03
\end{SACDFT}

\SACTitle{说明}
当打开此选项时,实际轴范围将根据插入因子改变,确定线性坐标轴范围的方法是:
\[ YDIFF=YFUDGE*(YMAX-YMIN) \]
\[ YMIN=YMIN-YDIFF \]
\[ YMAX=YMAX+YDIFF \]
其中YMIN和YMAX是数据的极值,YFUDGE是插入因子,这个算法对于对数坐标也是相似的。
该选项仅当坐标轴的范围设置为数据极值时方可使用。

\SACTitle{相关命令}
\nameref{cmd:ylim}
