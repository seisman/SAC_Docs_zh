\SACCMD{rtrend}
\label{cmd:rtrend}

\SACTitle{概要}
去除线性趋势

\SACTitle{语法}
\begin{SACSTX}
RTR!END! [Q!UIET!|!V!ERBOSE]
\end{SACSTX}

\SACTitle{输入}
\begin{description}
\item [QUIET] 不显示线性拟合信息
\item [VERBOSE] 终端显示线性拟合信息
\end{description}

\SACTitle{缺省值}
\begin{SACDFT}
rtrend quiet
\end{SACDFT}

\SACTitle{说明}
该命令利用最小二乘方法将数据拟合成一条直线,然后从数据中减去该直线所
表征的线性趋势。数据可以是不等间隔的。

若有$n$个数据$(x_i,y_i)$,利用最小二乘法拟合直线$y=ax+b$。其中斜率为
\[
    a = \frac{n\sum x_i y_i - \sum x_i \sum y_i}
    {n\sum x_i^2 - (\sum x_i)^2}
\]

Y轴截距为
\[
    b = \frac{\sum x_i^2 \sum y_i - \sum x_i \sum x_i y_i}
    {n\sum x_i^2 - (\sum x_i)^2}
\]

内存中最后一个文件的线性拟合参数将会写入到如下黑板变量中:
\begin{itemize}
\item \verb|RTR_SLP| 直线斜率
\item \verb|RTR_YINT| 直线的Y截距
\item \verb|RTR_SDSLP| 斜率的标准差
\item \verb|RTR_SDYINT| Y截距的标准差
\item \verb|RTR_SDDTA| 数据的标准差
\item \verb|RTR_CORRCF| 数据和拟合结果的相关系数
\end{itemize}

\SACTitle{头段变量改变}
depmin、depmax、depmin
