\SACCMD{sgf}
\label{cmd:sgf}

\SACTitle{概要}
控制SGF设备选项

\SACTitle{语法}
\begin{SACSTX}
SGF [P!REFIX! text] [N!UMBER! n] [D!IRECTORY! CURRENT|pathname]
    [S!IZE! N!ORMAL!|F!IXED! v|S!CALED! v] [O!VERWRITE! ON|OFF]
\end{SACSTX}

\SACTitle{输入}
\begin{description}
\item [PREFIX text] 设置SGF的文件名前缀为 !text!(最多24字符长)
\item [NUMBER n] 设置下一个SGF的编号为n。若n为0,则SAC搜索SGF文件
    目录下的SGF文件最大编号,并将其值加1
\item [DIRECTORY CURRENT] 将SGF文件放在当前目录
\item [DIRECTORY pathname] 将SGF放在指定目录下
\item [SIZE NORMAL] 产生一个常规大小绘图,常规图形有一个10*7.5英寸的
    viewspace(最大绘图区域)。viewport(viewspace中除轴和标签之外的、
    绘图区)的默认值为8*5英寸
\item [SIZE FIXED v] 产生一个在X方向viewport为v英寸长的图形
\item [SIZE SCALED v] 产生一个视口在X方向上为v乘以X坐标极限值的图形
\item [OVERWRITE ON|OFF] 当打开时,文件号不递增,每个新文件将擦除先前的
    文件
\end{description}

\SACTitle{缺省值}
\begin{SACDFT}
sgf prefix f number 1 directory current size normal
\end{SACDFT}

\SACTitle{说明}
该命令控制SGF文件的命名规则和后续SGF文件的图形尺寸。SGF文件名由4部分
组成,分别为:
\begin{itemize}
\item !pathname! 目录路径名,默认为当前目录
\item !prefix! 前缀,默认值为 !f!
\item !number! 三位数的frame编号,默认值为 !001!
\item !.sgf! 用于表示SAC图形文件的后缀
\end{itemize}
因而SGF的第一个文件名为 !f001.sgf!。每次新建一个frame时,frame
编号加1。可以强制frame编号从某个给定的值开始。如果你为准备一个报告需要
工作几天时间,而同时又希望所有图形都按一个统一的顺序排列,那么令frame
编号不从1开始便很有用了。

有多个选项可以控制绘图的尺寸,常规的绘图为10*7.5英寸的veiwport范围,
使用默认的viewport的结果是产生一个近似为8*5英寸的图形区。你可强制viewport
的X方向为固定长度或将viewport的X方向与整个坐标范围成比例。尺寸信息
会被写入SGF文件中,然后由其他转换程序负责将SGF转换成合适大小的图片。

\SACTitle{示例}
设置SGF文件的保存目录为非当前目录,并重置frame编号:
\begin{SACCode}
SAC> sgf dir /mydir/sgfstore frame 0
\end{SACCode}

设置viewport的X方向长度为3英寸:
\begin{SACCode}
SAC> sgf size fixed 3.0
\end{SACCode}

创建一个大小相当于海报的图形:
\begin{SACCode}
SAC> sgf size fixed 30.0
\end{SACCode}

设置vierport的X方向的尺寸与地震数据的时间长度成比例:
\begin{SACCode}
SAC> sgf size scaled 0.1  // 10s的数据长度为1英寸
\end{SACCode}
本例中,持续60秒的数据图形将有6英寸长,而持续600秒的数据将有60英寸长,
过长的图形需要特殊的后期处理。
