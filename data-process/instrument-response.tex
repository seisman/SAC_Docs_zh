\section{去仪器响应}
\label{sec:instrument-response}
相关命令:\nameref{cmd:transfer}

SAC中可以使用命令~\nameref{cmd:transfer}~实现去仪器响应,本节只列出日常数据处理
过程中最常用的命令。关于仪器响应的详细介绍,请参考附录``\nameref{chap:resp}''
以及教科书中的相关内容。关于~\nameref{cmd:transfer}~命令的具体用法,参考该命令的
语法说明及示例。

本节会介绍RESP文件和PZ文件的几种用法,并对每种方法的优缺点进行比较。
\subsection{RESP文件}
通常RESP文件可以通过如下命令从SEED数据中提取出来:
\begin{minted}{console}
$ rdseed -Rdf infile.seed
\end{minted}
解压得到的仪器响应文件名的格式为~\verb+RESP.NET.STA.LOC.CHN+。

\subsubsection{RESP方法1}
\begin{SACCode}
SAC> r *.SAC
SAC> trans from evalresp to none freq 0.004 0.007 0.2 0.4
\end{SACCode}
由于波形数据和RESP文件都是用rdseed从SEED数据中解压出来的,因而二者是一一对应的。
对于内存中的每一个波形,transfer命令会提取台站信息,并根据特定的格式在当前目录下
寻找对应的RESP文件。

\subsubsection{RESP方法2}
可以使用fname选项为每个波形指定RESP文件:
\begin{SACCode}
SAC> r 2006.253.14.30.24.0000.TA.N11A..LHZ.Q.SAC
SAC> rmean; rtr; taper
SAC> trans from evalresp fname RESP.TA.N11A..LHZ to none \
                                freq 0.004 0.007 0.2 0.4
\end{SACCode}

\subsubsection{RESP方法3}
把所有的PZ文件合并到同一个文件中,比如文件名叫``\verb+RESP.ALL+'',然后使用如下命令:
\begin{SACCode}
SAC> r *.SAC
SAC> trans from evalresp fname RESP.ALL to none \
                            freq 0.004 0.007 0.2 0.4
\end{SACCode}

\subsection{PZ文件}
SAC PZ仪器响应文件可以直接用rdseed从SEED文件中提取出来:
\begin{minted}{console}
$ rdseed -pdf infile.seed
\end{minted}

\subsubsection{PZ方法1}
手动为每个波形指定PZ响应文件:
\begin{SACCode}
SAC> r OR075_LHZ.SAC
SAC> rmea; rtr; taper
SAC> trans from polezero subtype SAC_PZs_XC_OR075_LHZ to none \
                        freq 0.008 0.016 0.2 0.4
SAC> mul 1.0e9      // 用PZ文件transfer to none得到的位移数据的单位为m
                    // 而SAC默认的单位为nm,因而必须乘以1.0e9
SAC> w OR075.z      // 此时位移数据的单位为nm
\end{SACCode}

\subsubsection{PZ方法2}
可以将多个台站的PZ文件合并到一个总PZ文件中,直接使用如下命令:
\begin{SACCode}
SAC> r *.SAC
SAC> trans from pol s SAC_PZs_ALL to none freq 0.008 0.016 0.2 0.4
SAC> mul 1.0e9
SAC> w over
\end{SACCode}

\subsection{对比}
从易用性来看,RESP方法1、RESP方法3和PZ方法2都是比较易于使用的,只需要一个简单的
命令,即可对所有波形数据进行处理。而RESP方法2和PZ方法1,需要用户自己从数据文件的
文件名或头段中提取信息,并指定对应的响应文件,这需要通过写少量的脚本来实现。

从执行效率来看,做了一个简单的测试。共670个波形数据,用不同的方法去仪器响应,
完成所需时间如下:
\begin{description}
\item [RESP方法1] 58秒;
\item [RESP方法2] 43秒;
\item [RESP方法3] 227秒;
\item [PZ方法1] 8秒;
\item [PZ方法2] 90秒;
\end{description}
从中可以总结出执行效率的如下规律:
\begin{enumerate}
\item RESP2和PZ1相比,RESP3与PZ2相比,可知,PZ文件的效率要高于RESP文件。这很容易理解,
    毕竟RESP文件中包含了更为完整的信息,计算量要更大一些;PZ文件中仅包含了零极点信息
    和总增益信息,对于日常的使用来说,已经足够;
\item RESP1和RESP2相比,区别只是在于后者使用指定的文件,而后者需要自己构建文件名,
    但两者之间的执行时间却相差很多,目前无法理解这一时间差;
\item RESP3和PZ2方法,都是把多个响应函数放在同一个响应文件中,对于每个波形
    都需要对响应文件做遍历以找到匹配的响应函数,因而速度慢也是正常的;这里5倍和
    10倍的时间差还是挺大的;
\end{enumerate}

总结下来:
\begin{itemize}
\item 想要写起来简单,用RESP方法1;
\item 想要执行快,可以用PZ方法1;
\end{itemize}
