\section{数据来源}
地震的波形数据来源有很多:
\begin{enumerate}
\item \href{http://www.seisdmc.ac.cn/}{中国地震局国家测震台网数据备份中心}
\item \href{http://e-service.cwb.gov.tw/wdps/}{台湾中央气象局}
\item 
\href{http://www.aeic.alaska.edu/}{Alaska Earthquake Information Center}
\item 
\href{http://www.earthquakescanada.nrcan.gc.ca/stndon/CNSN-RNSC/index-eng.php}
{Canadian National Seismic Network}
\item 
\href{http://www.iris.edu/hq/}
{Incorporated Research Institutions for Seismology}
\item \href{http://www.fnet.bosai.go.jp/}{NIED F-net}
\item \href{http://www.hinet.bosai.go.jp/}{NIED Hi-net}
\item 
\href{http://www.ncedc.org/}{Northern California Earthquake Data Center}
\item \href{http://pnsn.org/}{Pacific Northwest Seismic Network}
\item \href{http://www.scsn.org/}{Southern California Seismic Network}
\item 
\href{http://scedc.caltech.edu/}
{Southern California Seismic Network at Caltech}
\end{enumerate}
\subsection{中国地震局国家测震台网数据备份中心}
这个数据源提供全球M5.5级以上地震事件的全国地震台站波形数据和
国内M4.0级以上地震事件的区域地震台站的波形数据。

要申请这个数据源的数据,你和你的课题组成员必须是中华人民共和国公民,并且遵守相关规定
\footnote{\url{http://www.seisdmc.ac.cn/class/view?id=8}}。
