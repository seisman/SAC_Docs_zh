\section{学习git和GitHub}
git是一款软件,GitHub是一个网站。二者协同使用是为了组织和协调我和你及其他维护者的工作。
\subsection{注册GitHub账户}
注册GitHub账户;\footnote{\url{https://github.com/join}}
向GitHub账户中添加当前机器的SSH秘钥 , 参考GitHub官方文档
\footnote{\url{https://help.github.com/articles/generating­ssh­keys/}}
。添加秘钥的目的是使得GitHub账户信任当前计算机,并赋予其写权限。
\subsection{安装和配置Git}
\begin{itemize}
\item git是版本控制工具,gitk是用于查看的图形工具:
\begin{minted}{console}
sudo yum install git gitk
\end{minted}
\item git全局配置:
\begin{minted}{console}
git config ­­global user.name "Your Name"
git config ­­global user.email "you@example.com"
\end{minted}
\end{itemize}
\subsection{第一次使用}
\begin{itemize}
\item 进入该手册的项目主页 \url{https://github.com/seisman/SAC_Docs_zh}
,点击右上角的fork;该操作会将叫seisman 账户下的~\verb+SAC_Docs_zh+ 项目复制到你的账户下。
下面均假定你的账户名USER。
\item 在终端执行如下操作:
\begin{minted}{console}
# 下载源码至本机
git clone git@github.com:USER/SAC_Docs_zh.git
# 添加seisman账户下的repo作为其中一个远程repo,并命名为seisman
cd SAC_Docs_zh/
git remote add seisman https://github.com/seisman/SAC_Docs_zh.git
# 新建mydev分支,并切换至该分支,分支名任意
git co -­b mydev
# 用编辑器修改文档
#   比如先修改contributor.tex文件,该文件中包括了该手册的维护者的列表
#   参照已有的记录,添加自己的姓名/昵称、邮箱
#   开始时间是你开始维护此手册的时间
#   结束时间是你决定不再维护此手册的时间
# 修改该文档后,按如下操作提交并推送修改
git add contributor.tex           # 将修改的文件添加到缓存区
git commit ­m "add contributor"   # 提交修改,­m后接注释信息
git push origin mydev          # 将mydev分支推送到GitHub服务器
\end{minted}
\item 进入 \url{https://github.com/USER/SAC_Docs_zh},点击Pull Request即可;
\item 提交完Pull Request之后,我会审核修改,并决定是否接受Pull Request;
\item 若PR已被接受,则可以删除mydev分支:
git push origin :mydev
\end{itemize}
\subsection{第n次使用}
第一次使用的时候有些复杂,第n次使用的时候步骤就简单很多了。
命令行操作如下:
\begin{minted}{console}
# 从seisman的repo中拉取源码的最新版本
git pull seisman
# 将seisman的repo中的最新版本与本地版本合并
git merge seisman/master
# 新建新分支
git co -b mydev2
#
# 修改文件 xxx.0 xxx.1 xxx.2
#
# 添加到缓冲区
git add xxx.0 xxx.1 xxx.2
# 提交更改
git commit ­m 'commit messages'
# 推送更改到服务器
git push
\end{minted}
Push之后,进入GitHub网站,提交Pull Request即可。
\subsection{其他说明}
\begin{itemize}
\item 我对于Git也只是了解皮毛,上面的步骤也许有更简单的操作;
\item Git协作的方式有很多种,可以参考《Git使用规范流程》
\footnote{\url{http://www.ruanyifeng.com/blog/2015/08/git­use­process.html}}
、《git简易指南》\footnote{\url{http://www.bootcss.com/p/git­guide/}}
和《廖雪峰的Git教程》
\footnote{\url{http://www.liaoxuefeng.com/wiki/0013739516305929606dd18361248578c67b8067c8c017b000}}
\item 可以多次add多次再commit,多次commit再push,多次push之后再pull request;
\end{itemize}
总之,希望有人能够参与进来,哪怕只是改几个错别字也很好。
\subsection{如何维护}
前面提到了手册中有哪些东西需要维护,根据工作量的大小大致可以分为两类:
\begin{itemize}
\item 小工作量维护,比如修复简单的bug和typo、整理部分语句、微调LaTeX模板,可以直接修
改并提交Pull Requests;
\item 大工作量维护,比如新增章节、调整文档整体结构等,为了避免多人重复劳动,请先到项
目主页 \url{https://github.com/seisman/SAC_Docs_zh}中提交Issues
\begin{itemize}
\item 若暂时不打算解决该Issue,则设置标签为“Pull Request Welcomed”;若正在解决该Issue,则设置标签为“In Progress”;
\item Issue列表中所有标签为“Pull Request Welcomed”的Issue均可随意认领;
\end{itemize}
\end{itemize}
