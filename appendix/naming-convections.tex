用rdseed程序从标准SEED格式中解压得到的SAC文件,通常都具有固定格式的文件名。具体格式为:
\begin{verbatim}
    yyyy.ddd.hh.mm.ss.ffff.NN.SSSSS.LL.CCC.Q.SAC
\end{verbatim}
其中
\begin{itemize}
\item \verb+yyyy.ddd.hh.mm.ss.ffff+是SAC文件中第一个数据点对应的时间
    \begin{itemize}
    \item \verb+yyyy+为年;
    \item \verb+ddd+为一年的第多少天;
    \item \verb+hh.mm.ss+为时、分、秒;
    \item \verb+ffff+为毫秒;需要注意的是$\SI{1}{\s}=\SI{1000}{\ms}$,
        这里毫秒用了4位来表示。
    \end{itemize}
\item \verb+NN+为台网名,2字符;
\item \verb+SSSSS+为台站名,一般是3个字符,偶尔见到4字符的;
\item \verb+LL+为location id,为空或两字符;
\item \verb+CCC+为通道名;
\item \verb+Q+为质量控制标识;
\item \verb+SAC+为文件后缀;
\end{itemize}

\section{Location ID}
关于Location ID,我并没有找到权威的解释。

常见的Location ID为空值,非空值一般常见到~\verb+00+、\verb+01+、\verb+10+,
偶尔也会见到~\verb+60+等。

Location ID可以认为是台站的subcode,即一个台站处,有多套仪器,这些仪器可能
是相同的型号,但是位于不同的深度或者指向不同的方位;也有可能是不同型号的仪器。

\section{质量控制}
质量控制符Q用于表征当前SAC数据的数据质量。该标识符可以取如下四种:
\begin{itemize}
\item \verb+D+ 不确定状态的数据
\item \verb+M+ 已合并的数据
\item \verb+R+ 原始波形数据
\item \verb+Q+ 经过质量控制的数据
\end{itemize}
常见的数据标识符为H或Q。

\section{通道名}
通道名用三个字符来表示,这三个字符分别代表了Band Code、Instrument Code和Orientation Code。

\subsection{Band Code}
Band Code是通道名的第一个字符,表示了仪器的采样率以及响应频带等信息。

\begin{table}[H]
\centering\small
\caption{Band Code}
\label{tbl:bandcode}
\begin{tabular}{cllc}
\toprule
Band Code   &   Band Type   &   Sample rate (Hz)    & Corner peroid (sec)   \\
\midrule
F           &   ...         &   1000-5000   &     > 10    \\
G           &   ...         &   1000-5000   &     < 10    \\
D           &   ...         &   250-1000    &     < 10    \\
C           &   ...         &   250-1000    &     > 10    \\
E           &   Extremely Short Period  & 80-250    &     < 10    \\
S           &   Short Period          & 10-80   & < 10    \\
H           &   High Broad Band         &   80-250    &   < 10    \\
B           &   Broad Band          &   10-80   & > 10    \\
M           &   Mid Period          &   1-10   & > 10    \\
L           &   Long Period         &   $\approx$ 1   &   \\
V           &   Very  Long Period         & $\approx$ 0.1   &   \\
U           &   Ultra Long Period         & $\approx$ 0.01    &   \\
R           &   Extremely Long Period     & 0.0001-0.001    &   \\
P           &   Order of 0.1 to 1 days   & 0.00001-0.0001    &   \\
T           &   Order of 1 to 10 days    & 0.000001-0.00001    &   \\
Q           &   Greater than 10 days          &     < 0.000001    &   \\
Q           &   Administrative Instrument Channel   & variable    & NA    \\
O           &   Opaque Instrument Channel         & variable    &   NA    \\
\bottomrule
\end{tabular}
\end{table}
常见的仪器一般是宽频带(B)或长周期(L)。

\subsection{Instrument Code}
Instrument Code是通道名的第二个字符,代表了不同的仪器传感器。
\begin{table}[H]
\centering
\caption{Instrument Code}
\begin{tabular}{cl}
\toprule
Instrument Code    &     说明   \\
\midrule
H        &       High Gain Seismometer      \\
L        &       Low Gain Seismometer       \\
G        &       Gravimeter                 \\
M        &       Mass position Seismometer  \\
N        &       Accelerometer              \\
\bottomrule
\end{tabular}
\end{table}
常见的是高增益(H)仪器,记录地面运动速度。

\subsection{Orientation Code}
Orientation Code表示了传感器记录的地面运动的方向。
\begin{table}[H]
\centering
\caption{Orientation Code}
\begin{tabular}{cl}
\toprule
Code     &     说明   \\
\midrule
N E Z   &   南北向、东西向、垂向   \\
1 2 3   &   3为垂向;1、2为水平方向,正交但与东西南北向有偏离   \\
T R Z   &   切向、径向,主要用于射线束中    \\
A B C   &   三轴向(正交)    \\
U V W   &   可选分量    \\
\bottomrule
\end{tabular}
\end{table}
常见的是N、E、Z以及1、2、3。

需要注意的是:当仪器的方向与东西方向的夹角小于5度时,Orientation Code取为E;
当与东西方向夹角大于5度时,Orientation Code取为1(或2)。对于南北方向同理。
因而,即便Orientation Code为N,也并不意味着台站是南北方向的,真实的方向还是
需要看SAC头段中的cmpaz和cpminc。
