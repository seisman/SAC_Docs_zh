\SACCMD{writespe}
\label{spe:writespe}

\SACTitle{概要}
写一个包含谱估计的SAC文件

\SACTitle{语法}
\begin{SACSTX}
W!RITE!SPE! [file]
\end{SACSTX}

\SACTitle{输入}
\begin{description}
\item [file] 要写入的SAC文件名
\end{description}

\SACTitle{缺省值}
\begin{SACDFT}
writespe spe
\end{SACDFT}

\SACTitle{说明}
频谱估计文件包含从0到截止频率的频谱。频谱估计由FFT算法计算得出,文件中的
采样点数是FFT使用的长度的一半再加上1,选择这种格式以使由SPE计算出的多个
频谱能通过 \nameref{cmd:plot2} 绘制函数来进行比较而无需事先为绘图而剪裁文件。

\SACTitle{错误消息}
\begin{itemize}
\item 5004 未计算谱估计
\end{itemize}
