\SACCMD{fileid}
\label{cmd:fileid}

\SACTitle{概要}
控制绘图时文件ID的显示

\SACTitle{语法}
\begin{SACSTX}
FILEID [ON|OFF] [T!YPE! D!EFAULT!|N!AME!|L!IST! hdrlist]
    [L!OCATION! UR|UL|LR|LL] [F!ORMAT! E!QUALS!|C!OLONS!|N!ONAMES!]
\end{SACSTX}

\SACTitle{输入}
\begin{description}
\item [ON] 显示文件id,不改变文件id类型或位置
\item [OFF] 不显示文件id
\item [TYPE DEFAULT] 设置文件id为默认类型
\item [TYPE NAME] 使用文件名作为文件id
\item [TYPE LIST hdrlist] 定义在文件id中显示的头段列表
\item [LOCATION UR|UL|LR|LL] 文件id的显示位置,分别表示右上角、左上角、
    右下角、左下角
\item [FORMAT EQUALS] 格式为 \texttt{variable=value}
\item [FORMAT COLON] 格式为 \texttt{variable:value}
\item [FORMAT NONAMES] 格式只包含头段值
\end{description}

\SACTitle{缺省值}
\begin{SACDFT}
fileid on type default location ur format nonames
\end{SACDFT}

\SACTitle{说明}
文件ID用于标识绘图的内容。默认的文件ID包括事件名、台站名、分量、参考
日期及时间。如果需要也可以使用文件名代替默认的文件id,或者根据头段变量
定义一个特殊的文件ID,这个ID最多可以由10个SAC头段变量构成。文件ID的位置
以及格式也可以修改。

\SACTitle{示例}
将文件名放在左上角:
\begin{SACCode}
SAC> fileid location ul type name
\end{SACCode}

定义一个特殊的文件id,包含台站分量、经纬度:
\begin{SACCode}
SAC> fileid type list kstcmp stla stlo
\end{SACCode}

文件id为头段名后加一个冒号:
\begin{SACCode}
SAC> fileid format colon
\end{SACCode}
