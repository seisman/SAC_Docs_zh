\section{SAC debug}
不管是在SAC中执行稍复杂的命令,还是写SAC宏文件,又或者是在脚本中调用SAC,
都会遇到各种各样奇怪的错误,明明仔细检查了很多遍,执行起来还是有问题。
这个时候先不要怀疑SAC有bug,而是要先把自己写的东西好好debug一下。

SAC其实自带了一个debug工具,即命令 \nameref{cmd:echo}。该命令用于控制
是否显示SAC的输出(即警告信息、错误信息和正常的输出信息)以及输入(包括
传递给SAC的命令、宏以及对命令的处理)。

以在Perl中调用SAC为例:
\begin{minted}{perl}
#!/usr/bin/env perl
use strict;
use warnings;

my $var0 = 3.0;
my $var1 = 5.0;
open(SAC, "|sac");
print SAC "r STA.BHN STA.BHE STA.BHZ \n";
print SAC "ch t0 $var0+$var1\n";
print SAC "w over";
print SAC "q\n";
close(SAC);
\end{minted}

这个脚本是有问题的,但是对于刚刚写脚本的人来说,可能看不出问题。直接执行会出现
如下错误:
\begin{SACCode}
 ERROR 1312: Bad number of files in write file list: 1 3
SAC Error: EOF/Quit
     SAC executed from a script: quit command missing
     Please add a quit to the script to avoid this message
     If you think you got this message in error,
     please report it to: sac-help@iris.washington.edu
\end{SACCode}
从这些输出信息里其实看不出来太多有用的信息。

如果加上``\texttt{echo on}'',脚本如下:
\begin{minted}{perl}
#!/usr/bin/env perl
use strict;
use warnings;

my $var0 = 3.0;
my $var1 = 5.0;
open(SAC, "|sac");
print SAC "echo on\n";      # 加了这一行
print SAC "r STA.BHN STA.BHE STA.BHZ \n";
print SAC "ch t0 $var0+$var1\n";
print SAC "w over";
print SAC "q\n";
close(SAC);
\end{minted}

运行结果如下:
\begin{SACCode}
 r STA.BHN STA.BHE STA.BHZ
 ch t0 3+5
 w overq
 ERROR 1312: Bad number of files in write file list: 1 3
SAC Error: EOF/Quit
     SAC executed from a script: quit command missing
     Please add a quit to the script to avoid this message
     If you think you got this message in error,
     please report it to: sac-help@iris.washington.edu
 quit
\end{SACCode}
此时会显示所有脚本传递给SAC来执行的命令,从中,可以很明显地看到两个错误。

一个是``\verb|print SAC "ch $var0+$var1\n"|'',由于使用的是 \texttt{print} 语句,perl会直接做
变量替换然后把结果传递给SAC,因而真正传递给SAC的是 \texttt{ch t0 3+5},而不是想象
中的 \texttt{ch t0 8} ,这个错误可以通过使用 \texttt{printf} 语句来解决。

另一个是,由于 \texttt{print SAC "w over"} 语句中忘了加换行符,导致实际传递给SAC的
不是 \texttt{w over} ,而是 \texttt{w overq} ,即内存中有三个波形文件,而write命令中
却只给了一个文件名,因而出现了错误1312。

由此可见,\nameref{cmd:echo} 命令可以帮助用户清楚地知道真正传递给SAC的是什么,
因而是一个很好的SAC调试工具。