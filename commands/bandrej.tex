\SACCMD{bandrej}
\label{cmd:bandrej}

\SACTitle{概要}
应用一个无限脉冲带阻滤波器

\SACTitle{语法}
\begin{SACSTX}
B!AND!R!EJ! [BU!TTER!|BE!SSEL!|C1|C2] [C!ORNERS! v1 v2] [N!POLES! n] [P!ASSES! n]
    [T!RANBW! v] [A!TTEN! v]
\end{SACSTX}

\SACTitle{输入}
\begin{description}
\item [BUTTER] 应用一个Butterworth滤波器
\item [BESSEL] 应用一个Bessel滤波器
\item [C1] 应用一个Chebyshev I型滤波器
\item [C2] 应用一个Chebyshev II滤波器
\item [CORNERS v1 v2] 设定拐角频率分别为v1和v2,即频率通带为v1--v2
\item [NPOLES n] 设置极数为N,范围:1--10
\item [PASSES n] 设通道数为N,范围:1--2
\item [TRANBW v] 设置Chebyshev转换带宽为v
\item [ATTEN v] 设置Chebyshev衰减因子为v
\end{description}

\SACTitle{缺省值}
\begin{SACDFT}
bandrej butter corner 0.1 0.4 npoles 2 passes 1 tranbw 0.3 atten 30
\end{SACDFT}

\SACTitle{说明}
参见命令 \nameref{cmd:bandpass} 的说明

\SACTitle{示例}
应用一个四极Butterworth滤波器,拐角频率为 \SI{2}{\Hz} 和 \SI{5}{\Hz}:
\begin{SACCode}
SAC> br n 4 c 2 5
\end{SACCode}

在此之后如果要应用一个二极双通具有相同频率的Bessel:
\begin{SACCode}
SAC> br n 2 be p 2
\end{SACCode}

\SACTitle{头段变量改变}
depmin, depmax, depmen

\SACTitle{相关命令}
\nameref{cmd:bandpass}
