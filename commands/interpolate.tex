\SACCMD{interpolate}
\label{cmd:interpolate}

\SACTitle{概要}
对等间隔或不等间隔数据进行插值以得到新采样率

\SACTitle{语法}
\begin{SACSTX}
INTERP!OLATE! [D!ELTA! v|N!PTS! v] [B!EGIN! v]
\end{SACSTX}

\SACTitle{输入}
\begin{description}
\item [DELTA v] 设置新采样率为 \texttt{v}。数据的时间跨度(E-B)保持不变,
    \texttt{npts}变化,\texttt{E} 由于需要与 \texttt{b} 的间距为 \texttt{delta}
    的整数倍,所以可能会有微调
\item [NPTS n] 强制设置插值后文件的数据点数为 \texttt{n}。时间宽度不变,
    \texttt{delta} 发生变化。
\item [BEGIN v] 在 \texttt{v} 处开始插值,该值将作为插值文件的起始时间。
    \texttt{BEGIN} 可以和 \texttt{DELTA} 或 \texttt{NPTS} 选项一起使用。
\end{description}

\SACTitle{说明}
该命令使用Wiggins的weighted average-sloped插值方法将不等间隔数据转换为
等间隔数据,以及对等间隔数据插值得到新的采样率。不像三次样条插值,在输入
样本数据点间不会存在极值。如果要降低采样率,即减采样,由于该命令没有
抗混叠滤波器,所以最好使用 \nameref{cmd:decimate} 命令。

\texttt{DELTA} 选项和 \texttt{NPTS} 选项只能同时使用一个,若二者同时使用,
则命令中的后者起作用。

\texttt{BEGIN} 选项用于控制输入数据的插值起点,也可以通过 \nameref{cmd:cut}
命令设置 \texttt{b} 和 \texttt{e} 再进行插值操作。

\SACTitle{示例}
假定filea是等间隔数据,采样间隔为 \SI{0.025}{\s},为了将转换到采样间隔为
\SI{0.02}{\s}:
\begin{SACCode}
SAC> r filea
SAC> interp delta 0.02
\end{SACCode}
由于新 \texttt{delta} 小于原数据 \texttt{delta},可能会出现混叠现象,
所以会输出警告信息。

假定fileb数据点数为3101,想要保持其时间跨度,并采样至4096个点:
\begin{SACCode}
SAC> r fileb
SAC> interp npts 4096
\end{SACCode}

假设filec是不等间隔数据,为了将其转换为采样率为 \SI{0.01}{\s} 的等间隔数据:
\begin{SACCode}
SAC> read filec
SAC> interpolate delta 0.01
\end{SACCode}

\SACTitle{头段变量}
delta、npts、e、b、leven
