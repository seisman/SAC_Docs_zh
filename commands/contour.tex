\SACCMD{contour}
\label{cmd:contour}

\SACTitle{概要}
利用内存中的数据绘制等值线图

\SACTitle{语法}
\begin{SACSTX}
CONT!OUR! [A!SPECT! ON|OFF]
\end{SACSTX}

\SACTitle{输入}
\begin{description}
\item [ASPECT ON] 打开视图比开关。当这个开关打开时,等值线图的视口将会调整保持数据中y与x的比值
\item [ASPECT OFF] 关闭视图比开关,这时将使用整个视口。
\end{description}

\SACTitle{缺省值}
\begin{SACDFT}
contour  aspect  off
\end{SACDFT}

\SACTitle{说明}
这个命令用于绘制二维数组数据的等值线图,包括 \nameref{cmd:spectrogram}
命令的输出。这个文件
操作的SAC文件必须``XYZ''类型的(SAC头段中IFTYPE为``IXYZ'')。有些命令可以
控制数据显示的方式:
\begin{itemize}
\item \nameref{cmd:zlevels} 控制等值线的数目以及间隔
\item \nameref{cmd:zlines} 控制等值线的线型
\item \nameref{cmd:zlabels} 控制等值线标签
\item \nameref{cmd:zticks} 控制方向标记
\item \nameref{cmd:zcolors} 控制等值线颜色
\end{itemize}

根据contour选项的不同,有两种不同的绘制等值线算法。一种快速扫描方法用于既不
选择实线型也没有时标和标识的情况。另一种慢一点的方法,在绘图之前要组合
全部的线段。你可以使用快速扫描方法粗看你的数据,然后选择其他选项绘制最终
图形。

\SACTitle{示例}
参见``\nameref{sec:contour}''一节。

\SACTitle{头段变量改变}
要求:iftype (为``IXYZ''), nxsize, nysize

使用:xminimum, xmaximum, yminimum, ymaximum

\SACTitle{相关命令}
\nameref{cmd:zcolors}、\nameref{cmd:zlabels}、\nameref{cmd:zlevels}、
\nameref{cmd:zlines}、\nameref{cmd:zticks}、\nameref{cmd:spectrogram}
