\section{SAC头段结构}
SAC头段区包含了一系列头段变量,从这些头段变量中可以了解到波形
记录的很多信息,比如台站经纬度、发震时刻、震相到时等等。
表 \ref{table:header-variables} 列出了SAC头段区的全部头段变量。

\begin{table}[H]
\ttfamily
\small
\centering
\caption{SAC头段变量列表}
\label{table:header-variables}
\begin{tabular}{c|c|lllll}
\toprule
Byte	&	Type	&	\multicolumn{5}{c}{Names}\\
\midrule
0		&	F	&	delta	&	depmin	&	depmax	&	scale	&	odelta	\\
20		&	F	&	b		&	e		&	o		&	a		&	internal\\
40		&	F	&	t0		&	t1		&	t2		&	t3		&	t4		\\
60		&	F	&	t5		&	t6		&	t7		&	t8		&	t9		\\
80		&	F	&	f		&	resp0	&	resp1	&	resp2	&	resp3	\\
100		&	F	&	resp4	&	resp5	&	resp6	&	resp7	&	resp8	\\
120		&	F	&	resp9	&	stla	&	stlo	&	stel	&	stdp	\\
140		&	F	&	evla	&	evlo	&	evel	&	evdp	&	mag		\\
160		&	F	&	user0	&	user1	&	user2	&	user3	&	user4	\\
180		&	F	&	user5	&	user6	&	user7	&	user8	&	user9	\\
200		&	F	&	dist	&	az		&	baz		&	gcarc	&	internal\\
220		&	F	&	internal&	depmen	&	cmpaz	&	cmpinc	&	xminimum\\
240		&	F	&	xmaximum&	yminimum&	ymaximum&	unused	&	unused	\\
260		&	F	&	unused	&	unused	&	unused	&	unused	&	unused	\\
280		&	N	&	nzyear	&	nzjday	&	nzhour	&	nzmin	&	nzsec	\\
300		&	N	&	nzmsec	&	nvhdr	&	norid	&	nevid	&	npts	\\
320		&	N	&	internal&	nwfid	&	nxsize	&	nysize	&	unused	\\
340		&	I	&	iftype	&	idep	&	iztype	&	unused	&	iinst	\\
360		&	I	&	istreg	&	ievreg	&	ievtyp	&	iqual	&	isynth	\\
380		&	I	&	imagtyp &	imagsrc	&	unused	&	unused	&	unused	\\
400		&	I	&	unused	&	unused	&	unused	&	unused	&	unused	\\
420		&	L	&	leven	&	lpspol	&	lovrok	&	lcalda	&	unused	\\
440		&	K	&	kstnm	&	kevnm*	&			&			&			\\
464		&	K	&	khole	&	ko		&	ka		&			&			\\
488		&	K	&	kt0		&	kt1		&	kt2		&			&			\\
512		&	K	&	kt3		&	kt4		&	kt5		&			&			\\
536		&	K	&	kt6		&	kt7		&	kt8		&			&			\\
560		&	K	&	kt9		&	kf		&	kuser0	&			&			\\
584		&	K	&	kuser1	&	kuser2	&	kcmpnm	&			&			\\
608		&	K	&	knetwk	&	kdatrd	&	kinst	&			&			\\
\bottomrule
\end{tabular}
\end{table}

整个头段区,共有头段变量133个,占632个字节。头段区的前四个字节是第一个
头段变量 !delta!,第5--8个字节是第二个头段变量 !depmin!,
第21--24个字节是第6个头段变量 !b!,以此类推。

表的第一列给出了当前行的第一个头段变量在文件中的起始字节,第二列给出了
当前行的头段变量的变量类型。

表 \ref{table:header-variables-type} 列出了SAC头段中的头段变量类型及其
相关信息。第一列为头段变量类型代码,第二类给出了其代表的头段变量类型,
第三列指出C源码中该变量的是用什么类型定义的,第四列给出了每个变量所占据
的字节数,第五列给出了写字符型SAC文件时的输出格式,最后一列则给出该类型
的未定义值。

\begin{table}[H]
\caption{变量类型说明}
\label{table:header-variables-type}
\centering
\ttfamily
\small
\begin{tabular}{cllcll}
\toprule
Code    &	Type        &   C Type & sizeof &   printf	&   未定义值        \\
\midrule
F		&	浮点型		&   float  &  4     &	\%15.7f &   -12345.0        \\
N		&	整型		&   int    &  4     &	\%10d   &   -12345        \\
I		&	枚举型		&   int    &  4     &	\%10d   &   -12345	        \\
L		&	逻辑型		&   int    &  4     &	\%10d   &   FALSE        \\
K		&	字符型		&   char*  &  8     &	\%-8.8s & \lstinline[showspaces=true]!"-12345  "!     \\
A		&	辅助型		&          &        &			& 	    \\
\bottomrule
\end{tabular}
\end{table}

说明:
\begin{itemize}
\item 所有头段变量的变量名均以变量类型码开头,比如 !nvhdr! 是
    N型变量,!leven! 是L型变量,但F型变量除外;
\item 枚举型变量本质上是 !int! 型,只能在固定的几个值中取值
\item 逻辑型变量可以取值 !TRUE! 和 !FALSE!,本质上分别是
    整型的1和0
\item 字符型变量长度为8\footnote{C语言中用``!\0!''作为字符串的
    结束标识符,因而源码中变量的实际长度为9。},只有 !kevnm!
    很特殊,其长度为16;
\item 变量名为 !internal! 表示该变量是SAC内部使用的头段变量,
    用户不可对其进行操作;
\item 变量名为 !unused! 表示该变量尚未使用,为以后可能出现的
    新头段变量占位;
\item 当某个头段变量未定义时,其包含未定义值;不同类型的头段变量有不同
    的未定义值;若一个整型头段变量的值为 !-12345!,则认为该变量
    未定义;实际使用时,可以直接用 !undef! 表示所有类型的头段变量
    的未定义值,SAC会根据头段变量的类型自动将其转换成相应类型的未定义值。
\item 辅助型变量并不在SAC头段区中,而是从其它头段变量推导得到的;
\end{itemize}
