\SACCMD{systemcommand}
\label{cmd:systemcommand}

\SACTitle{概要}
在SAC内部执行系统命令

\SACTitle{语法}
\begin{SACSTX}
S!YSTEM!C!OMMAND! command [ options ]
\end{SACSTX}

\SACTitle{输入}
\begin{description}
\item [command] 系统命令名
\item [options] 命令需要的选项
\end{description}

\SACTitle{说明}
在SAC中是可以执行大部分系统命令的,比如常见的~\verb+ls+、\verb+cp+等。

但是某些命令无法直接在SAC中执行,比如用于查看PS文件的gs命令会首先被SAC解释为
\nameref{cmd:grayscale}~的简写,故而在SAC中无法直接调用gs命令。

另一个经常使用但无法直接调用的命令是~\verb+rm+。由于~\nameref{cmd:read}~命令
可以被简写为~\verb+r+,在读入文件时键入~\verb+r *.SAC+很可能会一时手误
敲成~\verb+rm *.SAC+,为了避免类似的误操作,故而在SAC中禁止直接调用~\verb+rm+命令。

当需要在SAC内部执行删除命令时,则需要使用~\nameref{cmd:systemcommand}~调用系统命令。

\SACTitle{示例}
调用系统命令删除某些SAC文件:
\begin{SACCode}
SAC> rm junks
 ERROR 1106: Not a valid SAC command.
SAC> sc rm junks
\end{SACCode}
