\section{数据截窗}
相关命令:\nameref{cmd:cut}、\nameref{cmd:cutim}

数据申请时一般会选择尽可能长的时间窗,而实际进行数据处理和分析时可能只需要其中
的一小段时间窗,这就需要对数据时间窗进行截取。

SAC中有两个命令可以用于数据截窗,分别是~\nameref{cmd:cut}~和~\nameref{cmd:cutim},
二者的语法基本是相同的,但前者是``参数设定类''命令,后者是``操作执行类''命令。

除此之外,还有其他命令也会需要定义时间窗,比如rms、mtw、xlim等,SAC中使用pdw来
定义时间窗。

\subsection{pdw}
\label{subsec:pdw}
pdw即partial data window,其定义了一个开始时间和一个结束时间,格式为
\verb+ref offset ref offset+。其中ref为参考时刻,可以取~\verb+Z|B|E|O|A|F|N|Tn+,
而offset为相对于参考时刻的时间偏移量。

参考时刻ref可以取如下值:
\begin{itemize}
\item B: 磁盘文件起始值
\item E: 磁盘文件结束值
\item O: 事件开始时间
\item A: 初动到时
\item F: 信号结束时间
\item Tn: 用户自定义时间标记,n = 0,1...9
\item Z: 参考时刻
\item N: 将offset解释为数据点数而非时间偏移量,其仅可以用于结束值
\end{itemize}

若开始或结束的offset省略则认为其值为0。若开始ref省略则认为其为Z;若结束ref省略则
认为其值与开始ref相同。

下面的例子中展示了一些常见的pdw及其含义:
\begin{SACCode}
 B E        // 文件开始到文件结束,即与cut off相同
 B 0 30     // 文件开始的30秒
 A -10 30   // 初动前10秒到初动后30秒
 B N 2048   // 文件最初的2048个点
 30.2 48    // 相对磁盘文件0点的30.2到48秒
\end{SACCode}

\subsection{cut}
\nameref{cmd:cut}命令是``参数设定类''命令,因而需要先cut再read:
\begin{SACCode}
SAC> cut t0 -5 5        // 截取t0前后各5秒,共计10秒的数据
SAC> r *.SAC            // 先cut再read
\end{SACCode}

\nameref{cmd:cutim}命令是``操作执行类''命令,因而需要先read再cutim:
\begin{SACCode}
SAC> r *.SAC
SAC> cutim t0 -5 5
\end{SACCode}

注意:似乎cutim有bug,请勿使用。
