\SACCMD{zlabels}
\label{cmd:zlabels}

\SACTitle{概要}
根据等值线的值控制等值线的标记

\SACTitle{语法}
\begin{SACSTX}
ZLABELS [ON|OFF] [SPACING v1 [v2 [v3]]] [SIZE v] [ANGLE v] [LIST c1 c2 ... cn]
\end{SACSTX}
!LIST! 选项只能放在这个命令的最后

\SACTitle{输入}
\begin{description}
\item [ON|OFF] 打开/关闭等值线标签选项开关
\item [SPACING v1 v2 v3] 设置相邻标签名的最小、适中和最大间隔(视口坐标系)
    分别为 !v1!、!v2! 和 !v3!。如果第二、三个值省略
    则使用前面一个值
\item [SIZE v] 设置标签的尺寸(高度)为 !v!
\item [ANGLE v] 设置标签文本最大角度为 !v!(自水平方向起算的角度,
    单位为度)
\item [LIST c1 c2 . cn] 设置使用的等值线标签的列表。在这个表上的每个输入
    用于相应的等值线,如果等值线数目大于这个表的长度,则重复使用整个等值线表
\item [cn]  可以取 !ON|OFF|INT|FLOATn|EXPn|text!
\item [ON] 在相应的等值线上放置标签,使用Fortran自由格式,用等值线值形成标签名
\item [OFF] 在相应的等值线上不放置标签名
\item [INT] 在相应的等值线上放置整数标签名
\item [FLOATn] 在相应的等值线上放置小数点后面n位的浮点数作为标签名。
    如果n被忽略则使用先前值
\item [EXPn] 在相应的等值线上放置小数点后面n位数的指数幂形式标签名,
    如果n忽略则使用先前值
\item [text] 使用文本标注相应的等值线
\end{description}

\SACTitle{缺省值}
\begin{SACDFT}
zlabels off spacing 0.1 0.2 0.3 size 0.0075 angle 45.0 list on
\end{SACDFT}

\SACTitle{示例}
参考``\nameref{sec:contour}''中的相关示例。
