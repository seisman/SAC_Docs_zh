\SACCMD{divomega}
\label{cmd:divomega}

\SACTitle{概要}
在频率域进行积分操作

\SACTitle{语法}
\begin{SACSTX}
DIVOMEGA
\end{SACSTX}

\SACTitle{说明}
根据傅里叶变换的微分性质:
\[
\mathcal{F}[f'(x)]= i \omega \mathcal{F}[f(x)]
\]
其中$\omega = 2 \pi f $,即函数积分在频率域可以用简单的除法来表示。

该命令仅可对谱文件进行操作,谱文件可以是振幅-相位型或实部-虚部型。
对于正常的谱数据来说还是很方便的,但不适用于谱跨越几个量级的数据。
比如,假设你使用 \texttt{dif} 命令对数据进行预白化,然后对数据进行
Fourier变换,用此命令在频率域积分可以去除时间域微分的效应。

若为振幅-相位型:
\[
\mathcal{F}[f(x)] = \frac{\mathcal{F}[f'(x)]}{i \omega}
                  = \frac{A(\omega)e^{\theta(\omega)}}{i \omega}
                  = \frac{A(\omega)}{\omega}e^{\theta(\omega)-\pi/2}
\]

若为实部-虚部型:
\[
\mathcal{F}[f(x)] = \frac{\mathcal{F}[f'(x)]}{i \omega}
                  = \frac{a(\omega)+ib(\omega)}{i \omega}
                  = \frac{b(\omega)}{\omega}-i\frac{a(\omega)}{\omega}
\]

在零频部分,直接设置其值为0比避免除以0的问题。

\SACTitle{示例}
\begin{SACCode}
SAC> read file1
SAC> dif                // 微分预白化
SAC> fft amph           // FFT
SAC> divomega           // 积分
\end{SACCode}

\SACTitle{头段变量}
depmin、depmax、depmen
