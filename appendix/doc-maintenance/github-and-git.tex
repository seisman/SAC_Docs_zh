\section{GitHub和Git}
Git是一个非常强大的版本管理工具,GitHub则是一个基于Git的开源项目托管库。

本手册使用Git作为版本管理工具,并将源码托管在GitHub上。因而用户需要
学会使用GitHub和Git才能更好地参与到项目的维护中来。

\subsection{熟悉GitHub}
维护者需要首先\href{https://github.com/join}{注册}一个GitHub账户,
并向GitHub账户中添加当前机器的SSH秘钥,以使得当前机器拥有GitHub账户的
读写权限。具体流程可以参考
\url{http://www.worldhello.net/gotgithub/02-join-github/010-account-setup.html}。

\subsection{安装和配置Git}
Linux下直接用系统自带的包管理器即可安装 \texttt{git},同时也建议安装
用于用于查看git提交信息的可视化工具 \texttt{gitk}。

CentOS/RHEL用户:
\begin{minted}{console}
$ sudo yum install git gitk
\end{minted}

Ubuntu/Debian用户:
\begin{minted}{console}
$ sudo apt-get install git gitk
\end{minted}

安装完git之后还需要告诉git你的名字和邮箱,这些信息会出现在每次的提交历史中:
\begin{minted}{console}
$ git config --global user.name "Your Name"
$ git config --global user.email "you@example.com"
\end{minted}

\subsection{准备工作}
在准备维护该项目之前,建议先阅读相关书籍以了解git的原理以及GitHub的
使用说明。推荐的参考书目包括(由易到难排序):
\begin{itemize}
\item \href{http://rogerdudler.github.io/git-guide/index.zh.html}{git简明指南}
\item \href{http://www.worldhello.net/gotgithub/index.html}{GotGitHub}
\item \href{http://www.liaoxuefeng.com/wiki/0013739516305929606dd18361248578c67b8067c8c017b000}{廖雪峰的Git教程}
\item \href{https://git-scm.com/book/zh/v2}{Pro Git}
\end{itemize}

首先,进入该手册的项目主页 \url{https://github.com/seisman/SAC_Docs_zh},
点击右上角的Fork按钮,将该项目复制到你的GitHub账户下。

下面假定你的GitHub账户名为 \texttt{USER},在终端执行如下操作:
\begin{enumerate}
\item Clone源码到当前机器
\begin{minted}{console}
$ git clone git@github.com:USER/SAC_Docs_zh.git
$ cd SAC_Docs_zh/
\end{minted}
\item 将seisman账户下的官方repo添加为远程repo,并命名为seisman,
    这样方便以后与官方repo同步进度
\begin{minted}{console}
$ git remote add seisman https://github.com/seisman/SAC_Docs_zh.git
\end{minted}
\end{enumerate}

\subsection{正式维护该项目}
至此,可以开始正式维护该项目了。使用Git进行协作的方式有很多中,这里
参考了阮一峰的\href{http://www.ruanyifeng.com/blog/2015/08/git-use-process.html}{Git使用规范流程}
一文中的协作方式。请按照如下流程参与到该项目的维护中:
\begin{enumerate}
\item 从seisman的官方repo中拉取源码的最新版本,并合并到本地master分支,
    以保证本地的master分支与官方master分支同步
\begin{minted}{console}
# 切换到本地的master分支
$ git checkout master
# pull = fetch + merge
$ git pull seisman master
\end{minted}
\item 不要在master分支中修改文档。对文档进行修改,应新建一个单独在单独的
    分支。分支名任意,但应尽量反映要维护的内容。这里假定分支名为 \texttt{mydev}
\begin{minted}{console}
$ git checkout -b mydev
\end{minted}
\item 对文档做修改,并提交commit,此过程可以循环多次
\begin{minted}{console}
$ git status
$ git add --all
$ git status
$ git commit -m "此处填写本次提交的注释信息"
\end{minted}
\item 分支开发的过程中,可能seisman的官方master分支已经更新,可以经常
    与主干保持同步
\begin{minted}{console}
# 获取seisman的更新
$ git fetch seisman
# rebase使得当前分支的提交基于最新的seisman/master
$ git rebase seisman/master
\end{minted}
\item 分支开发的过程中,可能会有很多次commit,某些commit可能不那么重要,
    可以将多个commit压缩成一个或若干个commit,这样不仅清晰,也容易管理
\begin{minted}{console}
# 以seisman/master作为基准,对当前分支的commit做rebase
$ git rebase -i seisman/master
\end{minted}
rebase操作相对比较复杂,可以参考前面提到的阮一峰的博文或者其他git相关书籍。
\item 将分支推送到远程仓库
\begin{minted}{console}
# 将mydev分支推送到USER的repo下
$ git push -u origin mydev
\end{minted}
\item 进入 \url{https://github.com/USER/SAC_Docs_zh},点击Pull Request即可提交PR
\item seisman在收到PR后,会对代码进行审核、评论以及修改,并决定是否结束该PR
\item 若PR已被接受,则可以删除本地和GitHub上的mydev分支
\begin{minted}{console}
# 删除本地分支
$ git branch -D mydev
# 删除GitHub上的远程分支,也可以在GitHub上点击按钮删除分支
$ git push origin :mydev
\end{minted}
\end{enumerate}

\subsection{对维护的若干说明}
为了简化维护流程,避免重复或不必要的劳动,请遵循如下原则。

若只是对文档做简单的微调,比如修改简单的bug或typo,整理部分语句等,
可以直接修改并提交PR。若需要对文档做大量修改,比如新增章节、调整文档
结构等,请先到项目主页中提交Issue,由众多维护者一起讨论是否有必要做修改。

若暂时不打算解决某Issue,则该Issue会有标签``Pull Request Welcomed'',
维护者可以随意认领具有``Pull Request Welcomed''标签的Issue,若某Issue
已经被认领,则设置标签为``In Progress''。标签的设置和修改只能由seisman
完成。
