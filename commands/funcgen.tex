\SACCMD{funcgen}
\label{cmd:funcgen}

\SACTitle{概要}
生成一个函数并将其存在内存中

\SACTitle{语法}
\begin{SACSTX}
F!UNC!G!EN! [type] [D!ELTA! v] [N!PTS! n] [BE!GIN! v]
\end{SACSTX}
其中 \texttt{type} 是下面中的一个:
\begin{SACSTX}
IMP!ULSE! | ST!EP! | B!OXCAR! | T!RIANGLE! | SINE [v1 v2] | L!INE! [v1 v2] |
Q!UADRATIC! [v1 v2 v3] | CUBIC [v1 v2 v3 v4] | SEIS!MOGRAM! |
R!ANDOM! [v1 v2] | IMPSTRIN  [n1 n2 ... nN]
\end{SACSTX}

\SACTitle{输入}
\begin{description}
\item [IMPULSE] 位于时间序列中点的脉冲函数
\item [IMPSTRIN n1 n2 ... nN] 在指定的一系列数据点处产生脉冲函数
\item [STEP] 阶跃函数。数据的前半段为0,后半段为1
\item [BOXCAR] 矩形函数。数据的前、后三分之一值为0,中间三分之一值为1
\item [TRIANGLE] 三角函数。数据的第一个四分之一值为0,第二个四分之一的
    值从0线性增加到1,第三个四分之一的值从1线性减少到0,最后四分之一值为0
\item [SINE v1 v2] 正弦函数。\texttt{v1} 表示频率,单位为 \si{\Hz};
    \texttt{v2} 是以度为单位的相位角。正弦函数的振幅为1,
    注意在相位参数中有一个$2\pi$因子:$F = 1.0 \sin (2\pi (v_1t+v_2))$
\item [LINE v1 v2] 线性函数。斜率为 \texttt{v1},截距为 \texttt{v2},
    即$ v_1 t + v_2 $
\item [QUADRATIC v1 v2 v3] 二次函数 $v_1 t^{2} + v_2 t + v_3 $
\item [CUBIC v1 v2 v3 v4] 三次函数 $ v_1 t^{3} + v_2 t^2 + v_3t + v_4 $
\item [SEISMOGRAM] 地震样本数据。此样本数据有1000个数据点。\texttt{DELTA}、
    \texttt{NPTS} 和 \texttt{BEGIN} 选项对该样本数据无效
\item [RANDOM v1 v2] 生成随机序列(高斯白噪声)。\texttt{v1} 是要生成的
    随机序列文件的数目,\texttt{v2} 是用于产生第一个随机数的``种子'',
    该种子值保存在 \texttt{USER0} 中,因而如果需要你可以在稍后生成一个
    完全相同的随机序列
\item [DELTA v] 设置采样周期为 \texttt{v},储存在头段 \texttt{delta} 中
\item [NPTS n] 设置函数的数据点数为 \texttt{n},储存在头段 \texttt{npts} 中
\item [BEGIN v] 设置起始时间为 \texttt{v},储存在头段 \texttt{b} 中
\end{description}

\SACTitle{缺省值}
\begin{SACDFT}
funcgen impulse npts 100 delta 1.0 begin 0.
\end{SACDFT}
对于正弦函数频率和相位缺省值分别为0.05和0。
一次、二次、三次函数的系数都是1。
随机序列数为1,种子是12357。

\SACTitle{说明}
执行这个命令等效于读取单个文件(\texttt{RANDOM} 选项会生成多个文件)
到内存中,文件名即为函数名。内存中原有的数据会被该命令生成的函数所替换。
