\section{IRIS地震数据申请工具}
从IRIS申请地震波形数据时,有很多工具可供选择与使用。这一节不会去介绍工具
的具体用法,而是试着总结不同工具各自的优缺点以及适用范围。读者应根据实际
需求选择合适的工具,并自行阅读相应工具的说明文档。

\subsection{Wilber III}
主页:\url{http://ds.iris.edu/wilber3/find_event}

教程:\url{http://seisman.info/wilber3.html}

\begin{itemize}
\item 适用范围:仅用于申请基于事件的事件波形数据
\item 特色及优点:
    \begin{itemize}
    \item 网页端
    \item 提供地震目录,筛选条件:经纬度范围、发震时刻、震级范围
    \item 基于Google地图服务显示地震分布
    \item 台站筛选条件:台网名、虚拟台网名、通道名、震中距范围、方位角范围
    \item 基于Google地图服务显示台站分布
    \item 提供波形预览
    \item 根据发震时刻、P波到时、S波到时确定数据时间窗
    \item 数据格式:SAC、SEED、miniSEED、ASCII、dataless SEED
    \item 申请得到的数据位于IRIS的FTP中
    \item 申请得到的数据已经写入事件信息
    \end{itemize}
\item 缺点:
    \begin{itemize}
    \item 一次只能下载一个事件的波形数据,难以自动化
    \item 只能用于基于事件的波形数据,无法用于基于台站的波形数据或
        连续波形数据
    \end{itemize}
\end{itemize}

\subsection{BREQ\_FAST}
主页:\url{https://ds.iris.edu/ds/nodes/dmc/manuals/breq_fast/}

教程:\url{http://seisman.info/iris-breq-fast.html}

\begin{itemize}
\item 适用范围:连续波形数据
\item 特色及优点:
    \begin{itemize}
    \item 发送特定格式的邮件到特定邮箱即可申请数据
    \item 邮件格式相对简单
    \item 易于自动化和批量处理
    \item 数据格式:发送到不同的邮箱可分别得到SEED、miniSEED和
        dataless格式的数据
    \item 申请的数据位于IRIS的FTP里
    \end{itemize}
\item 缺点:需要一定的编程基础
\end{itemize}

\subsection{DMC Web Service}
主页:\url{http://service.iris.edu/fdsnws/dataselect/1/}

\begin{itemize}
\item 适用范围:连续波形数据
\item 特色及优点:
    \begin{itemize}
    \item 基于Web的服务,是其他多个工具的基础
    \item 按照固定的格式构建一个URL贴到浏览器中即可申请相应的数据
    \item 数据格式:miniSEED
    \end{itemize}
\item 缺点:Web服务太原始,仅在个别情况下比较适用,大多数情况下都需要
    编程调用该服务
\end{itemize}

\subsection{JWEED}
主页:\url{https://ds.iris.edu/ds/nodes/dmc/software/downloads/jweed/}

\begin{itemize}
\item 适用范围:基于事件的事件波形数据+基于台站的事件波形数据
\item 特色及优点:
    \begin{itemize}
    \item Java语法写的GUI客户端
    \item 基于DMC Web Service
    \item 跨平台,但兼容性较差,某些平台下无法正常使用
    \item 运行速度稍慢
    \item 地震目录类型:NEIC PDE、GCMT、ISC、ANF
    \item 地震事件筛选条件:时间范围、震级范围、深度范围
    \item 台站筛选条件:台网名、虚拟台网名、通道名
    \item 地震台站对筛选条件:方位角范围、反方位角范围、震中距范围
    \item 数据格式:SAC、RESP、PZ、ASCII、miniSEED
    \item 可以生成 \verb|BREQ_FAST| 格式的文件,供发邮件申请数据
    \item 数据会直接下载保存到本地
    \end{itemize}
\item 缺点:无法下载连续数据
\end{itemize}

\subsection{Web Service fetch scripts}
主页:\url{https://seiscode.iris.washington.edu/projects/ws-fetch-scripts}

\begin{itemize}
\item 适用范围:连续波形数据
\item 特色及优点:
    \begin{itemize}
    \item Perl语言写的脚本,在命令行使用
    \item 基于DMC Web Service
    \item 数据格式:miniSEED
    \end{itemize}
\item 缺点:仅适用于连续波形数据。相对于原始的Web Service的优点在于,
    可编程实现批量申请
\end{itemize}

\subsection{irisfetch.m}
主页:\url{https://ds.iris.edu/ds/nodes/dmc/software/downloads/irisfetch.m/}

\begin{itemize}
\item 适用范围:连续波形数据
\item 特色及优点:
    \begin{itemize}
    \item Matlab脚本,可以在Matlab中直接调用相关函数获取数据
    \item 基于DMC Web Service
    \item 数据格式:保存为Matlab自定义结构体
    \end{itemize}
\end{itemize}

\subsection{SOD}
主页:\url{http://www.seis.sc.edu/sod/}

\begin{itemize}
\item 适用范围:基于事件的事件波形数据+基于台站的事件波形数据
\item 特色及优点:
    \begin{itemize}
    \item 命令行工具,易于批量处理
    \item 申请数据的同时可以对数据进行预处理
    \item 数据格式:SAC、miniSEED
    \item 可生成 \texttt{BREQ_FAST} 格式的文件
    \end{itemize}
\item 缺点:学习成本较高
\end{itemize}

\subsection{SeismiQuery}
主页:\url{http://ds.iris.edu/SeismiQuery/breq_fast.phtml}

\begin{itemize}
\item 适用范围:连续波形数据
\item 特色及优点:
    \begin{itemize}
    \item 网页工具
    \item 可以生成 \texttt{BREQ\_FAST} 所需的文件
    \end{itemize}
\end{itemize}
