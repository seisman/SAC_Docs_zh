\SACCMD{listhdr}
\label{cmd:listhdr}

\SACTitle{概要}
列出指定的头段变量的值

\SACTitle{语法}
\begin{SACSTX}
L!IST!H!DR! [D!EFAULT!|P!ICKS!|SP!ECIAL!] [FILES ALL|NONE|LIST]
    [COLUMNS 1|2] [INCLUSIVE ON|OFF] [hdrlist]
\end{SACSTX}

\SACTitle{输入}
\begin{description}
\item [DEFAULT] 使用默认的头段变量列表。列出所有已定义的头段变量
\item [PICKS] 使用picks头段列表。列出与到时拾取相关的头段变量
\item [SPECIAL] 使用用户自定义的特殊头段变量列表
\item [FILES ALL] 列出内存中所有文件的头段
\item [FILES NONE] 不列出头段,为将来的命令设置默认值
\item [FILES list] 列出部分文件的头段,list是要列出的文件的文件号
\item [COLUMNS 1|2] 输出格式为每行一/两列
\item [INCLUSIVE ON|OFF] ON表示列出未定义的头段变量的值,OFF则不列出
\item [hdrlist] 指定头段变量列表
\end{description}

\SACTitle{缺省值}
\begin{SACDFT}
listhdr default files all columns 1 inclusive off
\end{SACDFT}

\SACTitle{说明}
用户可以自定义要列出的项或使用两个标准列表中的某个,DEFAULT列表包含全部头段变量,
PICKS列表包含直接或间接用于定义到时拾取的头段变量,这个列表包含B、E、O、A、Tn、
KZTIME、KZDATE。用户可以随时定义一个特殊列表而且可以通过SPECIAL选项再次使用该列表。
该命令的输出包含头段变量名、一个等于号以及头段变量的当前值。某些文件的某些头段
变量可能未定义。SAC对这些未定义的头段变量,用一个特别的标志以标记它们,这个特殊的
标识是``undefined''。

\SACTitle{示例}
获取picks列表,输出为两列显示:
\begin{SACCode}
SAC> lh picks column 2
\end{SACCode}

获得第三、四个文件的默认头段列表:
\begin{SACCode}
SAC> lh files 3 4
\end{SACCode}

列出文件开始和结束时间:
\begin{SACCode}
SAC> lh b e
\end{SACCode}

定义一个包含台站参数的特殊列表:
\begin{SACCode}
SAC> lh kstnm stla stlo stel stdp
\end{SACCode}

稍后再次使用上面的特殊列表:
\begin{SACCode}
SAC> lh special
\end{SACCode}

只是设置默认两列输出:
\begin{SACCode}
SAC> lh columns 2 files none
\end{SACCode}
