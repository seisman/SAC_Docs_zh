\section{Matlab脚本中的SAC I/O}
SAC官方自带了几个可以读写SAC文件的Matlab脚本,位于 \verb|${SACHOME}/utils| 目录下。

\subsection{readsac}
\verb+readsac.m+ 中定义了函数 \verb|readsac|,可以用于读取SAC文件,其参数
为要读取的SAC文件名。文件名可以是:
\begin{enumerate}
\item 单个文件名
\item 多个文件名组成的字符串数组
\item 含通配符的字符串或字符串数组
\item 空字符串,表示读取当前目录下的全部SAC文件
\end{enumerate}

\verb|readsac| 函数有三种用法。第一种用法:
\begin{minted}{matlab}
% 读取SAC文件,并保存到结构体S中
>> S = readsac('seis.SAC');
% 通过S.varname形式引用结构体成员的值
>> S.NPTS
ans = 1000
>> S.DELTA
ans = 0.0100
% 数据保存到 S.DATA1 数组中
>> S.DATA1(10)
ans = -0.0934
\end{minted}

第二种用法:
\begin{minted}{matlab}
% 读取SAC文件,时间和波形振幅分别保存到数组X和Y中
>> [X, Y] = readsac('seis.SAC');
% 若发震时刻O未定义,则X数组中保存相对于B值的时间
% 若发震时刻O有定义,则X数组中保存相对于O值的时间
>> X(1)
ans = 50.8900
\end{minted}

第三种用法,有三个返回值:
\begin{minted}{matlab}
% 用于读取不等间隔数据或SAC谱数据
>> [X, Y1, Y2] = readsac('seis.SAC');
% 用于读取XYZ类型的SAC文件
>> [X, Y, Z] = readsac("seis.SAC.xyz");
\end{minted}

\subsection{getsacdata}
在用 \verb|readsac| 读取SAC文件时,可以直接读入到结构体S中,也可以读入到
多个数组中。读入到数组中便于matlab处理,但是却丢失了SAC头段中的很多信息;
读入到结构体S中,有时却需要用数组做处理,这就可以使用 \verb|getsacdata|
函数,可以从结构体S中提取出自变量和因变量数组:
\begin{minted}{matlab}
% 从文件中读入结构体S
>> S = readsac('seis.SAC');
% 从结构体S中提取自变量和因变量
>> [X, Y] = getsacdata(S);
\end{minted}

\subsection{writesac}
\verb|writesac.m| 中定义了函数 \verb|writesac|,用于写SAC格式的文件,其
输入是结构体S,返回值是状态码:
\begin{minted}{matlab}
% 读入SAC数据
>> S = readsac('seis.SAC');
% 做一些数据处理
% ...
% 修改结构体S中的文件名
>> S.FILENAME = 'new.SAC';
% 写入到新文件中
>> status = writesac(S);
\end{minted}

\subsection{其他}
除了SAC官方提供的几个Matlab脚本之外,还有其他人也提供了一些Matlab脚本:
\begin{itemize}
\item \url{http://geophysics.eas.gatech.edu/people/zpeng/Teaching/MatSAC.tar.gz}
\item \url{http://web.utah.edu/thorne/software.html}
\end{itemize}
