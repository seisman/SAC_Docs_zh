\SACCMD{qdp}
\label{cmd:qdp}

\SACTitle{概要}
控制低分辨率快速绘图选项

\SACTitle{语法}
\begin{SACSTX}
QDP [ON|OFF|n] [TERM ON|OFF|n] [SGF ON|OFF|n]
\end{SACSTX}

\SACTitle{输入}
\begin{description}
\item [ON|OFF] 打开/关闭终端和SGF设备的QDP选项
\item [n] 打开终端和SGF设备的QDP选项,并设定要绘制的点数为n
\item [TERM ON|OFF] 打开/关闭终端qdp绘图选项
\item [TERM n] 打开终端的QDP选项,并设定要绘制的数据点数为n
\item [SGF ON|OFF] 打开/关闭SGF设备的qdp绘图选项
\item [SGF n] 打开SGF设备的QDP选项,并设定绘制的数据点数为n
\end{description}

\SACTitle{缺省值}
\begin{SACDFT}
qdp term 5000 sgf 5000
\end{SACDFT}

\SACTitle{说明}
当文件中的数据点数很多的时候,绘制波形要花费很长的时间。``quick and dirty plot''
选项绘制数据文件的部分数据点的方式来加速绘图。

当打开QDP选项时,SAC用文件的数据点数除以QDP中的指定的数据点,由此计算
每个子区间所包含的数据点数。文件越大,每个子区间中数据点就越多,然后计算
并绘制每个子区间内最小和最大数据点,同时在绘图的右下角的矩形框中显示
减采样因子。实际显示的数据点数可能与该值所表示的数据点数有所偏差。

以目前计算机的性能而言,大型文件的绘制基本都是瞬间完成的,所以一般都
设置关闭此选项。

\SACTitle{示例}
假设文件FILE1有20000个数据点,文件FILE2有40000个数据点,如果你输入:
\begin{SACCode}
sac> r file1 file2
sac> p
\end{SACCode}
那么两张图都将包含5000个点。对第一个文件每4个点取一个点用于绘图,第二个
文件每8个点取一个点绘图。

如果想要绘制全部数据点,则需要关闭QDP选项:
\begin{SACCode}
SAC> qdp off
SAC> p
\end{SACCode}
