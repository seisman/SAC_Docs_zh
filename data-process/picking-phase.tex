\section{震相拾取}
\label{sec:phase-picking}

相关命令:\nameref{cmd:plotpk}

震相拾取是SAC中最常用的一个功能。plotpk的用法很简单,但却是教程中最难解释的一部分。
当在SAC中键入plotpk(简写为ppk)后,即进入ppk模式。
此时会出现一个绘图窗口,且焦点位于绘图窗口上,SAC程序进入了ppk模式。
要退出ppk模式,按“q” 。
注意:不是把光标移动到终端中去输入“q”,而是直接按“q”键即可,也不需要回车。
另外,在ppk模式,按任何键都不会显示在终端中。

\begin{SACCode}
SAC> fg seis
SAC> ppk        // 进入ppk模式
// 光标保持在绘图窗口上,按“q”就退出ppk模式
SAC> wh         // 覆盖原文件
SAC> q          // 退出SAC
\end{SACCode}


\begin{table}[H]
\centering
\small
\ttfamily
\caption{ppk模式命令一览表}
\label{table:plotpk-commands}
\begin{tabular}{cll}
	\toprule
    命令	&	含义	                                &   说明    \\
	\midrule
    a	    &	定义事件初至a                           &   1,7	    \\
    b	    &	如果有,则显示上一张绘图	            &           \\
    c	    &	计算事件的初至和结束                    &   1,4,7	\\
    d	    &	设置震相方向为DOWN	                    &           \\
    e     	&	设置震相起始为EMERGENT(急始)	        &           \\
    f	    &	定义事件结束f                           &  1,2,3,7	\\
    g	    &	以HYPO格式将拾取显示到终端              &   4   	\\
    h   	&	将拾取写成HYPO格式                      &   3,4 	\\
    i	    &	设置震相起始为IMPULSIVE	                &           \\
    j	    &	设置噪声水平                            &   2,6,8	\\
    k       &   即kill,退出ppk模式                     &           \\
    l	    &	显示光标当前位置                        &   2,4	    \\
    m	    &	计算最大振幅波形                        &   2,3,5	\\
    n	    &	显示下一绘图	                        &           \\
    o	    &	显示前一个绘图窗,最多可以保存5个绘图窗	&           \\
    p	    &	定义P波到时                             &   1,2,3,7	\\
    q	    &	即quit,结束ppk模式	                            &           \\
    s	    &	定义S波到时                             &   1,2,3,7 \\
    t	    &	用户自定义到时tn,输入t之后需要输入0到9中的任一数	&   1,2,7\\
    u	    &	设置震相方向为UP	                    &           \\
    v	    &	定义一个Wood-Anderson波形               &   2,5 	\\
    w	    &	定义一个通用波形                        &   2,5 	\\
    x	    &	使用一个新的x轴时间窗,简单说就是放大。 &           \\
    z	    &	设置参考水平                            &   2,6,8	\\
    \textbackslash	    &	删除当前全部拾取的定义。当一个文件中包含多个事件时有用。&	\\
    +	    &	设置震相方向为PLUS	                    &           \\
    -	    &	设置震相方向为MINUS	                    &           \\
    \lstinline[showspaces]{ }   &	设置震相方向为NEUTRAL	                &           \\
    n	    &	设置震相质量为n,n取0-4	                &           \\
	\bottomrule
\end{tabular}
\end{table}

移动光标到绘图窗口内部,光标会以交叉瞄准线的形式出现。在``\nameref{sec:sac-install}''
一节中提到的SAC全局变量~\verb+SAC_PPK_USE_CROSSHAIRS+~可以控制该光标的形态。当此全局
变量为0时,光标形态为
``\tikz[scale=0.3]{
    \draw[very thick] (0,0) -- (1,0);
    \draw[very thick] (0.5,-0.5) -- (0.5,0.5);
}'';当此全局变量为1时,会在0的基础上再加上水平和垂直线。个人推荐设置其值为1。

此时所有的键入都会解释成``ppk模式''下的命令,表~\ref{table:plotpk-commands}~列出了
``ppk模式''支持的所有命令。

不同的命令效果可能不同,有些会在绘图窗口显示信息,有些会将信息写入头段变量,
下面对表~\ref{table:plotpk-commands}~中的说明进行一个说明:
\begin{description}
    \item [1] 会将信息写入头段变量
    \item [2] 写入字符型震相拾取文件(若已打开)
    \item [3] 写入HYPO格式震相拾取文件(若已打开)
    \item [4] 在绘图窗口中显示信息
    \item [5] 窗口显示包含波形的矩形
    \item [6] 在指定的水平处放置水平光标
    \item [7] 绘图窗口显示含有到时标识的垂直线
    \item [8] 绘图窗口显示含有标识的水平线
\end{description}

``ppk模式''的命令几乎都是由单个字符组成的(唯一的例外是命令``t'',由字符``t''和0-9的
整数构成),每个命令都有相应的功能,同时也会产生一定的效果。经常使用的包括
``b''、``l''、``n''、``o''、``q''、``t''和``x''。

关于``x'',首先需要将光标移动到窗口中某位置,键入``x'',再向右移动至某位置,
再次键入``x'',两次键入确定了一个时间窗,绘图窗口中将只显示该时间窗内的波形,
即X轴的放大,常用于查看波形的细节。可以不断重复此步骤,进行多次放大。

对于101.5之后的版本,可以不使用``x''进行放大,直接在绘图窗口中某位置按下鼠标左键,并拖动至另一位置再松开鼠标左键,则该时间窗内的波形会被放大。这种方法相对于``x''来说更加高效。

放大之后的另一个问题就是缩小,这可以通过``o''来实现,``o''最多可以回退5次绘图历史。

当只有一个波形时:
\begin{SACCode}
SAC> fg seis
SAC> ppk
// 键入"t"和"0"标记到时
SAC> wh         // 保存头段
\end{SACCode}

当有多个波形时,一般每次只显示若干个:
\begin{SACCode}
SAC> dg sub teleseis ntkl.z nykl.z onkl.z sdkl.z
SAC> ppk p 1    // 每次显示1个波形
// 键入"n"显示下一个波形,键入"b"显示上一个波形
// 键入"t"和"0"标记到时
SAC> wh
SAC> q
\end{SACCode}

当有多个波形时,每次显示三分量,且希望一次性标记三个分量的到时:
\begin{SACCode}
SAC> dg sub teleseis ntkl.[nez] nykl.[nez] onkl.[nez] sdkl.[nez]
SAC> ppk p 3 r m
// 键入"t0"标记到时,此时键入一次"t0"可以标记当前显示的全部波形
SAC> wh
SAC> q
\end{SACCode}
