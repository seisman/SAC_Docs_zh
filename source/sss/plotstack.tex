\SACCMD{plotstack}
\label{sss:plotstack}

\SACTitle{概要}
绘制叠加文件列表中的文件

\SACTitle{语法}
\begin{SACSTX}
P!LOT!S!TACK! [S!UM! ON|OFF] [P!ERPLOT! ON|OFF|n] [W!EIGHT! ON|OFF] [P!OLARITY! ON|OFF]
\end{SACSTX}

\SACTitle{输入}
\begin{description}
\item [SUM ON|OFF] 若打开该选项,则首先绘制叠加后的波形再绘制叠加文件列表中的文件;若关闭该选项,则只回执叠加文件列表中的文件
\item [PERPLOT ON|OFF] 若打开该选择,则每次只绘制固定数目的文件;若关闭该选项,则一次绘制叠加列表中的全部文件
\item [PERPLOT n] 打开PERPLOT选项,并设置每次绘制n个文件
\item [WEIGHT ON|OFF] 打开/关闭文件权重选项
\item [POLARITY ON|OFF] 打开/关闭文件极性选项
\end{description}

\SACTitle{缺省值}
\begin{SACDFT}
plotstack sum on perplot off weight on polarity on
\end{SACDFT}

\SACTitle{说明}
该命令绘制叠加文件列表中的文件,所有的文件首先根据静/动延迟进行时移,
该命令可以控制绘制文件时是否考虑权重因子和极性。

该命令的用法与 \nameref{cmd:plot1} 类似,在每个子图的左上角会显示文件名
以及其他非默认的属性值。
