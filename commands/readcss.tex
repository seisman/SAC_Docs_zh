\SACCMD{readcss}
\label{cmd:readcss}

\SACTitle{概要}
从磁盘读取CSS文件到内存。

注意:READCSS命令读取CSS 3.0或2.0的文本文件。
3.0版本支持如下数据库表:

wfdisc、wftag、origin、arrival、assoc、sitechan、site、affiliation、origerr、origin、event、sensor、instrument、gregion、
stassoc和remark sacdata。

\SACTitle{语法}
\begin{SACSTX}
READCSS [BINARY|ASCII] [MAXMEM v] [MORE] [TRUST ON|OFF]
[VERBOSE ON|OFF] [SHIFT ON|OFF] [SCALE ON|OFF] [COMMIT|ROLLBACK|RECALLTRACE] 
[DIR name] wfdisclist [filelist] [css options] [filelist]
\end{SACSTX}

css选项至少是下列之一:

\begin{SACSTX}
STATION station
CHANNEL channel
BANDWIDTH band code
ORIENTATION orientation code
\end{SACSTX}

这些选项使命令从符合相关记录内容的wfdisc文件中进行选择。

\SACTitle{输入}
\begin{description}
\item [ASCII] (默认)读取普通的文本文件
\item [BINARY] 读取二进制CSS文件。见WRITECSS了解更多信息
\item [TRUST ON|OFF] 这个选项用于判决从SAC格式转为CSS格式的模棱两可。
    当转化这数据时,比较事件ID可能意味着这些文件有用于可识别的事件信息,
    或者它们可能是合并了两个非常不同格式的数据的人工制品。
    当设置TRUST为ON时,相较于TRUSRT为OFF,SAC更可能接受比较事件ID为事件识别信息,
    依赖于和现在在内存中的数据文件相关的READ命令的历史。%翻译有疑问
\item [MAXMEM]用于设定读取数据时所需的物理内存的最大百分比。
    当达到设定的限制时,即使已经读取了其他数据库表,也不会再读取更多的波形数据。
    MAXMEN的默认值是0.3。
\item [MORE] 见READ命令。
\item [VERBOSE ON|OFF] 如果VERBOSE是ON,SAC展示关于正在读取的波形数据的延伸信息,
    并且打印出包含的CSS数据库表的概要。
    SAC也为向SAC内部格式的转化的显示一个进度指示器。
\item [SHIFT ON|OFF] 如果SHIFT是ON,发震时刻将被设置为0,其他相关时间头段变量将会做相应修改。
    一些震中距相关头段变量也会受影响。
    SHIFT ON是默认值。
\item [SCALE ON|OFF] SCALE选项默认是OFF。
    当SCALE选项是OFF时,SAC认为从.w文件读取的数据的单位是counts。
    \footnote{解释counts}
    当SCALE选项是ON时,SAC会在单位counts的基础上乘以一个比例因子。
    这个比例因子由.wfdisc文件里的校准(CALIB)给出。
    这个SCALE选项随着READ SEGY、READGSE和READSUDS的SCALE选项变化。
    如果SCALE选项是OFF,从这个文件来的CALIB值会被存入SAC的SCALE头段变量中。
    如果SCALE选项是ON,SAC的SCALE头段区域(header field)将被设置为1.0。
    SCALE是一个处理仪器响应的粗陋方法。
    更好的方式是使用TRANSFER命令。
    建议READ时,让SCALE为OFF,并且用TRANSFER命令。
    只有在TRANSFER命令需要的响应信息无法获得时,SCALE才应该为ON。
\item [COMMIT]如果MORE选项进行了设置,COMMIT选项会把头段和波形记录到SAC的内存中,
    在读取更多文件前从RAM删除之前任何版本的头段和波形。
    COMMIT是默认值。
\item [ROLLBACK] 如果设置了MORE选项,ROLLBACK选项在读取更多文件前恢复到上次记录到的头段和波形的版本。
\item [RECALLTRACE]如果设置了MORE选项,RECALLTRACE选项:

    波形回滚到上一个版本,

    这些波形紧密相关的头段变量回滚到上一个版本,

    记录那些和这些波形不紧密相关的头段变量(可用来获得哪些变量被记录了,哪些则是回滚到上一个版本)。%这一段和前面两段是前面加点的列表,该用哪个列表宏包?

    注意:如果MORE选项没有设置,COMMIT、ROLLBACK和RECALLTRACE选项就不发挥作用。%这个『注意』这样排版好吗?原文看上去像一个\item
\item [DIR name] wfdisc文件所在的路径
    \footnote{关于dir选项,有一个很大的陷阱,详见``\nameref{sec:read-dir}''。}
\item [filelist] 包含在先前的specified wfdisc(s)文件的名字列表。%这个specified wfdisc(s)的specified怎么翻译?
    这些文件会被搜索,首先搜索的是DIR选项中设定的路径,然后使用wfdisc文件中设定的路径。
    如果没有filelist,会将specified wfdisc(s)文件内指定的文件读入内存。
\item[STATION station]station是一个6个或更少字符构成的字符串。
    .wfdisc文件中那些KSTNM匹配的台站的行会被选中。
    station可以包含通配符:*和?。
\item[CHANNEL channel]channel是一个8个或更少字符构成的字符串。
    .wfdisc文件中那些channel匹配的台站的行会被选中。
    channel可以包含通配符:*和?。
\item [BANDWIDTH type] 一个单字符的编码。
    常见的值有E:Extremely Short Period,S:Short Period,H:High Broad Band,B:Broad Band,M:Mid Period
    L:Long Period,V:Very Long Period,U:Ultra Long Period,R:Extremely Long Period。
    选择那些``chan'' field有个打头字符为``s''的文件%不知何意
    字符``*'',选择所有文件。
\item [ORIENTATION type]值常常是
    ``Z N E''表示竖直、北和东,
    ``A B C''表示Triaxial along edges of cube standing on corner%不知道怎么翻译
    ``1 2 3''表示正交但非标准的坐标。
    选择那些``chan'' field有个相符的结尾字符的文件%不知何意
    字符``*'',选择所有文件。
\end{description}
\SACTitle{震级}
MB

MS

ML

DEF

确定哪一个震级的值放到SAC的magnitude头段变量。
MB是体波震级。MS是面波震级。ML是地方震震级。
DEF指default:如果Ms存在而且大于或等于6.6,则最优先用Ms。否则,如果Mb存在,用Mb。如果Mb不存在,而Ms存在,用Ms。其他情况用Ml。%MB=Mb,MS=Ms,ML=Ml
\SACTitle{默认值}
\begin{SACSTX}
READCSS * ASCII MAXMEM 0.3 VERBOSE OFF COMMIT STATION * BAND * CHAN * ORIENT *
\end{SACSTX}
\SACTitle{说明}%说明部分已经“写好”,注意不是read旧的部分,为了排版从前面移动过来的!
2.8版本只支持wfdisc、arrival和origin。
READCSS之前的版本要求origin文件只有一行与波形有关%此句存疑
现在的版本可以用来自一个wftag文件或者wfdisc文件中an evid column的信息来提取正确的源(position 284 - 291)。%此句存疑
如果没有可用的信息,READCSS会默认以前的方式用origin文件的第一行。
现在,用READCSS读取数据不再会有数据丢失。
尽管现有的SAC命令只能获取一部分CSS数据,从CSS文本文件读取的存放在内存中的每个数据都可以用命令WRITECSS写入磁盘。

READCSS现在(可以)读取一个非标准的数据库表,名为sacdata(由WRITECSS命令输出的)。%此句存疑
比如,你现在可以用WRITECSS把频率域数据写入磁盘,然后用READCSS读取

READCSS现在有一个二进制选项,以便读取WRITECSS生成的二进制的CSS文件。
在二进制模式中,css选项没有作用。
也就是说,指定文件的全部内容都会读取。

READCSS支持如下的二进制数据格式:在大端序机器(比如Sun)上的t8、t4、f8、f4、s4、s3、s2、i4、i2、g2、e1和ri (real-imag)。

在小端序机器(比如DEC和PC)上: f8、f4、t8、t4、i4、i2、s4、s2和g2。

%以下不是,我为了好写,是在read.tex的基础上改内容来写的
\SACTitle{示例}%暂无示例
\begin{SACCode}
SAC> read dir event1 f01 f02
\end{SACCode}
\SACTitle{相关命令}
\nameref{cmd:read}、\nameref{cmd:pickprefs}、\nameref{cmd:pickauthor}、\nameref{cmd:PICKPHASE}、COMMIT, ROLLBACK, RECALLTRACE
%此处命令还没有!
