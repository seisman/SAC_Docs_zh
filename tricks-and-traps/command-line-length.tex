\section{命令长度}
大多数程序在处理命令行的输入时,都会先定义一个固定长度的字符数组,将
命令行的输入读入到字符数组中,再进行进一步的解析。一般来说,这个字符
数组是足够大的,很少会担心命令过长的问题。

那么SAC所能允许的命令的最大长度是多少呢?看一下SAC源码可以知道答案是
1001。考虑到C语言中字符串需要以``\verb|\0|''结尾,所以实际上能够允许的
最大长度是1000。

1000这个长度似乎够了,因为大多数SAC命令的长度都不会超过20,而且用户也
没有那个心情去敲一个长度超过1000的命令。一个特殊的情况是命令中包含文件名。

假设在当前目录下有120个形如 \texttt{XX.XXX.BHZ} 的SAC文件,每个文件的
文件名长度为10字符。如果要将全部SAC文件读入到内存中,最简单的办法是
使用通配符:
\begin{SACCode}
SAC> r *.BHZ
\end{SACCode}
当然如果不觉得麻烦,完全可以把120个文件名一个一个敲到命令行里:
\begin{SACCode}
SAC> r XX.001.BHZ XX.002.BHZ ... XX.120.BHZ // 此处省略了一堆文件名
\end{SACCode}

这两种读入SAC文件的方式,看上去很相似,结果却是完全不同的。

前一种方式中,SAC获取的命令长度实际为7字符,远小于1000字符的长度上限,
然后SAC会在程序内部将通配符展开为120个文件的文件列表。

后一种方式中,SAC获取的命令长度超过1200字符,但只有前1000字符会真正被
SAC真正接收并处理,这将导致仅有不到100个SAC文件会被读入,而SAC不会对此
给出任何警告。这是一件非常危险的事情。

这两种方式比较起来,不管是从简便性还是安全性角度来看,都应该选择
通配符这种方式。

在脚本中使用通配符,有一点需要注意。以Perl脚本为例,下面的Perl脚本
调用了sac,并读取全部文件,然后做了简单的数据处理,最后保存退出。

\begin{minted}{perl}
#!/usr/bin/evn perl
use strict;
use warnings;

open(SAC, "| sac") or die "Error in opening sac\n";
print SAC "r *.BHZ\n";
print SAC "rmean; rtr; taper\n";
print SAC "bp c 1 3 n 4 p 2\n";
print SAC "w over\n";
print SAC "q\n";
close(SAC);
\end{minted}

上面的脚本是可以正常工作的,但是如果改成如下看上去很像的脚本,则会出问题。
\begin{minted}{perl}
#!/usr/bin/evn perl
use strict;
use warnings;

my @files = glob "*.BHZ";

open(SAC, "| sac") or die "Error in opening sac\n";
print SAC "r @files\n";
print SAC "rmean; rtr; taper\n";
print SAC "bp c 1 3 n 4 p 2\n";
print SAC "w over\n";
print SAC "q\n";
close(SAC);
\end{minted}

两种都使用了通配符或通配函数,前者直接将 \texttt{*.BHZ} 传递给SAC,
而后者却将通配(\texttt{glob})后的文件列表传给SAC,所以后者会出现问题。
在写脚本的时候应避免后一种写法。

通配符不是万金油,很多时候无法使用通配符来通配一堆没有规律的文件。

例如,当前目录下有200个形如 \texttt{XX.XXX.BHZ} 的SAC文件,其中140个
有清晰的P波,P波的到时被标记到 \texttt{T0} 中。现在想要读取这140个有
清晰P波的数据,这个时候显然通配符是没法用了。

错误的写法如下:
\begin{minted}{perl}
#!/usr/bin/evn perl
use strict;
use warnings;

# 获取全部T0有定义的文件名列表
my @files = ();
open(SACLST, "saclst t0 f *.BHZ |");
foreach (<SACLST>) {
    my ($fname, $t0) = split ' ', $_;
    push @files, $fname if $t0 > 0;
}
close(SACLST);

# 调用SAC进行数据处理
open(SAC, "| sac") or die "Error in opening sac\n";
print SAC "r @files\n";
print SAC "rmean; rtr; taper\n";
print SAC "bp c 1 3 n 4 p 2\n";
print SAC "w over\n";
print SAC "q\n";
close(SAC);
\end{minted}

调用SAC进行数据处理的正确写法:
\begin{minted}{perl}
open(SAC, "| sac") or die "Error in opening sac\n";
foreach (@files) {
    print SAC "r more $_\n";
}
print SAC "rmean; rtr; taper\n";
print SAC "bp c 1 3 n 4 p 2\n";
print SAC "w over\n";
print SAC "q\n";
close(SAC);
\end{minted}

效率稍低的正确写法:
\begin{minted}{perl}
open(SAC, "| sac") or die "Error in opening sac\n";
foreach (@files) {
    print SAC "r $_\n";
    print SAC "rmean; rtr; taper\n";
    print SAC "bp c 1 3 n 4 p 2\n";
    print SAC "w over\n";
}
print SAC "q\n";
close(SAC);
\end{minted}
