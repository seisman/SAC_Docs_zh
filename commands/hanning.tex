\SACCMD{hanning}
\label{cmd:hanning}

\SACTitle{概要}
对数据文件加hanning窗

\SACTitle{语法}
\begin{SACSTX}
HAN!NING!
\end{SACSTX}

\SACTitle{说明}
hanning窗是一种对数据点的递归平滑算法。对于每个非端点数据点($j\in[2,N-1]$)而言,有
\[
    Y(j)=0.25Y(j-1)+0.50Y(j)+0.25Y(j+1)
\]
更新$Y(2)$时使用了$Y(1)$、$Y(2)$、$Y(3)$的旧值,而更新$Y(3)$时则使用了$Y(2)$的
新值以及$Y(3)$、$Y(4)$的旧值,这也是其称为\textbf{递归}平滑算法的原因。

两个端点的数据,需要做特殊处理,令$Y(1)$等于$Y(2)$的新值,$Y(N)$等于$Y(N-1)$的新值。

\SACTitle{头段变量}
depmin、depmax、depmen
