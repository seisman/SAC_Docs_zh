\SACCMD{filterdesign}
\label{cmd:filterdesign}

\SACTitle{概要}
产生一个滤波器的数字和模拟特性的图形显示,包括:振幅、相位、脉冲响应和群延迟。

\SACTitle{语法}
\begin{SACSTX}
F!ILTER!D!ESIGN! [FILE [prefix]] [filteroptions] [delta]
\end{SACSTX}

\SACTitle{输入}
\begin{description}
\item [FILE prefix] 生成的三个SAC文件的前缀
\item [filteroptions] 与SAC中其他的滤波命令相同,包括滤波器类型
\item [delta] 数据的采样间隔。
\end{description}

\SACTitle{缺省值}
!delta! 缺省值为 \SI{0.025}{\s},其他参数无缺省值

\SACTitle{说明}
!filterdesign! 命令调用了函数 !xapiir!(一个递归数字滤波器包)。
!xapiir! 通过模拟滤波器原型的双线性变换实现标准递归数字滤波器的设计。
这些原型滤波器由零点和极点给定,然后使用模拟谱变换,变换到高通、带通和带
阻滤波器。

!filterdesign! 用实线显示数字滤波器响应,用虚线显示模拟滤波器响应。
在彩色显示器上,数字曲线是蓝色的而模拟曲线是琥珀色的。

生成的三个SAC文件分别为 !prefix.spec!、!prefix.grd!、
!prefix.imp!。其中 !prefix.spec! 为该命令产生的振幅相位信息,
为振幅-相位格式谱文件。!prefix.grd! 为该命令产生的群延迟信息,
是时间序列文件。需要注意的是,尽管这个文件是时间序列文件,但是实际上群
延迟是频率的函数。用户要记住,虽然绘图时横轴单位是秒,实际的单位却是
\si{\Hz}。!prefix.imp! 是时间序列文件,包含脉冲响应信息。

在这三个SAC文件中,用户自定义头段变量 !USERn!、!KUSERn! 设置如下:
\begin{itemize}
\item user0:表示pass code。1代表LP;2代表HP;3代表BP;4代表BR;
\item user1:type code。1代表BU,2代表BE,3代表C1,4代表C2;
\item user2:number of poles
\item user3:number of passes
\item user4:tranbw
\item user5:attenuation
\item user6:delta
\item user7:first corner
\item user8:second corner if present, or -12345 if not
\item kuser0:pass (lowpass, highpass, bandpass, or bandrej)
\item kuser1:type (Butter, Bessel, C1, or C2 )
\end{itemize}

\SACTitle{示例}
下面的例子展示了如何使用 !filterdesign! 命令产生一个高通,拐角
频率为 \SI{2}{\Hz},六极、双通滤波器的数字和模拟响应曲线,数据采样间隔为
\SI{0.025}{\s}:
\begin{SACCode}
SAC> fd hp c 2 n 6 p 2 delta .025
\end{SACCode}
