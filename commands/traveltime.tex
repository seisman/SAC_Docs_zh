\SACCMD{traveltime}
\label{cmd:traveltime}

\SACTitle{概要}
根据预定义的速度模型计算指定震相的走时

\SACTitle{语法}
\begin{SACSTX}
T!RAVEL!TIME [MODEL string] [PICKS number] [PHASE phase list]
    [VERBOSE|QUIET] [M|KM]
\end{SACSTX}

\SACTitle{输入}
\begin{description}
\item [MODEL] iasp91或ak135,缺省为iasp91
\item [PICKS] 如果number值位于0到9,则第一个震相的到时将储存在头段Tn中。如果命令行选项中没有包含PICKS,VERBOSE就会打开并显示震相到时而不写入头段。
\item [PHASE] 要pick或显示的震相列表,如果有PICKS n选项,则则震相到时以及其标签将写入到头段Tn和KTn中。
\item [VERBOSE|QUIET] 若使用VERBOSE,则震相走时将以相对发震时刻(O)和文件起始时间(B)两种方式显示;若使用QUIET,则不在屏幕上显示到时,若二者都没有显示,则显示traveltime命令使用的深度
\item [M|KM] 头段变量evdp的单位为 \si{\m} 或者 \si{\km}
\end{description}

\SACTitle{缺省值}
\begin{SACDFT}
traveltime MODEL iasp91 KM PHASE P S Pn Pg Sn Sg
\end{SACDFT}

\SACTitle{说明}
该命令使用iaspei-tau
\footnote{\url{https://seiscode.iris.washington.edu/projects/iaspei-tau}}
程序计算走时,要求内存中的波形文件的事件和台站位置以及发震时刻必须定义。

内存中所有文件拾取的震相的到时存储在头段变量Tn中,其中n的范围为0到9。计算的走时是相对于发震时刻(O)的,但存储在Tn中的时间是相对于文件起始时间(B)。

走时表根据地震深度及震中距获得走时,震中距(gcarc)使用球三角几何,根据事件和台站经纬度计算。

由于历史原因,事件深度evdp以前的单位为 \si{\m},rdseed产生的SAC波形数据单位为 \si{\m}。
在新版的SAC中,默认的evdp单位为 \si{\km},但由于很多波形数据evdp仍然为 \si{\m},这里引入这个
命令选项以指定深度单位。

在SAC2000中,命令traveltime位于子程序SSS中,当前版本(v101.5)没有SAC2000版本的全部功能,
但可以不用进入SSS而直接运行。

\SACTitle{示例}
区域事件,使用默认震相:
\begin{SACCode}
SAC> fg seismo
SAC> traveltime
traveltime: depth: 15.000000
traveltime: error finding phase P
traveltime: error finding phase S
traveltime: setting phase Pn       at 10.464321 s [ t = 51.894321 s ]
traveltime: setting phase Pg       at 22.904724 s [ t = 64.334724 s ]
traveltime: setting phase Sn       at 50.047722 s [ t = 91.477722 s ]
traveltime: setting phase Sg      at 66.414337 s [ t = 107.844337 s ]
\end{SACCode}

对于区域事件,初至波为Pn或Pg,因而这里没有P波到达,对上面的例子,拾取不会写入头段中。
\begin{SACCode}
SAC> fg seismo
SAC> traveltime picks 0 phase Pn Pg Sn Sg
traveltime: depth: 15.000000
SAC> lh AMARKER T0MARKER T1MARKER T2MARKER T3MARKER
AMARKER = 10.464
T0MARKER = 10.464           (Pn)
T1MARKER = 22.905           (Pg)
T2MARKER = 50.048           (Sn)
T3MARKER = 66.414           (Sg)
SAC> write seismo-picks.z
\end{SACCode}
可以看到已经将A定义为Pn的到时,文件seismo-picks.z将有T0到T3四个头段,
命令PLOT1可以看到有各震相在相应到时处有标签名。
注意尽管VERBOSE没有开启,深度(单位 \si{\km})还是会被打印出来。
这是为了保证用户对EVDP的单位有正确的了解,可以通过QUIET选项取消深度的显示。

下面例子给出的波形文件是由rdseed v5.0产生的,震源深度evdp的单位为 \si{\m}:
\begin{SACCode}
SAC> r 2008.052.14.16.03.0000.XC.OR075.00.LHZ.M.SAC
SAC> lh evdp
evdp = 6.700000e+03
SAC> traveltime M picks 0
traveltime: depth: 6.700000 km
SAC> lh t0marker t1marker t2marker t3marker
t0marker = 61.48            (Pn)
t1marker = 76.413           (Pg)
t2marker = 109.66           (Sn)
t3marker = 132.11           (Sg)
SAC> ch evdp (0.001 * :1,evdp:)
SAC> setbb station :1,KSTNM:
SAC> write %station%.z
\end{SACCode}
保存的文件OR075.z单位为 \si{\km},并且各个震相Pn、Pg、Sn及Sg都有注释。
震相名是大小敏感的,具体参见iaspei-tau的文档。
