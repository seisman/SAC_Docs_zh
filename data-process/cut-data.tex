\section{数据截窗}
相关命令:\nameref{cmd:cut}

数据申请时一般会选择尽可能长的时间窗,而实际进行数据处理和分析时可能只
需要其中的一小段数据,这就需要对数据进行时间窗截取。

SAC中的 \texttt{cut} 命令可以实现数据截取,需要注意的是,该命令是
``参数设定类'' 命令,即需要先执行 \texttt{cut} 命令再执行
\texttt{read} 命令。

\subsection{pdw}
\label{subsec:pdw}
使用 \texttt{cut} 命令对数据进行截取时需要定义数据时间窗。除了截取数据
之外,其他一些命令也会需要定义时间窗,比如 \nameref{cmd:rms}、
\nameref{cmd:mtw}、\nameref{cmd:xlim} 等,这些命令都使用同样的方式定义
时间窗,在SAC中称为pdw,即partial data window。

pdw定义了一个开始时间和一个结束时间,其格式为 \texttt{ref offset ref offset}。
其中 \texttt{ref} 为参考时刻,可以取 \texttt{Z}、\texttt{B}、\texttt{E}、
\texttt{O}、\texttt{A}、\texttt{F}、\texttt{N} 和 \texttt{Tn}(n=0--9),
而 \texttt{offset} 为相对于参考时刻的时间偏移量。

参考时刻 \texttt{ref} 可以取如下值:
\begin{itemize}
\item \texttt{B}:磁盘文件起始值
\item \texttt{E}:磁盘文件结束值
\item \texttt{O}:事件开始时间
\item \texttt{A}:初动到时
\item \texttt{F}:信号结束时间
\item \texttt{Tn}:用户自定义时间标记,n=0--9
\item \texttt{Z}:参考时刻
\item \texttt{N}:将 \texttt{offset} 解释为数据点数而非时间偏移量,
    其仅可以用于结束值
\end{itemize}

若开始或结束的 \texttt{offset} 省略则认为其值为0。若开始 \texttt{ref}
省略则认为其为Z;若结束 \texttt{ref} 省略则认为其值与开始 \texttt{ref}
相同。

下面的例子中展示了一些常见的pdw及其含义:
\begin{SACCode}
 B E            // 文件开始到文件结束,即与cut off相同
 B 0 30         // 文件开始的30秒
 A -10 30       // 初动前10秒到初动后30秒
 B N 2048       // 文件最初的2048个点
 T0 -10 N 1000  // 从T0前10秒起的1000个点
 30.2 48        // 相对磁盘文件0点的30.2到48秒
\end{SACCode}

\subsection{cut}
\nameref{cmd:cut} 命令是``参数设定类''命令,因而需要先 \texttt{cut} 再
\texttt{read}:
\begin{SACCode}
SAC> cut t0 -5 5        // 截取t0前后各5秒,共计10秒的数据
SAC> r *.SAC            // 先cut再read
\end{SACCode}
