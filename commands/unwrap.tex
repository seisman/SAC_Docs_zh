\SACCMD{unwrap}
\label{cmd:unwrap}

\SACTitle{概要}
计算振幅和展开相位

\SACTitle{语法}
\begin{SACSTX}
UNWRAP [F!ILL! ON|OFF|n] [I!NTTHR! v] [P!VTHR! v]
\end{SACSTX}

\SACTitle{输入}
\begin{description}
\item [FILL ON|OFF] 打开/关闭补零选项
\item [FILL n] 打开补零选项并设置填充值为n
\item [INTTHR v] 改变积分阈值常量为v
\item [PVTHR v] 改变主值阈值常量为v
\end{description}

\SACTitle{缺省值}
\begin{SACDFT}
unwrap fill off intthr 1.5 pvthr 0.5
\end{SACDFT}

\SACTitle{说明}
这个命令将内存中的时间序列数据转换为谱数据,其包含了振幅谱和展开相位谱。
这个过程对于有光滑变化相位的数据起作用。数据在转换之前补零以使数据点数为2的幂数倍,
可以使用FILL选项指定更大的补零数。

这是Tribolet算法的一个具体实现。有两种方法用于估算在每个频率的展开相位。
其中一个是通过快速傅氏变换做相位偏导的数值积分,如果需要得到一个一致的估计,
则可将梯形积分的步长在每个频率上对分。可以使用INTTHR选项控制这个验算的阈值,
此值单位为弧度。减少INTTHR将改进相位计算结果,若该值太小,会导致解的发散。

算法中使用的第二个方法是先用反正切函数计算相位的主值。展开相位的计算方法是相位主值加上
$2\pi$的整数倍,直到相位的突变小于给定的阀值为止。可以使用PVTHR选项控制这个验算的阀值。
与上一个算法类似,减少这个阀值将改进相位估算的结果,但也增加了无解的可能性。

这两个阀值的初值通常经验地取为:
\[ \pi/4 < PVTHR < INTTHR < 2\pi \]

\SACTitle{头段变量}
b、e和delta分别改变为变换的起始频率、结束频率和采样频率。原始的b、e和delta被保存在为sb、se、sdelta,当进行反变换时将值带回。

\SACTitle{限制}
目前可以转换的数据最大长度为4096。
