\SACCMD{setmacro}
\label{cmd:setmacro}

\SACTitle{概要}
定义SAC宏文件的搜索路径

\SACTitle{语法}
\begin{SACSTX}
SETMACRO [MORE] directory [directory ...]
\end{SACSTX}

\SACTitle{输入}
\begin{description}
\item [directory] 放置SAC宏文件的目录,可以是相对路径或绝对路径
\item [MORE] 将 \texttt{directory} 加到已有的路径列表之后
\end{description}

\SACTitle{说明}
该命令让你能够定义一系列执行宏文件时搜索路径,最多可以定义100个。

当 \texttt{setmacro} 使用 \texttt{more} 选项时,指定的路径会追加到
已存在的路径列表的后面;若没有使用 \texttt{more} 选项,则已存在的列表
将被新列表取代。

当执行 \texttt{macro} 命令时,SAC会先搜索当前目录,若没有找到则搜索
\texttt{setmacro} 指定的目录,若依然没有找到则在全局宏目录中寻找。
