\SACCMD{ifft}
\label{cmd:ifft}

\SACTitle{概要}
对数据进行离散反傅立叶变换

\SACTitle{语法}
\begin{SACSTX}
IFFT
\end{SACSTX}

\SACTitle{说明}
数据文件必须是之前利用 \nameref{cmd:fft} 命令生成的谱文件,可以是
实部-虚部格式或振幅-相位格式。

该命令会从 \texttt{sb}、\texttt{sdelta}、\texttt{nsnpts} 中读取原始
数据在时间域的起始时间、采样周期和数据点数。频率域的起始频率、采样频率、
采样点数将被保存到 \texttt{SB}、\texttt{SDELTA}、\texttt{NSNPTS} 中。

\SACTitle{头段变量}
b、delta、npts、sb、sdelta、nsnpts

\SACTitle{限制}
目前IFFT所允许的最大数据点数为65536。
