\SACCMD{fft}
\label{cmd:fft}

\SACTitle{概要}
对数据做快速离散傅立叶变换

\SACTitle{语法}
\begin{SACSTX}
FFT [WO!MEAN!|W!MEAN!] [R!LIM!|A!MPH!]
\end{SACSTX}

\SACTitle{输入}
\begin{description}
\item [WOMEAN] 进行变换之前先去除均值
\item [WMEAN] 变换中不去除均值
\item [RLIM] 输出为实部-虚部格式
\item [AMPH] 输出为振幅-相位格式
\end{description}

\SACTitle{缺省值}
\begin{SACDFT}
fft wmean amph
\end{SACDFT}

\SACTitle{说明}
该命令对数据进行离散傅立叶变换,为了使用快速傅立叶变换算法,在进行变换之前,需要对
数据文件进行补零以保证数据点数为2的整数次幂,比如1000个点的时间序列文件会被补零至
1024个点,且头段变量npts也会被相应修改。

进行离散傅立叶变换之后,头段变量b为谱文件的起始频率,其值为0;delta为谱文件的采样频率,
取值为1/(delta*npts);e为谱文件的结束频率。

谱数据可以是振幅-相位格式或实部-虚部格式。头段变量IFTYPE会告诉你谱文件是哪种格式。

对于实序列而言,由于离散傅立叶变换后的结果具有“共轭对称性”,因而在使用plotsp绘制谱文件时
只显示一半的数据点数。

\SACTitle{示例}
\begin{SACCode}
SAC> fg seis
SAC> lh b e delta npts iftype

          b = 9.459999e+00
          e = 1.945000e+01
      delta = 1.000000e-02
       npts = 1000
     iftype = TIME SERIES FILE
SAC> fft
 DC level after DFT is -0.98547
SAC> lh b e delta npts iftype

          b = 0.000000e+00              // b值为0
          e = 5.000000e+01
      delta = 9.765625e-02              // delta=1/(1024*0.01)
       npts = 1024                      // 1000 -> 1024
     iftype = SPECTRAL FILE-AMPL/PHASE
SAC> lh sb sdelta nsnpts                // 保留原值

         sb = 9.459999e+00
     sdelta = 1.000000e-02
     nsnpts = 1000
\end{SACCode}

\SACTitle{头段变量}
变换过程中,B、E和DELTA分别被修改为起始频率、结束频率以及采样频率。
B、E、NPTS和DELTA的原值被保存在SB、SE、NSNPTS和SDELTA中,这些值在做反傅立叶变换时会用到。

\SACTitle{限制}
离散傅立叶变换所允许的最大数据点数为$2^{24}=16777216$个。
