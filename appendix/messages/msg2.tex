\pdfbookmark[1]{E2xxx}{E2xxx}
\SACTitle{2001 Command requires an even number of data files}
使用 \nameref{cmd:rotate} 命令进行数据对的旋转时,需要内存中有成对的数据文件(即偶数个),
若内存中数目不为偶数个,则报错。注意检查是否有数据漏读。

\SACTitle{2002 Following files are not an orthogonal pair}
出现该错误的原因是使用 \nameref{cmd:rotate} 旋转的两个分量不完全正交,
此时应注意检查两个分量的头段变量 \texttt{cmpinc} 和 \texttt{cmpaz}。
若两个头段变量未定义,则需要根据仪器的其他信息确定两个头段变量的值;
若两个头段变量有定义,但的确不正交,则无法进行分量旋转。

\SACTitle{2003 Following files are not both horizontals}
\nameref{cmd:rotate} 命令的 \texttt{TO} 选项只能用于将两个水平的分量旋转到
某个角度,出现该错误时应注意检查两个分量的头段变量 \texttt{cmpinc} 是否等于90度。

\SACTitle{2004 Insufficient header information for rotation}
\nameref{cmd:rotate} 命令的 \texttt{TO GCP} 选项要求头段变量中 \texttt{stla}、
\texttt{stlo}、\texttt{evla} 和 \texttt{evlo} 必须定义。该选项会读取一对分量中的
第一个文件中的头段
变量,并计算反方位角,而不是直接使用头段变量中已有的反方位角值。

\SACTitle{2008 Requested begin time is less than files begin time. Output truncated.}
\nameref{cmd:interpolate} 命令中指定的开始时间小于文件起始时间,此时输出会被截断。

\SACTitle{2111 Taper frequency limits are invalid. No taper applied.}
该警告出现在 \nameref{cmd:transfer} 命令中,出现该错误的原因是 \texttt{freqlimits}
选项的参数设置有误。四个频率应满足 \texttt{f1<f2<f3<f4}。

出现该警告时,\nameref{cmd:transfer} 会忽略 \texttt{freqlimits} 选项,即在去仪器
响应时,不使用taper函数,进而可能导致去仪器响应后的波形出现问题。