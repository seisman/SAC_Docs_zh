\section{在Mac下安装SAC}
\label{sec:sac-install-for-mac}
本节介绍如何在Mac OS X 如何安装SAC。

Mac下安装SAC,可以直接使用官方提供的二进制包,也可以手动编译源码包
\footnote{似乎在最新版本的 macOS 下手动编译的 SAC 无法使用,暂无解决办法}。
对于大多数用户而言,建议安装二进制包。下面会分别介绍两种安装方法。

\subsection{准备工作}
首先要安装Mac下的命令行工具。在终端执行如下命令:
\begin{minted}{console}
$ xcode-select --install
\end{minted}
即可。

此外,还需要安装X11图形界面相关工具,即 XQuartz。可以按照\href{https://support.apple.com/zh-cn/HT201341}{Apple 的官方支持}来解决。
对于 Homebrew 用户,可以使用
\begin{minted}{console}
$ brew cask install xquartz
\end{minted}
安装。

\subsection{安装二进制包}
直接将官方的二进制包解压并移动到安装目录即可:
\begin{minted}{console}
$ tar -xvf sac-101.6a-mac_x86_64.tar.gz
$ sudo mv sac /opt
\end{minted}

\subsection{编译源码}
按照如下命令即可正确编译源码。需要注意的是,由于SAC默认使用的editline库
在Mac下无法正常编译,因而执行 !configure! 时使用了
!--enable-readline! 选项使得SAC使用readline库而不是editline库。
\begin{minted}{console}
$ tar -xvf sac-101.6a_source.tar.gz
$ cd sac-101.6a
$ mkdir build
$ cd build
$ ../configure --prefix=/usr/local/sac --enable-readline
$ make
$ sudo make install
\end{minted}

\subsection{配置变量}
向 !~/.bash_profile! 中加入如下语句以配置环境变量和SAC全局变量:
\begin{minted}{bash}
export SACHOME=/usr/local/sac
export SACAUX=${SACHOME}/aux
export PATH=${SACHOME}/bin:${PATH}

export SAC_DISPLAY_COPYRIGHT=1
export SAC_PPK_LARGE_CROSSHAIRS=1
export SAC_USE_DATABASE=0
\end{minted}

其中,
\begin{itemize}
\item !SACHOME! 为SAC的安装目录
\item !SACAUX! 目录中包含了SAC运行所需的辅助文件
\item !PATH! 为系统环境变量
\item !SAC_DISPLAY_COPYRIGHT! 用于控制是否在启动SAC时显示版本和版权
    信息,一般设置为1。在脚本中多次调用SAC时会重复显示版本和版权信息,
    干扰脚本的正常输出,因而在脚本中一般将其值设置为0。具体的设置方法
    可以参考``\nameref{chap:sac-script}''中的相关内容;
\item !SAC_PPK_LARGE_CROSSHAIRS! 用于控制震相拾取过程中光标的大小,
    在 \nameref{sec:phase-picking} 时会用到
\item !SAC_USE_DATABASE! 用于控制是否允许将SAC格式转换为GSE2.0格式,
    一般用不到该特性,故而设置其值为0;
\end{itemize}

修改完 !~/.bash_profile! 后,执行以下命令使配置的环境变量生效:
\begin{minted}{console}
$ source ~/.bash_profile
\end{minted}

\subsection{启动SAC}
终端键入小写的sac,显示如下则表示SAC安装成功:
\begin{minted}{console}
$ sac
 SEISMIC ANALYSIS CODE [11/11/2013 (Version 101.6a)]
 Copyright 1995 Regents of the University of California

SAC>
\end{minted}
