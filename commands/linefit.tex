\SACCMD{linefit}
\label{cmd:linefit}

\SACTitle{概要}
对内存中数据的进行最小二乘线性拟合

\SACTitle{语法}
\begin{SACSTX}
LINEFIT
\end{SACSTX}

\SACTitle{说明}
此命令的底层实现与 \nameref{cmd:rmean} 命令是相同的。

对数据使用最小二乘拟合得到一条直线,并将拟合结果写到黑板变量中:
\begin{itemize}
\item \texttt{SLOPE}:直线的斜率
\item \texttt{YINT}:Y轴截距
\item \texttt{SDSLOPE}:斜率的标准差
\item \texttt{SDYINT}:截距的标准差
\item \texttt{SDDATA}:数据的标准差
\item \texttt{CORRCOEF}:数据和模型间的相关系数
\end{itemize}

\SACTitle{示例}
\begin{SACCode}
SAC> fg seis
SAC> linefit            // 线性拟合
 Slope and standard deviation are: 0.00023042 0.0035114
 Intercept and standard deviation are: -0.10165 0.048355
 Data standard deviation is: 0.32054
 Data correlation coefficient is: 0.0020772
SAC> getbb             // 查看黑板变量
 CORRCOEF     = 0.00207718
 NUMERROR = 0
 SACERROR = 'FALSE'
 SACNFILES = 1
 SDDATA     = 0.32054
 SDSLOPE     = 0.00351136
 SDYINT     = 0.0483548
 SLOPE     = 0.000230417
 YINT     = -0.10165
SAC> getbb SLOPE       // 查看单个头段变量时出错,猜测是bug
 ERROR 1201: Could not find VARS variable SLOPE
\end{SACCode}
