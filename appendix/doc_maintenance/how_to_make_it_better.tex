\section{如何让《SAC参考手册》变得更好?}
《SAC参考手册》是一本详细介绍SAC用法的中文电子书。
2011年暑假,我尝试翻译了官方的英文文档,2012年初将其整理为PDF,并放在网络上免费下载。
至今,手册依然在频繁更新。
\begin{itemize}
\item 文档发布页:\url{http://seisman.info/sac­manual.html}
\item 项目主页:\url{https://github.com/seisman/SAC_Docs_zh }
\end{itemize}

这本手册已经帮助了很多人,我相信其中也包括你。
我希望你能为完善这本电子书提供帮助。如果你发现错误或者有任何建议,请发邮件给我。
如果你有时间、精力,肯定人人为我,我为人人的互助精神,希望参与一个开源项目,那么你可以考虑成为这本电子书的维护者。
\subsection{维护者的职责与权利}
维护者的职责主要是:
\begin{enumerate}
\item 随着SAC软件新版本的发布,新增或更新手册中相应的内容;
\item 完善未讲清楚或易引起歧义的部分;
\item 改正错别字;
\item 增加日常数据处理过程中的一些SAC技巧;
\item 对于命令及用法,若不同版本间存在不兼容,则以最新版本的SAC为准,并在正文或脚注中适当提及版本间的差异。
\item 增加你认为值得加入的部分;
\item 文档整体布局的调整以及LaTeX源码的简化;
\end{enumerate}

维护者完全无偿工作,没有任何物质报酬(若是有机会,我可以请吃顿饭)。
除此之外,维护者还可以得到如下一些虚无缥缈的东西:
\begin{enumerate}
\item 手册中会将你列为维护者,并留有邮箱;
\item 结识更多的SAC用户以及地震学同行;
\item 了解MarkDown和LaTeX的基础语法;
\item 学会使用Git,参与开源项目;
\end{enumerate}
\subsection{对维护者的最低要求}
如果你想参与进来,你必须满足以下最低要求:
\begin{enumerate}
\item SAC用户(SAC初学者);
\item 拥有Linux操作系统;
\item 有时间和精力;
\item 了解LaTeX、Git的基础知识(至少愿意学习和了解);
\end{enumerate}
\subsection{如何参与}
若有意参与到该手册的维护,请参考如下步骤,过程中遇到问题可以邮件联系我。
简单的说,参与该项目需要如下步骤:
\begin{enumerate}
\item 拥有Linux系统;
\item 拥有一个GitHub账户;
\item 安装TexLive(可选);
\item Fork手册源码,Clone至本地;
\item 修改源码,提交修改,并Push到自己的Repo里,再提交Pull Request;
\end{enumerate}
维护者需要了解三个工具:Linux、Git和LaTeX。其中,对于Git的要求较高,其余两者没有太多
要求。
关于如何利用LaTeX编译生成PDF,请参考项目主页里的README。
