\section{黑板变量}
既然是SAC编程,就必然少不了变量,SAC中的变量称之为黑板变量。

黑板变量是SAC中用于临时储存和取回信息而设计的。黑板变量不需要声明即可
直接使用,可以用 \nameref{cmd:setbb} 和 \nameref{cmd:evaluate} 命令给
黑板变量赋值,用 \nameref{cmd:getbb} 获取黑板变量的值。也可以用
\nameref{cmd:writebbf} 将黑板变量保存在磁盘文件中,然后使用
\nameref{cmd:readbbf} 命令重新将这些变量读入SAC中。

引用黑板变量的值的方式为:``!%bbvname%!'',其中 !bbvname!
为黑板变量的变量名。
\begin{SACCode}
SAC> echo on processed
SAC> fg seis
SAC> p
SAC> setbb low 2.45         // 黑板变量low=2.45
SAC> setbb high 4.94        // 黑板变量high=4.94
SAC> bp c %low% %high%      // 引用黑板变量low和high的值作为滤波的频带
 ==>  bp c 2.45 4.94        // echo on processed 显示代入值后的命令
SAC> p
SAC> getbb low high         // 查看黑板变量的值
 low = 2.45
 high = 4.94
\end{SACCode}

下例展示了如何将黑板变量写入磁盘文件,等需要时再从磁盘文件中获取:
\begin{SACCode}
$ sac
SAC> setbb var1 10          // 整型
SAC> setbb var2 "text"      // 字符串
SAC> setbb var3 0.2         // 浮点型
SAC> wbbf bbf.file          // 写入到文件
SAC> q
$ ls
bbf.file
$ sac
SAC> readbbf ./bbf.file
SAC> getbb
 NUMERROR = 0
 SACERROR = 'FALSE'
 SACNFILES = 0
 VAR1 = 10
 VAR2 = 'text'
 VAR3 = 0.2
SAC> getbb var2 var3
 var2 = 'text'
 var3 = 0.2
SAC> q
\end{SACCode}
