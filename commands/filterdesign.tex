\SACCMD{filterdesign}
\label{cmd:filterdesign}

\SACTitle{概要}
产生一个滤波器的数字和模拟特性的图形显示,包括:振幅,相位,脉冲响应和群延迟。

\SACTitle{语法}
\begin{SACSTX}
F!ILTER!D!ESIGN! [FILE [prefix]] [filteroptions] [delta]
\end{SACSTX}
其中filteroptions与SAC中其他的滤波命令相同,包括滤波器类型。delta是数据的采样间隔。

注意:选项的顺序很重要,如果使用了PRINT选项,则其必须是第一个选项。如果使用了FILE选项,其必须放在滤波器选项之前。

\SACTitle{输入}
\begin{description}
\item [FILE prefix] 生成的三个SAC文件的前缀
\end{description}

这三个SAC文件分别为prefix.spec、prefix.grd、prefix.imp。其中prefix.spec为
该命令产生的振幅相位信息,为振幅-相位格式谱文件。prefix.gd为该命令产生的群延迟信息,
是时间序列文件。需要注意的是,尽管这个文件是时间序列文件,但是实际上群延迟是频率的函数。
用户要记住,虽然绘图时横轴单位是秒,实际的单位却是Hz。prefix.imp是时间序列文件,
包含脉冲响应信息。

在这三个SAC文件中,用户自定义头段变量USERn、KUSERn设置如下:
\begin{itemize}
\item user0: 表示pass code。1代表LP;2代表HP;3代表BP;4代表BR;
\item user1: type code。1代表BU,2代表BE,3代表C1,4代表C2;
\item user2:  number of poles
\item user3:  number of passes
\item user4:  tranbw
\item user5:  attenuation
\item user6:  delta
\item user7:  first corner
\item user8:  second corner if present, or -12345 if not
\item kuser0: pass (lowpass, highpass, bandpass, or bandrej)
\item kuser1: type (Butter, Bessel, C1, or C2 )
\end{itemize}

\SACTitle{缺省值}
只有delta参数有一个缺省值(0.025s)。其他参数必须给出

\SACTitle{说明}
FILTERDESIGN命令使用了一个工具程序xapiir,它是一个递归数字滤波器包。xapiir通过模拟滤波器原型的双线性变换实现标准递归数字滤波器的设计。这些原型滤波器由零点和极点给定,然后使用模拟谱变换,变换到高通、带通和带阻滤波器。

FILTERDESIGN用实线显示数字滤波器响应,用虚线显示模拟滤波器响应。在彩色显示器上,数字曲线是蓝色的而模拟曲线是琥珀色的。

\SACTitle{示例}
下面的例子展示了如何使用FILTERDESIGN命令产生一个高通,拐角频率为2Hz,六极、双通滤波器的数字和模拟响应曲线,数据采样间隔为0.025s:
\begin{SACCode}
SAC> fd hp c 2 n 6 p 2 delta .025
\end{SACCode}

\SACTitle{相关命令}
\nameref{cmd:highpass}、\nameref{cmd:lowpass}、\nameref{cmd:bandpass}、\nameref{cmd:bandrej}
