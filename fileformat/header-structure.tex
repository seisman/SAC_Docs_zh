\section{SAC头段结构}
SAC头段区的大小为632个字节,由一系列头段变量构成。从这些头段变量中可以了解到波形
记录的很多信息,比如台站经纬度、发震时刻、震相到时等等。

表~\ref{table:header-variables}~列出了SAC头段区的全部头段变量。

\begin{table}[H]
\ttfamily
\small
\centering
\caption{SAC头段变量列表}
\label{table:header-variables}
\begin{tabular}{c|c|lllll}
	\toprule
    Byte	&	Type	&	\multicolumn{5}{c}{Names}\\
	\midrule
	0		&	F	&	delta	&	depmin	&	depmax	&	scale	&	odelta	\\
	20		&	F	&	b		&	e		&	o		&	a		&	internal\\
	40		&	F	&	t0		&	t1		&	t2		&	t3		&	t4		\\
	60		&	F	&	t5		&	t6		&	t7		&	t8		&	t9		\\
	80		&	F	&	f		&	resp0	&	resp1	&	resp2	&	resp3	\\
	100		&	F	&	resp4	&	resp5	&	resp6	&	resp7	&	resp8	\\
    120		&	F	&	resp9	&	stla	&	stlo	&	stel	&	stdp	\\
	140		&	F	&	evla	&	evlo	&	evel	&	evdp	&	mag		\\
	160		&	F	&	user0	&	user1	&	user2	&	user3	&	user4	\\
	180		&	F	&	user5	&	user6	&	user7	&	user8	&	user9	\\
	200		&	F	&	dist	&	az		&	baz		&	gcarc	&	internal\\
	220		&	F	&	internal&	depmen	&	cmpaz	&	cmpinc	&	xminimun\\
	240		&	F	&	xmaximum&	yminimum&	ymaximum&	unused	&	unused	\\
	260		&	F	&	unused	&	unused	&	unused	&	unused	&	unused	\\
	280		&	N	&	nzyear	&	nzjday	&	nzhour	&	nzmin	&	nzsec	\\
	300		&	N	&	nzmsec	&	nvhdr	&	norid	&	nevid	&	npts	\\
	320		&	N	&	internal&	nwfid	&	nxsize	&	nysize	&	unused	\\
	340		&	I	&	iftype	&	idep	&	iztype	&	unused	&	iinst	\\
	360		&	I	&	istreg	&	ievreg	&	ievtyp	&	iqual	&	isynth	\\
    380		&	I	&	imagtyp &	imagsrc	&	unused	&	unused	&	unused	\\
	400		&	I	&	unused	&	unused	&	unused	&	unused	&	unused	\\
	420		&	L	&	leven	&	lpspol	&	lovrok	&	lcalda	&	unused	\\
	440		&	K	&	kstnm	&	kevnm*	&			&			&			\\
	464		&	K	&	khole	&	ko		&	ka		&			&			\\
	488		&	K	&	kt0		&	kt1		&	kt2		&			&			\\
	512		&	K	&	kt3		&	kt4		&	kt5		&			&			\\
	536		&	K	&	kt6		&	kt7		&	kt8		&			&			\\
	560		&	K	&	kt9		&	kf		&	kuser0	&			&			\\
	584		&	K	&	kuser1	&	kuser2	&	kcmpnm	&			&			\\
	608		&	K	&	knetwk	&	kdatrd	&	kinst	&			&			\\
    \bottomrule
\end{tabular}
\end{table}

在源代码中,整个头段区被定义为一个结构体。结构体中的第一个头段变量是~\verb+delta+~,
第二个头段变量是~\verb+depmin+~,第六个头段变量是~\verb+b+~,以此类推。表的
第一列给出了当前行的第一个头段变量在文件中(或在结构体中)的起始字节。第二列给出了
当前行的头段变量的变量类型码,包括F、N、I、L、K型,每个变量类型码的具体含义参见表
~\ref{table:header-variables-type}。

头段区中,共有头段变量133个。若变量名为~\verb+internal+~,表示该变量为SAC内部使用
的头段变量,用户不可对其进行操作。若变量名为~\verb+unused+~表示该变量暂时尚未使用,
为以后可能出现的新头段变量占位。


表~\ref{table:header-variables-type}~列出了SAC头段中的全部头段变量类型及其相关信息。
第一列为头段变量类型代码,第二类给出了其
代表的头段变量类型,第三列指出C源码中该变量的是用什么类型定义的,第四列给出了每个
变量所占据的字节数。第五列给出了写字符型SAC文件时的输出格式。最后一列则给出该类型
的未定义值。

\begin{table}[H]
\caption{变量类型说明}
\label{table:header-variables-type}
\centering
\ttfamily
\small
\begin{tabular}{cllcll}
	\toprule
    Code    &	Type        &   C Type & sizeof &   printf	&   未定义值        \\
	\midrule
    F		&	浮点型		&   float  &  4     &	\%15.7f &   -12345.0        \\
    N		&	整型		&   int    &  4     &	\%10d   &   -12345        \\
    I		&	枚举型		&   int    &  4     &	\%10d   &   -12345	        \\
    L		&	逻辑型		&   int    &  4     &	\%10d   &   FALSE        \\
    K		&	字符型		&   char*  &  8     &	\%-8.8s & \lstinline[showspaces=true]{"-12345  "}     \\
    A		&	辅助型		&          &        &			& 	    \\
	\bottomrule
\end{tabular}
\end{table}

SAC中定义了6种头段变量类型,分别为浮点型(F)、整型(N)、枚举型(I)、逻辑型(L)、字符型(K)和辅助型(A)。
其中辅助型不存在于SAC头段区中,其直接由其它头段变量推导得到。

除了F型变量外以外,其余所有头段变量的变量名均以变量类型码开头,比如~\verb+nvhdr+~
是N型变量,~\verb+leven+~是L型变量。

枚举型变量,本质上是~\verb+int+~型,其只能在固定的几个值中取值;C语言中本身
是没有规定Bool类型的,SAC自定义了~\verb+TRUE+~和~\verb+FALSE+~,用于表示真和假。
字符型变量长度为8,只有~\verb+kevnm+~很特殊,其长度为16。

对于一个SAC文件,并非所有的头段变量都必须要包含
有意义的值,若变量未定义称其包含未定义值,第6列给出了不同类型的未定义值的形式,
比如如果一个整型头段变量的值为-12345,则认为其处于未定义态。实际使用的时候,SAC
提供了参数~\verb+undef+~,其可以根据头段变量的类型自动转换成相应类型的未定义值。
