\SACCMD{plabel}
\label{cmd:plabel}

\SACTitle{概要}
定义通用标签及其属性

\SACTitle{语法}
\begin{SACSTX}
PLABEL [n] [ON|OFF|text] [SIZE T!INY!|S!MALL!|M!EDIUM!|L!ARGE!]
    [BELOW|POSITION x y [a]]
\end{SACSTX}

\SACTitle{输入}
\begin{description}
\item [n] 通用标签号。若省略,则在前一个标签号上加1
\item [ON] 打开绘图标签选项,但不改变标签文本
\item [OFF] 关闭绘图标签选项
\item [text] 改变绘图标签的文本内容,同时打开了绘图标签选项
\item [SIZE TINY|SMALL|MEDIUM|LARGE] 修改绘图标签的文本尺寸。TINY、SMALL、MEDIUM、LARGE分别表示
    一行132、100、80、50个字符。
\item [BELOW] 将此标签放在前一标签的下面
\item [POSITION x y a] 定义该标签的位置。其中x的取值为0到1,y的取值为0到最大视口(一般为0.75),a是标签相对于水平方向顺时针旋转的角度。
\end{description}

\SACTitle{缺省值}
默认字体大小为small,标签1的位置为0.15 0.2 0.。默认其他标签的位置为上一个标签之下。

\SACTitle{说明}
这个命令允许你为接下来的绘图命令定义通用的绘图标签。你可以定义每个标签的位置及文本尺寸。
文本质量以及字体可以用GTEXT命令设定,也可以使用TITLE、XLABEL、YLABEL生成图形的标题以及轴标签。

\SACTitle{示例}
下面的命令将在接下来的绘图中在左上角产生一个四行的标签:
\begin{SACCode}
SAC> PLABEL 'Sample seismogram' POSITION .12 .5
SAC> PLABEL 'from earthquake'
SAC> PLABEL 'on January 24, 1980'
SAC> PLABEL 'in Livermore Valley, CA'
\end{SACCode}

一个额外的小标签可以放在左下角:
\begin{SACCode}
SAC> PLABEL 5 'LLNL station: CDV' S T P .12 .12
\end{SACCode}

\SACTitle{相关命令}
\nameref{cmd:gtext}、\nameref{cmd:title}、\nameref{cmd:xlabel}、\nameref{cmd:ylabel}
