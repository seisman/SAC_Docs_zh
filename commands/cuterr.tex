\SACCMD{cuterr}
\label{cmd:cuterr}

\SACTitle{概要}
控制坏的截窗参数引起的错误

\SACTitle{语法}
\begin{SACSTX}
CUTERR FA!TAL!|U!SEBE!|FI!LLZ!
\end{SACSTX}

\SACTitle{输入}
\begin{description}
\item [FATAL] 将截窗错误设置为致命
\item [USEBE] 将坏的起始和结束截窗参数设置为文件开始和文件结束
\item [FILLZ] 在文件开始时间之前或文件结束时间之后填补适当数目的0以弥补坏的截窗参数
\end{description}

\SACTitle{缺省值}
对于信号迭加子程序默认值为FILLZ,其他的默认值为USEBE

\SACTitle{说明}
这个命令控制由于坏的截窗参数引起的错误条件。可以将这些错误定义为致命错误。
如果裁剪参数的起始值或结束值在文件头段中未定义,则可以选择为USEBE。如果
定义了要截取的时间窗但是其截窗起始值小于文件起始值或者截窗参数结束值大于
文件结束值,则可以分别用文件起始结束值代替截窗参数,或者也可以使用FILLZ
在文件前后补适当的0。

\SACTitle{示例}
假设文件FILE1起始时间为B=\SI{25}{s},初动到时A=\SI{40}{s},采样率为 \SI{0.01}{s}。
\begin{SACCode}
SAC> cut a -20 e
SAC> read file1
\end{SACCode}
截窗起始值为 \SI{20}{s},产生了一个错误条件。在USEBE模式下,截窗起始值
将替换为 \SI{25}{s}(即B)。在FILLZ模式下,在数据之前将将插入500个零值(5秒钟,
每秒100个点),截窗起始值保持为 \SI{20}{s}。

\SACTitle{相关命令}
\nameref{cmd:cut}、\nameref{cmd:read}
