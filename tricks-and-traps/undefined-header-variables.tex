\section{未定义的头段变量}
处理数据的时候会遇到一种情况,用SAC读入数据后,可以用lh查看到某个头段
变量的值,但是直接使用 !saclst! 却发现头段变量的值未定义。

直接读入SAC查看头段变量 !dist! 的值:
\begin{SACCode}
SAC> r XXXX.SAC
SAC> lh dist
     dist = 3.730627e+02
\end{SACCode}

用saclst查看同一个文件的头段变量dist的值:
\begin{minted}{console}
$ saclst dist f XXXX.SAC
XXXX.SAC        -12345.0
\end{minted}

用两种方法查看头段变量的值得到的结果不同,出现这种情况的原因,这个SAC
数据中 !dist! 本身是没有定义的,当SAC读入该数据时,会自动计算并
更新 !dist! 的值,所以使用 !lh! 会得到正确的 !dist!
值,而 !saclst! 是直接读取数据文件的头段,并不会对重新计算,因而
!saclst! 得到的是未定义值。也就是说,!saclst! 得到的是
文件中保存的值,!lh! 得到的是数据中应该保存的值。

如果想要 !saclst! 也获取正确的值,可以先用SAC把数据读进去,待
SAC把头段更新后,再写回到磁盘中:
\begin{SACCode}
SAC> r *.SAC
SAC> wh
SAC> q
\end{SACCode}

经常会出现这些问题的头段变量,换句话说,SAC在读入数据时会自动更新的
头段变量包括:!depmax!、!depmin!、!depmen!、
!e!、!gcarc!、!dist! 等。
