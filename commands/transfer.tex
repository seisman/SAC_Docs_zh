\SACCMD{transfer}
\label{cmd:transfer}

\SACTitle{概要}
反卷积以去除仪器响应,如果需要,还可以卷积其他仪器响应

\SACTitle{语法}
\begin{SACSTX}
TRANS!FER! [FROM type [options]] [TO type [options]] [FREQ!LIMITS! f1 f2 f3 f4]
    [PREWH!ITENING! ON|OFF|n]
\end{SACSTX}

\SACTitle{输入}
\begin{description}
\item [FROM type] 要去除的仪器响应类型
\item [TO type] 要加入的仪器响应类型
\item [FREQLIMITS f1 f2 f3 f4] 压制大于 !f4! 以及低于 !f1!的频段
\item [PREWHITENING ON|OFF|n] 预白化处理
\end{description}

\SACTitle{缺省值}
\begin{SACDFT}
trans from none to none
\end{SACDFT}

\SACTitle{说明}
\subsubsection{FROM and TO}
!transfer! 命令的作用是将波形数据从一种仪器响应转换成另一种仪器
响应。!FROM type! 指定要从波形数据中去除的仪器响应,!TO type!
指定了要加入到波形数据中的仪器响应。去仪器响应即反卷积,加仪器响应即卷积,
二者分别通过谱域的除法和乘法完成。

!type! 为仪器类型,可以是SAC预定义的标准仪器类型(见附录中表
\ref{table:instrument-type}),还可以是如下几种特殊的``仪器类型'':
\begin{description}
\item [none] 表示``位移''
\item [vel] 表示``速度''
\item [acc] 表示``加速度''
\item [evalresp] 表示使用RESP仪器响应文件
\item [polezero] 表示使用SAC PZ仪器响应文件
\item [fap] 表示使用fap仪器响应文件
\end{description}

!tranfser! 命令的默认值是``!transfer from none to none!'',
即默认的输入和输出``仪器类型''都是位移。因而当不指定 !FROM type!
或 !TO type! 时,SAC会假定仪器类型为 !NONE!。

\begin{itemize}
\item 若输出的仪器类型为 !NONE!,即表示从波形中去除仪器响应得到
    位移,此时SAC头段中的IDEP设置为 !IDISP!,单位为 \si{nm},
    若输出类型为 !VEL! 或 !ACC!,同理;
\item 若输出的仪器类型不是 !NONE!、!VEL! 或 !ACC!,
    则内存中的波形会卷积上 !TO type! 所指定的仪器响应;
\item 若不指定 !FROM! 选项,则假定原始波形数据是位移,且不会去除
    仪器响应;通常用于给理论地震图添加仪器响应;
\end{itemize}

\subsubsection{freqlimits}
!freqlimits! 用于在去除仪器响应时对波形的低频和高频部分进行压制。
当 !TO type! 为 !NONE!、!VEL! 或 !ACC!时,
必须使用该选项,且必须认真选择合适的参数。

所有地震仪在零频率处都具有零响应,在只进行反卷积时,需要在频率域做谱的
除法,此时可能会除以一个很小的值进而导致低频处有很大的谱振幅;在高频处,
信噪比通常很低,因而有必要对其响应进行压制。

!freqlimits! 会在去仪器响应时对频谱做低通和高通尖灭,以实现对高频
和低频的压制。四个频率参数应满足 !f1<f2<f3<f4!,即尖灭函数在零到
!f1! 之间为0,!f1! 到 !f2! 之间从0渐渐变成1,在
!f2! 和 !f3! 之间保持为1,在 !f3! 到 !f4!
之间从1渐渐变成0,大于 !f4! 的频段值为0。过渡带内分别为余弦波的
四分之一周期。如下图所示:

\begin{figure}[H]
\centering
\includegraphics[width=0.9\textwidth]{freqlimits}
\caption{Freqlimits尖灭函数}
\label{fig:freqlimits}
\end{figure}

四个频率参数除了要满足 !f1<f2<f3<f4! 外,还应注意如下几条原则:
\begin{itemize}
\item !f4! 应小于Nyquist采样率。比如若数据的采样周期为 \SI{0.01}{\s},
    则Nyquist采样率为 \SI{50}{\Hz},因而 !f4! 应小于\SI{50}{\Hz}
\item !f3! 不能与 !f4! 太接近
\item !f2! 与 !f3! 之间应尽可能宽,然后再根据具体需求进行滤波
\item !f1! 和 !f2! 不能太接近;
\item !f1! 的选取由具体需求决定,可以尝试不同的值并查看去仪器响应
    之后的效果来决定
\end{itemize}

若想要一个低通滤波器但在低频处不滤波,可以设置 !f1=-2! 和 !f2=-1!;
若想要一个高通滤波器但在高频处不滤波,可以设置 !f3! 等于Nyquist频率,
!f4! 为Nyquist频率的两倍。

需要注意,该滤波器是零相位、非因果滤波器,因而,若数据点数不为2的指数幂次,
会导致在频段 !(f1,f4)! 之外振幅不完全为0。若想要数据点数为2的幂次方,
可以参考SAC中的 \nameref{cmd:cut} 命令。

\subsubsection{prewhitening}
!prewhitening! 用于控制数据的预白化。预白化可以将输入时间序列在
变换到频率域之前,进行谱的平化。这会减小谱值的动态范围,并提高数据在高频
的计算精度。参见 \nameref{cmd:whiten} 命令。打开预白化选项,会在谱操作
之前在频率域进行谱白化,并在谱操作后在时间域做谱白化的补偿,也可以设置
预白化选项的阶数。默认情况下,预白化选项是关闭的,阶数为 !n=6!。

\SACTitle{示例}
\subsubsection{内置仪器类型}
SAC中内置了一堆预定义的仪器类型,可以在命令中直接使用。

从数据中去除LLL宽频带仪器响应。并卷积上SRO仪器响应,且对频带做尖灭及预白化:
\begin{SACCode}
SAC> read abc.z
SAC> rmean; rtr; taper
SAC> trans from lll to sro freq .02 .05 1. 2. prew 2
\end{SACCode}

当前的仪器类型为RSTN的子类型nykm.z,为了去除该仪器响应并卷积上DSS仪器响应:
\begin{SACCode}
SAC> read nykm.z
SAC> rmean; rtr; taper
SAC> trans from rstn subtype nykm.z to dss prew off
\end{SACCode}

将电磁仪器响应转换成位移:
\begin{SACCode}
SAC> r XYZ.Z
SAC> trans from elmag freep 15. mag 750. to none
\end{SACCode}

从波形中去除WWSP的仪器响应,得到位移波形:
\begin{SACCode}
SAC> read xyz.z
SAC> rmean; rtr; taper
SAC> trans from WWSP to none freq 0.05 0.01 5 10
                // 也可使用to vel或to acc得到速度或加速度
\end{SACCode}

向合成的位移地震图中加入WWSP仪器响应:
\begin{SACCode}
SAC> r syn.z
SAC> trans from none to WWSP    // 简写为trans to WWSP
\end{SACCode}

\subsubsection{evalresp类型}
!evalresp! 类型并不代表真正意义上的仪器类型,而是表示从RESP仪器
响应文件中读取仪器响应信息。在使用 !evalresp! 选项时,
\nameref{cmd:transfer} 依次从当前内存中的SAC波形数据中提取出各自的
头段信息,包括:!kstnm!、!kcmpnm!、!kzdate!、
!kztime!、!knetwk! 和 !locid!,然后会在当前目录下
寻找文件名为``!RESP.<NET>.<STA>.<LOCID>.<CHN>!''的RESP文件
(比如``RESP.IU.COLA..BHZ''),并检测RESP文件中给出的台站信息是否与数据
中的台站信息匹配\footnote{即,要求RESP文件名以及RESP文件中的台站信息都与
数据头段中的台站信息匹配}。
\begin{SACCode}
SAC> r 2006.253.14.30.24.0000.TA.N11A..LHZ.Q.SAC
SAC> rtr; rtr; taper
SAC> trans from evalresp to none freq 0.004 0.007 0.2 0.4
\end{SACCode}
该命令会首先从头段中提取台站信息,然后自动在当前目录下寻找文件
!RESP.TA.N11A..LHZ!,一旦文件中的台站信息与数据中的台站信息匹配,
则使用该响应函数。

SAC数据中的头段信息可以用一些选项来覆盖:
\begin{verbatim}
    STATION, CHANNEL, NETWORK, DATE, TIME, LOCID, FNAME
\end{verbatim}
每个选项都必须有一个合适的值。若 !DATE! 在SAC头段中未设定且在选项
中未指定,则使用当前系统日期,!TIME! 同理;若 !NETWORK!未
指定,则默认使用任意台网名;若 !LOCID! 或 !KHOLE! 未指定,
则默认使用任意LOCID。

假设台网IU的所有台站都具有完全相同的仪器响应函数,而此时你只有COLA台站的
RESP文件 !RESP.IU.COLA..BHZ!。为了给所有台站去除仪器响应,一种
办法是对IU台网的每一个台站复制一份 !RESP.IU.COLA..BHZ!,重命名,
并修改RESP文件中的台站信息。显然,这样很麻烦,利用上面的选项可以大大简化
这一过程:
\begin{SACCode}
SAC> r *.IU.*.BHZ
SAC> rmean; rtr; taper
SAC> trans from evalresp STATION COLA to none freq 0.01 0.02 5 10
\end{SACCode}
使用 !STATION! 选项覆盖了波形数据中的台站名,此时,对每一个波形数据,
!transfer! 命令都会去使用 !RESP.IU.COLA..BHZ!\footnote{这里
假定所有台站的LOCID都是未定义的}。

下面的命令会将三分量数据去仪器响应,并卷积上BHZ分量的仪器响应:
\begin{SACCode}
SAC> r *.IU.COLA.00.BH?
SAC> rmean; rtr; taper
SAC> trans from evalresp to evalresp CHANNEL BHZ
\end{SACCode}
操作完成后,BHZ分量相当于没有进行操作,BH1和BH2则去除了原本的仪器响应
并卷积上BHZ的仪器响应。

为了显示IU台网COL台站BHZ通道,1992年01月02日16:42:05的仪器响应:
\begin{SACCode}
SAC> fg impulse npts 16384 delta .05 begin 0.
SAC> trans to evalresp sta COL cha BHZ net IU \
                    date 1992/2 time 16:42:05
SAC> fft
SAC> psp am
\end{SACCode}

如果你的RESP文件名与SAC的标准格式不同,可以使用 !FNAME! 选项强制
指定要使用的RESP文件:
\begin{SACCode}
SAC> r 2006.253.14.30.24.0000.TA.N11A..LHZ.Q.SAC
SAC> rmean; rtr; taper
SAC> trans from evalresp fname /tmp/Resp/RESP.TA.N11A..LHZ to none \
                        freq 0.004 0.007 0.2 0.4
\end{SACCode}
!transfer! 命令默认会使用 !RESP.TA.N11A..LHZ! 作为响应文件,
此处使用 !FNAME! 选项强制指定使用 !/tmp/RESP/RESP.TA.N11A..LHZ!。
需要注意的是,即便是使用 !FNAME! 强制指定了RESP文件,该命令还是
会检测台站信息是否匹配。

由于一个RESP文件中可以包含多个响应函数,因而可以将所有仪器响应文件合并到
一个总的RESP文件中:
\begin{SACCode}
SAC> r *.SAC
SAC> rmean; rtr; taper
SAC> transfer from evalresp fname RESP.ALL to none freq 0.1 0.2 5 10
\end{SACCode}
这个例子中,!RESP.ALL! 包含了所有数据的响应函数,!transfer!
命令会读取 !RESP.ALL! 文件的内容,对于每一个波形数据,会从波形
数据中提取出台站信息,并与 !RESP.ALL! 中的众多响应函数进行匹配,
若匹配成功,则使用该响应函数。

\subsubsection{polezero类型}
!polezero! 类型并不代表真正意义上的仪器类型,而是表示从SAC零极点
文件中读取仪器响应函数。

polezero类型会从数据波形中提取台站信息,但不会根据台站信息去寻找默认的
PZ文件,用户必须使用 !subtype! 来指定要使用的PZ文件。若PZ文件
有注释行,则注释行中的台站信息必须与波形中的台站信息匹配,才能正确执行;
若PZ文件中无注释行,则不进行台站信息匹配的检测,直接执行。
\begin{SACCode}
SAC> r *IU.COLA.BHZ
SAC> rmean; rtr; taper
SAC> trans from polezero subtype SAC_PZs.IU.COLA.BHZ to WWSP
\end{SACCode}

一个PZ文件中可以包含多台站、多通道、多时间段的响应函数。可以将所有数据的
PZ文件合并得到总的PZ文件。下面的例子中读入全部波形数据,并利用总PZ文件
进行去仪器响应:
\begin{SACCode}
SAC> r *.SAC          // 读入全部数据
SAC> rmean; rtr; taper
SAC> trans from polezero s event.pz to none freq 0.05 0.1 10.0 15.0
SAC> mul 1.0e9        // 需要乘以1.0e9 !!!!!
SAC> w over
\end{SACCode}

需要格外注意,在用PZ文件去仪器响应得到位移物理量时,得到的数据的单位是
\si{m},而SAC中默认的单位是 \si{nm},因而需要将数据乘以 !1.0e9!
将数据的单位转换成 \si{nm}。对于转换得到速度或加速度,同理。

\subsubsection{fap选项}
fap选项表明使用FAP文件作为响应函数。

假设有fapfile文件 !fap.n11a.lhz_0.006-0.2!,其名字表示频率段为
\SI{0.006}{\Hz} 到 \SI{0.2}{\Hz},
要从波形 !2006.253.14.30.24.0000.TA.N11A..LHZ.Q.SAC! 中移除该仪器响应:
\begin{SACCode}
SAC> r 2006.253.14.30.24.0000.TA.N11A..LHZ.Q.SAC
SAC> rtr
SAC> taper
SAC> trans from fap s fap.n11a.lhz_0.006-0.2 to none freq 0.004 0.006 0.1 0.2
SAC> mul 1.0e9
\end{SACCode}
