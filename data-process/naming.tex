\section{数据重命名}
用rdseed软件从SEED格式中解压得到的SAC数据,一般都具有固定格式的文件名。
示例如下:
\begin{verbatim}
    2012.055.12.34.56.7777.YW.MAIO.01.BHE.Q.SAC
    2012.055.12.34.50.6666.YW.MAIO.01.BHN.Q.SAC
    2012.055.12.34.54.5555.YW.MAIO.01.BHZ.Q.SAC
\end{verbatim}
这三个文件是YW台网MAIO台站的宽频地震仪记录的宽频带三分量(BHE、BHN、BHZ)
波形数据。文件名中每一项的具体含义在``\nameref{chap:naming}''中有介绍,
这里不再重复。

默认的文件名比较长,在数据处理时可能会显得比较麻烦,一般都会根据实际
需求进行适当的简化。

是否要对数据文件做重命名,以及按照什么格式重命名,都是没有固定的标准的。
通常需要用户根据自己所做研究的实际情况来决定。

在某些情况下,需要将同一事件在所有台站的波形数据放在同一个文件夹下,
并将文件名以事件的发生日期/时间来命名。那么,SAC文件名中的时间等信息
就可以被省略掉。数据文件名简化为:
\begin{verbatim}
    YW.MAIO.BHE
    YW.MAIO.BHN
    YW.MAIO.BHZ
\end{verbatim}

有时候,需要将不同事件在同一个台站的波形数据放在同一个文件夹下,并将
文件名以台站名来命名,此时数据文件名中可能需要保留事件的日期信息。数据
文件名可以简化为:
\begin{verbatim}
    YW.MAIO.20120224.BHE
    YW.MAIO.20120224.BHN
    YW.MAIO.20120224.BHZ
\end{verbatim}

数据重命名这一步骤可以单独执行,也可以在执行其他操作的过程中顺便进行
重命名(比如将数据合并并写入磁盘的时候)。通常需要写脚本来完成重命名的
操作,重命名脚本的Bash版本位于第 \pageref{subsec:rename-in-bash} 页,
Perl版本位于第 \pageref{subsec:rename-in-perl} 页,
Python版本位于第 \pageref{subsec:rename-in-python} 页。
