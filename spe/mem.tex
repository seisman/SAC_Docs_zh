\SACCMD{mem}
\label{spe:mem}

\SACTitle{概要}
利用最大熵方法计算谱估计

\SACTitle{语法}
\begin{SACSTX}
MEM [O!RDER! n] [N!UMBER! n]
\end{SACSTX}

\SACTitle{输入}
\begin{description}
\item [ORDER n] 设置预测误差滤波器的时滞阶数为n
\item [NUMBER n] 设置用于谱估计的点数
\end{description}

\SACTitle{缺省值}
\begin{SACDFT}
mem order 25
\end{SACDFT}

\SACTitle{说明}
该命令实现了最大熵谱估计法,该方法使用一个预测误差滤波器对数据进行白化处理,
得到的谱估计正比于滤波器的能量频率响应的倒数。用户可以自由选择预测误差滤波器
的阶数,详情参考\nameref{spe:plotpe}命令。

该方法的主要优点是用相对少量的数据即可获得相当高的分辨率,它的缺点是跟传统
方法相比没什么理论好说。

\SACTitle{相关命令}
\nameref{spe:cor}、\nameref{spe:writespe}、\nameref{spe:plotspe}
