\SACCMD{ifft}
\label{cmd:ifft}

\SACTitle{概要}
对数据进行离散反傅立叶变换

\SACTitle{语法}
\begin{SACSTX}
IFFT
\end{SACSTX}

\SACTitle{说明}
数据文件必须是之前利用FFT命令生成的谱文件,可以是实部-虚部格式或振幅-相位格式。

\SACTitle{头段变量}
B、DELTA和NPTS被修改为原始数据的起始时间、采样周期、数据点数。频率域的起始频率B、
采样频率DELTA、采样点数NPTS被保存到SB、SDELTA、NSNPTS中。

\SACTitle{错误消息}
\begin{itemize}
\item[-]1301: 未读入文件
\item[-]1305: 对时间序列的非法操作
\item[-]1606: 超过IFFT所允许的最大数据点数
\end{itemize}

\SACTitle{限制}
目前IFFT所允许的最大数据点数为65536。

\SACTitle{相关命令}
\nameref{cmd:fft}
