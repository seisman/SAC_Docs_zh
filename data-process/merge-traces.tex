\section{合并数据}
相关命令: \nameref{cmd:merge}

有些时候,从SEED数据中解压出来的同一台站同一分量的连续波形数据,
会被切割成多个等长或不等长的文件,数据断开的可能原因是仪器在某些时刻存在问题导致
连续数据出现间断,也可能是出于其它考虑将数据进行切割。此时需要首先对数据进行合并。

假定解压出来的台网NET、台站STA、位置为00的BHZ分量的连续波形被分割成了多个文件,
需要将多个文件合并成单个文件。

对于v101.6之前的版本,只能用如下命令进行合并:
\begin{SACCode}
SAC> r 2012.055.12.00.00.0000.NET.STA.00.BHZ.Q.SAC
SAC> merge 2012.055.12.25.00.0000.NET.STA.00.BHZ.Q.SAC
SAC> merge 2012.055.12.40.00.0000.NET.STA.00.BHZ.Q.SAC
SAC> ...
SAC> merge 2012.055.13.20.00.0000.NET.STA.00.BHZ.Q.SAC
SAC> w NET.STA.00.BHZ
\end{SACCode}
即先读取第一段数据,然后合并第二段数据,再合并第三段数据,对于多个数据段的合并,
需要执行多次合并命令,且合并时需要遵循文件的先后顺序。

SAC从v101.6开始重写了 \nameref{cmd:merge} 命令,可以使用如下更简洁的形式:
\begin{SACCode}
SAC> r *.NET.STA.00.BHZ        // 读入所有需要合并的文件
SAC> merge                     // 内存中的所有文件被合并为一个文件
SAC> w NET.STA.00.BHZ          // 写回到磁盘中
\end{SACCode}

当然,SAC v101.6还提供了更简洁的形式:
\begin{SACCode}
SAC> merge *.NET.STA.00.BHZ     // 读取需要合并的文件,并合并
SAC> w NET.STA.00.BHZ           // 写回到磁盘
\end{SACCode}

对于所有要合并的数据文件,SAC会检测其knetwk、kstnm、kcmpnm和delta是否完全匹配,并
智能判断每个文件的合并顺序。

实际合并的过程中,可能会出现数据间断或数据重叠的情况。若数据存在间断,可对其直接
补零或线性插值;若数据存在重叠,则可以比较重叠部分数据是否相同或对重叠的波形进行平均。
