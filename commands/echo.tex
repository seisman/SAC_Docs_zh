\SACCMD{echo}
\label{cmd:echo}

\SACTitle{概要}
控制输入输出回显到终端

\SACTitle{语法}
\begin{SACSTX}
ECHO ON|OFF E!RRORS!|W!ARNINGS!|O!UTPUT!|C!OMMANDS!|M!ACROS!|P!ROCESS!
\end{SACSTX}

\SACTitle{输入}
\begin{description}
\item [ON|OFF] 打开/关闭列出的项的回显选项
\item [ERRORS] 命令执行过程中生成的错误信息
\item [WARNINGS] 命令执行过程中生成的警告信息
\item [OUTPUT] 命令执行过程中生成的输出信息
\item [COMMANDS] 终端键入的原始命令
\item [MACROS] 宏文件中出现的原始命令
\item [PROCESSED] 经过处理后的终端命令或宏文件命令,包括宏参数、黑板
    变量、头段变量、内联函数的计算和代入
\end{description}

\SACTitle{缺省值}
\begin{SACDFT}
echo on errors warnings output off commands macros processed
\end{SACDFT}

\SACTitle{说明}
该命令控制SAC输入输出流中哪一类要被回显到终端或屏幕。

输出分为三大类:错误消息、警告消息、输出消息;输入也分为三大类:终端键入
的命令、宏文件中执行的命令以及处理后的命令。处理后的命令指所有的宏参数、
黑板变量、头段变量、内联函数首先被计算,并代入到命令中而形成的命令。
你可以分别控制这些类的回显。

当在终端键入命令时,操作系统一般会显示用户键入的每个字符,因此该命令
在交互式会话中没有太大作用。设置显示宏命令以及处理后的命令在调试宏文件
时会很有用。
