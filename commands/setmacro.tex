\SACCMD{setmacro}
\label{cmd:setmacro}

\SACTitle{概要}
定义执行SAC宏文件时搜索的一系列目录

\SACTitle{语法}
\begin{SACSTX}
SETMACRO  [MORE] directory [directory ...]
\end{SACSTX}

\SACTitle{输入}
\begin{description}
\item [directory] 放置SAC宏文件的目录。可以是相对或绝对路径
\end{description}

\SACTitle{说明}
这个命令让你能够定义一系列执行宏文件时搜索的目录,最多可以定义100个。

当setmacro使用more选项时,指定的目录会追加到已经存在的列表的后面;若没有使用more选项,
则已经存在的列表将被新列表取代。

当执行macro命令时,SAC首先会搜索当前目录,若没有找到则搜索setmacro指定的目录,
若依然没有找到则在全局宏目录中寻找。
